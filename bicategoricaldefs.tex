%
% lire mccusker-harmer
%

\documentclass[10pt]{llncs}
%\documentclass[english]{lipics}
%\usepackage{prentcsmacro}
\usepackage[utf8]{inputenc}
\usepackage[T1]{fontenc}
\usepackage{aeguill}
\usepackage{stmaryrd}
\usepackage{amssymb}
\usepackage{shortcuts}
%\usepackage{proof}
%\usepackage{xypic}
\usepackage{graphicx}
\usepackage{graphics}
\usepackage[mathscr]{euscript}
%\usepackage{a4wide}
%\usepackage{times}

\usepackage{tikz-cd}
\usetikzlibrary{arrows}
\usetikzlibrary{matrix}
\newcounter{nodemaker}
\setcounter{nodemaker}{0}
\def\twocell#1#2{%
  \global\edef\mynodeone{twocell\arabic{nodemaker}}%
  \stepcounter{nodemaker}%
  \global\edef\mynodetwo{twocell\arabic{nodemaker}}%
  \stepcounter{nodemaker}%
  \ar[#1,phantom,shift left=3,""{name=\mynodeone}]%
  \ar[#1,phantom,shift right=3,""'{name=\mynodetwo}]%
  \ar[Rightarrow,from=\mynodeone,to=\mynodetwo,swap, "#2"]%
}
\usepackage[all]{xy}
%\CompileMatrices



%\addtolength{\textheight}{7.6em}
%\addtolength{\voffset}{-3.5em}

%\addtolength{\textwidth}{3.19em}
%\addtolength{\hoffset}{-1.6em}

%\addtolength{\textwidth}{4em}
%\addtolength{\hoffset}{-2em}

%\newtheorem{thm}{Theorem}
%\newtheorem{proposition}[theorem]{\textbf{Proposition}}
%\newtheorem{defn}{Definition}

%\renewcommand{\vxym}[1]{\vcenter{\xymatrix@C=3ex@R=3ex{#1}}}

\newcommand{\myparagraph}[1]{\paragraph{\textbf{#1}}}

% Asynchronous graphs
\newcommand{\AGrph}{\mathbf{AGrph}}
%\newcommand{\G}{\mathcal{G}}
\newcommand{\G}{G}
\newcommand{\moves}[1]{M_{#1}} % moves in a path
\newcommand{\setofmoves}[1]{M_{#1}} % moves of a ATS
\newcommand{\movesorder}[1]{\leq_{#1}} % order on moves
\newcommand{\lab}[1]{\ell({#1})}
\newcommand{\incompat}{\#}
\newcommand{\lattice}[1]{\mathcal{L}_{#1}}
%\newcommand{\tile}[1]{\diamond_{#1}}
\newcommand{\initpos}[1]{{*}_{#1}}
\newcommand{\hclass}[1]{[#1]} % class
\newcommand{\hcat}[1]{[#1]} % category of paths modulo homotopy
\newcommand{\htcat}[1]{\overline{#1}} % two category generated by an asynchronous graph
\newcommand{\paths}[1]{\mathrm{paths}(#1)} % paths starting from *
\newcommand{\pcompl}[1]{{#1}^\lightning} % path completion from complete positions
\newcommand{\subgraph}[1]{G_{#1}} % subgraph generated by a strategy
\newcommand{\concat}{\cdot} % concatenation of two paths
\newcommand{\hprefix}{\lesssim} % prefix modulo homotopy
\newcommand{\unfolding}[1]{T{#1}}

% Shortcuts
\newcommand{\homotopic}[1]{\sim_{#1}}
\newcommand{\qsim}{\quad\sim\quad}
\newcommand{\lbl}[2]{\ar@{}[#1]|-{#2}}
\newcommand{\hlbl}[1]{\lbl{#1}{\sim}}


% Strategies
%\newcommand{\interact}{\div}
\newcommand{\buffer}[1]{\mathrm{buf}_{#1}}
\newcommand{\Inno}{\mathbf{Inno}}
\newcommand{\alt}[1]{\mathrm{alt(#1)}}
\newcommand{\halting}[1]{\mathrm{halting}(#1)} % fixpoints

\newcommand{\todo}[1]{\textcolor{red}{TODO}: \underline{#1}}
% Closure operators
\newcommand{\clop}[1]{\mathrm{Cl}(#1)} % closure operator
\newcommand{\fixpoints}[1]{\mathrm{fix}(#1)} % fixpoints
\newcommand{\domain}[1]{\mathrm{dom}(#1)} % domain
\newcommand{\dynadom}[1]{\mathrm{dynadom}(#1)} % dynamic domain

% Positions
\newcommand{\pos}[1]{{#1}^\circ} % positions
\newcommand{\cpos}[1]{{#1}^\circ} % complete positions
\newcommand{\tcompl}[1]{{#1}^\top} % top-completion

% Boolean game
\newcommand{\booleangame}{\mathbb{B}} % notation for the boolean game
\newcommand{\questionmove}{\mathtt{q}} %question move
\newcommand{\answermove}{\mathtt{a}} %question move
\newcommand{\falsemove}{\mathtt{false}} %answ\er move --- false
\newcommand{\truemove}{\mathtt{true}} % answer move --- true
\newcommand{\questionposition}{q} % intermediate position q in the boolean game
\renewcommand{\root}{\ast} % root of the game
\newcommand{\tensorpair}[2]{#1\otimes #2} % pair of positions related by a tensor
\newcommand{\opposant}[1]{#1} % opponent colour of moves
\newcommand{\joueur}[1]{#1} % player colour of moves

\newcommand{\presheaf}[1]{\mathscr{P}#1}
\newcommand{\distributor}[1]{[#1]}

% Connectives
\newcommand{\before}{\varolessthan}
\newcommand{\after}{\varogreaterthan}

\newcommand{\Opt}{{\rm Opt}}
\newcommand{\Ppt}{{\rm Ppt}}
\newcommand{\Sw}{{\rm Switch}}
\newcommand{\Kl}{{\rm Kl}}
\newcommand{\restrict}{\hspace{-.05in}\upharpoonright \hspace{-.05in}}
\newcommand{\views}{\succeq}
\newcommand{\hp}{{\rm hp}}
\newcommand{\tensor}{\otimes}
\newcommand{\limp}{\multimap}
\newcommand{\aimp}{\rule[.042in]{.2in}{.003in}\hspace{-.12in}\boxplus} 


\newcommand{\questiontt}{\mathtt{q}}
\newcommand{\answertt}{\mathtt{a}}
\newcommand{\Unit}{\mathbf{C}}
\newcommand{\choice}{\mathtt{choice}}
\newcommand{\Pview}[1]{\ulcorner#1\urcorner}
\newcommand{\plays}[1]{P_{#1}}
\newcommand{\oracle}{\mathbf{oracle}}

\newcommand{\refcat}[1]{\PStrategy{#1}}

\newcommand{\refcatfun}{\PStrategyalone}
\newcommand{\oddcart}{\OTransduction}
\newcommand{\slender}[1]{#1^p}
\newcommand{\refin}[1]{\mathcal{T}_{#1}}
\newcommand{\image}[1]{\mathsf{image}(#1)}
%\newcommand{\comprehensionalone}{Comprehend}
\newcommand{\imagealone}{\mathsf{image}}
\newcommand{\imagestrat}[1]{\image{#1}}
\newcommand{\runtree}[1]{\mathcal{R}_{#1}}

%\newcommand{\Games}{\mathscr{G}}
\newcommand{\Games}{\mathscr{G}}
\newcommand{\BackForth}{\mathscr{BF}}
\newcommand{\BackForthplus}{\mathscr{BF}^{+}}
\newcommand{\BackForthminus}{\mathscr{BF}^{-}}
\newcommand{\SheafGames}{\mathscr{S}}
\newcommand{\Transduction}{\mathscr{T}}
\newcommand{\OTransduction}{\mathscr{T}_{O}}
\newcommand{\PTransduction}{\mathscr{T}_{P}}
%\newcommand{\PStrategy}[1]{\mathscr{S}_{#1}}
\newcommand{\PStrategy}[1]{\mathscr{P}({#1})}
\newcommand{\PStrategyalone}{\mathscr{P}}
\newcommand{\comp}[1]{\mathbf{comp}_{#1}}
%\newcommand{\projectionpaire}[1]{{\mathbf{\pi}_{#1}}}
%\newcommand{\projectionpaire}[1]{\mathcal{\pi}_{#1}}
\newcommand{\projectionpaire}[1]{\pi^2_{#1}}
\newcommand{\projectionpairealone}{\pi^2}
\newcommand{\projectionP}[1]{(\pi_P)_{#1}}
\newcommand{\projectionPalone}{\pi_P}

%\newcommand{\gameA}[1]{A_{[#1]}}
%\newcommand{\projectionpi}[1]{\pi_{[#1]}}
%\newcommand{\projectionpialone}{\pi}
\newcommand{\gameA}[1]{A_{#1}}
\newcommand{\gameAalone}{A}
\newcommand{\gameB}[1]{B_{#1}}
\newcommand{\gameBalone}{B}
\newcommand{\supportS}[1]{S_{#1}}
\newcommand{\supportSalone}{S}
%\newcommand{\support}[1]{\mathsf{support}(#1)}
\newcommand{\support}[2]{\{#1\,|\,#2\}}
\newcommand{\inclusion}[2]{\mathsf{supp}_{\,#2}}
%\newcommand{\inclusion}[2]{\mathsf{in}_{\support{#1}{#2}}}
%\newcommand{\supportalone}{\mathsf{support}}
\newcommand{\supportalone}[1]{\support{#1}{-}}
\newcommand{\projectionpi}[1]{\pi_{#1}}
\newcommand{\projectionpialone}{\pi}
\newcommand{\strategysigma}[1]{\sigma_{#1}}
\newcommand{\strategysigmaalone}{\sigma}
\newcommand{\strategytau}[1]{\tau_{#1}}
\newcommand{\strategytaualone}{\tau}

%\newcommand{\Grothendieck}[1]{\mathbf{Groth}({#1})}
%\newcommand{\Grothendieck}[1]{\int {#1}}
\newcommand{\Grothendieck}[1]{\textbf{tree}({#1})}
%\newcommand{\restricted}[1]{\widetilde{#1}}
\newcommand{\restricted}[1]{#1_P}
\newcommand{\exponential}[1]{{!{#1}}}
%
%\newcommand{\Rel}{\mathbf{Rel}}
\newcommand{\Dist}{\mathbf{Dist}}
\newcommand{\Schedule}{\Upsilon}

\newcommand{\seqcomp}{\centerdot}
\newcommand{\push}[2]{\mathbf{push}_{#1}#2}

\title{Categorical combinatorics\\ for non deterministic strategies on simple games}
 % \title{Categorical combinatorics\\ for non deterministic strategies}
\author{Cl\'ement Jacq and Paul-Andr\'e Melli\`es}
\institute{Institut de Recherche en Informatique Fondamentale, Universit\'e Paris Diderot}
 % \author{Cl\'ement Jacq\hspace{.2em}}
%  \author{\hspace{.2em} Paul-Andr\'e Melli\`es}
 % \affil{Institut de Recherche en Informatique Fondamentale (IRIF)\\
 %Universit\'e Paris Diderot}
%\title{Categorical combinatorics
%for non deterministic strategies
%on simple games}
%\author{Cl\'ement Jacq and Paul-Andr\'e Melli\`es}
%\affil{Institut de Recherche en Informatique Fondamentale, Universit\'e Paris Diderot}

\begin{document}
\section{Some bicategorical definitions} 
In this section, we recall a few definitions required by our bicategorical setting. 

\begin{definition}\label{definition/bicategory}
A bicategory $\mathcal{C}$ consists of :
\begin{itemize}
\item A collections of objects $A,B,C$.
\item For each pair of objects $A,B$, a category $\mathcal{C}(A,B)$ whose objects are called morphisms or $1-$cells and whose morphisms are called $2$-morphisms or $2-$cells.
\item For each object $A$, a distinguished $1-$cell $id_A\in \mathcal{C}(A,A)$ called the identity morphism.
\item For each triple of objects $A,B,C$ a functor $$\circ : \mathcal{C}(A,B) \times \mathcal{C}(B,C) \rightarrow \mathcal{C}(A,C)$$ called horizontal composition.
\item For each pair of objects $A,B$, two natural isomorphisms called the left and right unitors: $$l:  id_A \circ - \Rightarrow - :\mathcal{C}(A,B)\rightarrow \mathcal{C}(A,B) \text{ and } r:    - \circ id_B \Rightarrow - :\mathcal{C}(A,B)\rightarrow \mathcal{C}(A,B) $$
\item For each quadruple of objects $A,B,C,D$ a natural isomorphism called the associator $$a: (- \circ -) \circ - \Rightarrow - \circ ( - \circ -): \mathcal{C}(A,B) \times \mathcal{C}(B,C) \times \mathcal{C}(C,D)  \rightarrow \mathcal{C}(A,D)$$
\end{itemize}
such that the following diagrams commute for any object $A,~B,~C,~D,~E$ of $\mathcal{C}$ and $f,~g,~h,~i$ objects of $\mathcal{C}(A,B),~\mathcal{C}(B,C),~\mathcal{C}(C,D),~\mathcal{C}(D,E)$ respectively :
$$\xymatrix @-1.2pc {
((f \circ g) \circ h) \circ i
\ar[rrrr]_-{a(f,g,h) \circ Id_i}
\ar[dd]_-{a(f\circ g, h,i)}
&&&&
(f \circ (g \circ h)) \circ i
\ar[dd]_-{a(f, g \circ h, i)}
\\
\\
(f \circ g) \circ (h \circ i)
\ar[ddrr]_-{a(f,g,h \circ i)}
&&&&
f \circ ((g \circ h) \circ i)
\ar[ddll]^-{ Id_f \circ a(g,h,i)}
\\
\\
&&
f \circ(g \circ (h \circ i))
}
$$




$$
\xymatrix @-1.2pc {
(f \circ Id_B) \circ g
\ar[rrrr]_-{a(f, id_B, g)}
\ar[ddrr]_-{r(f) \circ Id(g) }
&&&&
f \circ (Id_B \circ g)
\ar[ddll]^-{Id_f \circ l(g) }
\\
\\
&&
f \circ g
}
$$

\end{definition}
\begin{definition}\label{definition/pseudofunctor}
Let $\mathcal{C}, \mathcal{D}$ be two bicategories. A pseudofunctor $F:\mathcal{C}\rightarrow \mathcal{D}$ is given by :
\begin{itemize}
\item For each object $A$ of $\mathcal{C}$, an object $F(A)$ of $\mathcal{D}$.
\item For each hom-category $\mathcal{C}(A,B)$ in $\mathcal{C}$, a functor $$F(A,B): \mathcal{C}(A,B) \rightarrow \mathcal{D}(F(A),F(B))$$
\item For each object $A$ of $\mathcal{C}$, an invertible $2$-cell $$F_{id_A}: id_{F(A)} \Rightarrow F(A,B)(id_A)$$
\item For each triple of objects $A,B,C$ of $\mathcal{C}$, a natural isomorphism $\phi$ whose elements are:$$\phi_{f,g} :F(f) \circ F(g) \Rightarrow F(f \circ g)$$ for $f,g$ objects of $\mathcal{C}(A,B), \mathcal{C}(B,C)$ respectively
\end{itemize}
such that, for any object $A,~B,~C,~D$ of $\mathcal{C}$, any objects $f,~g,~h$ of $\mathcal{C}(A,B),~\mathcal{C}(B,C),~\mathcal{C}(C,D)$ respectively, the following diagrams commute :

$$\xymatrix @-1.2pc {
F(f) \circ (F(g) \circ F(h))
\ar[dd]_-{Id_{F(f)} \circ \phi_{g,h} }
&&&&
(F(f) \circ F(g)) \circ F(h)
\ar[llll]_-{a(F(f),F(g),F(h)}
\ar[dd]_-{\phi_{f,g} \circ Id_{F(h)}}
\\
\\
F(f) \circ F(g \circ h)
\ar[dd]_-{\phi_{f,g \circ h} }
&&&&
F(f \circ g) \circ F(h)
\ar[dd]_-{\phi_{f \circ g, h} }
\\
\\
F(f\circ(g\circ h))
&&&&
F((f \circ g) \circ h)
\ar[llll]_-{F(a(f,g,h))}
}
$$
$$
\xymatrix @-1.2pc {
F(f)
&&&
F(f \circ id_B)
\ar[lll]_-{F(r(f))}
\\
\\
F(f) \circ id_{F(B)}
\ar[uu]_-{r(F(f))}
\ar[rrr]_-{ Id_{F(f)}  \circ F_{id_B} }
&&&
F(f) \circ F(id_B)
\ar[uu]_-{\phi_{f, id_B}}
}
$$

$$
\xymatrix @-1.2pc {
F(f)
&&&
F(id_A \circ f )
\ar[lll]_-{F(l(f))}
\\
\\
 id_{F(A)} \circ F(f) 
\ar[uu]_-{l(F(f))}
\ar[rrr]_-{ F_{id_A}  \circ Id_{F(f)}  }
&&&
 F(id_A) \circ F(f) 
\ar[uu]_-{\phi_{id_A, f}}
}
$$
\end{definition}

\begin{definition}
Let $\mathcal{C}, \mathcal{D}$ be two bicategories, and $F,G:\mathcal{C} \rightarrow \mathcal{D}$ two pseudofunctors. A pseudo-natural transformation $\gamma: F \Rightarrow G$ is given by :
\begin{itemize}
\item for every object $A$ of $\mathcal{C}$, a $1-$cell $\gamma_A: F(A) \rightarrow G(A)$
\item for every pair of objects $A,B$ of $\mathcal{C}$ and every $1-$cell $f$ of $\mathcal{C}(A,B)$, an invertible $2-$cell $\gamma_f$:
$$\begin{tikzcd}
  F(A) 
  \arrow{r}{\gamma_A}
  \arrow{d}[swap]{F(f)}
  &
   G(A) 
   \arrow{d}{G(f)}
  \\
  F(B) 
  \arrow{r}[swap]{\gamma_B}
  &
  \twocell{ul}{\gamma_f}
  G(B)
\end{tikzcd}$$

\end{itemize} 
such that the following properties are verified : 
\begin{itemize}
\item Naturality : For every $2-$cell $\tau:f \Rightarrow g : A \rightarrow B$, the $2-$cells associated to the following pasting diagrams are equal : 

$$\begin{tikzcd}
  F(A) 
  \arrow{rr}{\gamma_A}
  \arrow{dd}[swap]{F(f)}
  &&
   G(A) 
   \arrow[bend right]{dd}[swap]{G(f)} [name=U]{}
   \arrow[bend left]{dd}{G(g)}[name=D]{}
   \arrow[Rightarrow,from=U, to=D,"G(\tau)"]
   &&
  F(A) 
  \arrow{rr}{\gamma_A}
   \arrow[bend right]{dd}[swap]{F(f)} [name=E]{}
   \arrow[bend left]{dd}{F(g)}[name=R]{}
  \arrow[Rightarrow,from=E, to=R,"F(\tau)"]
   &&
   G(A) 
   \arrow{dd}{G(g)}
  \\
  &&&=&&&
  \\
  F(B) 
  \arrow{rr}[swap]{\gamma_B}
  &&
  \twocell{uull}{\gamma_f}
  G(B)
 &&
  F(B) 
  \arrow{rr}[swap]{\gamma_B}
  &&
  \twocell{uull}{\gamma_g}
  G(B)
  \end{tikzcd}$$

\item Unitality :For every object $A$ of $\mathcal{C}$, the $2-$cells associated to the following pasting diagrams are equal : 
$$\begin{tikzcd}
  F(A) 
  \arrow{rr}{\gamma_A}
  \arrow{dd}[swap]{Id_{F(A)}}
  &&
   G(A) 
   \arrow[bend right]{dd}[swap]{Id_{G(A)}} [name=U]{}
   \arrow[bend left]{dd}{G(Id_A)}[name=D]{}
   \arrow[Rightarrow,from=U, to=D,"G_A"]
   &&
  F(A) 
  \arrow{rrr}{\gamma_A}
   \arrow[bend right]{dd}[swap]{Id_{F(A)}} [name=E]{}
   \arrow[bend left]{dd}{F(Id_A)}[name=R]{}
  \arrow[Rightarrow,from=E, to=R,"F_A"]
   &&&
   G(A) 
   \arrow{dd}{G(Id_A)}
  \\
  &\equiv&&=&&&
  \\
  F(A) 
  \arrow{rr}[swap]{\gamma_A}
  &&
%  \twocell{uull}{\gamma_f}
  G(A)
 &&
  F(A) 
  \arrow{rrr}[swap]{\gamma_A}
  &&&
  \twocell{uulll}{\gamma_{Id_A}}
  G(A)
  \end{tikzcd}$$
\item Compositionality : for every triple of objects $A,B,C$ of $\mathcal{C}$ and every pair of $1-$cells $f,g$ of $\mathcal{C}(A,B),~\mathcal{C}(B,C)$ respectively, the $2-$ cells associated to the following pasting diagrams are equal : 

$$\begin{tikzcd}
F(A)
\arrow{rr}{F(f)}
\arrow[bend right]{rrrr}[swap]{F(g \circ f)} [name=E]{}
\arrow{dd}{\gamma_A}
&&
F(B)
\arrow{rr}{F(g)}
\arrow[Rightarrow,to=E,"F_{g,f}"]
&&
F(C)
\arrow{dd}{\gamma_C}
&&
F(A)
\arrow{rr}{F(f)}
\twocell{rrd}{\gamma_{f}}
\arrow{dd}{\gamma_A}
&&
F(B)
\arrow{rr}{F(g)}
\arrow{d}{\gamma_B}
\twocell{rrd}{\gamma_{g}}
&&
F(C)
\arrow{dd}{\gamma_C}
\\
&&&&&=&&&
G(B)
\arrow[Rightarrow,shorten >=0.3cm,shorten <=0.3cm]{d}{G_{g,f}}
\arrow{rrd}{G(g)}
&&{}
\\
G(A)
\arrow{rrrr}{G(g \circ f)}
&&
&&
G(C)
&&
G(A)
\arrow{rru}{G(f)}
\arrow{rrrr}[swap]{G(g \circ f)}
&&
{}
&&
G(C)
\end{tikzcd}$$
\end{itemize}
\end{definition}

\begin{definition}
Let $\gamma,\delta:F \Rightarrow G: \mathcal{C} \rightarrow \mathcal{D}$ be two pseudo-natural transformations., a modification $m: \gamma \Rrightarrow \delta$ is given by a $2-$cell $m_A: \gamma_A \Rightarrow \delta_A$ for every object $A$ of $\mathcal{C}$ such that for every $f:A\rightarrow B$ in $\mathcal{C}$, we have : 
$$\begin{tikzcd}
F(A)
  \arrow[bend right]{rr}[swap]{\gamma_A} [name=E]{}
   \arrow[bend left]{rr}{\delta_A}[name=R]{}
  \arrow[Rightarrow, shorten >=0.2cm, shorten <=0.2cm,from=E, to=R,"m_A"]
  \arrow{ddd}{F(f)}
 &&
G(A)
\arrow{ddd}{G(f)}
&&
F(A)
\arrow{rr}{\delta_A}
\arrow{ddd}{F(f)}
&{}&
G(A)
\arrow{ddd}{G(f)}
\\
{}&{}&{}
{}&{}&{}
{}&{}&{}
\\
\\
F(B)
\arrow{rr}{\gamma_B}
 &&
G(B)
\twocell{uull}{\gamma_f}
&&
F(B)
  \arrow[bend right]{rr}[swap]{\gamma_B} [name=T]{}
   \arrow[bend left]{rr}{\delta_B}[name=Y]{}
  \arrow[Rightarrow, shorten >=0.2cm, shorten <=0.2cm,from=T, to=Y,"m_B"]
&&
G(B)
\twocell{uuull}{\delta_f}
\end{tikzcd}$$

\end{definition}

\begin{definition}
Let $A,B$ be two objects in a bicategory $\mathcal{C}$. An equivalence from $A$ to $B$ is given by :
\begin{itemize}
\item a pair of $1-$cells $f:A\rightarrow B$ and $g:B \rightarrow A$.
\item a pair of invertible $2-$cells $e:id_A \Rightarrow g \circ f$ and $e': id_B \Rightarrow f \circ g$.
\end{itemize}
We say that $f$ is an equivalence if such $g,e,e'$ exist. 
\end{definition}

\begin{definition}
A monoidal bicategory $\mathcal{C}$ is a bicategory equipped with : 
\begin{itemize}
\item a unit object $I$.
\item a pseudo-functor $\otimes : \mathcal{C} \times \mathcal{C} \rightarrow \mathcal{C}$
\item three pseudo-natural transformations $a,l,r$ whose components are equivalences and given by : 
$$a_{A,B,C}: (A \otimes B) \otimes C \rightarrow A \otimes (B\otimes C)$$
$$l_A : I \otimes A \rightarrow A$$
$$r_A : A \otimes I \rightarrow A$$
\item four invertible modifications $\pi, \mu, L, R$ whose components are given by :
$$\begin{tikzcd}
((A \otimes B) \otimes C) \otimes D
\arrow{r}{a_{A \otimes B,C,D}}
\arrow{d}{a_{A,B,C} \otimes id_D}
&
(A \otimes B) \otimes (C \otimes D)
\arrow{r}{a_{A,B,C\otimes D}}
\arrow[Rightarrow,shorten >=0.3cm,shorten <=0.3cm]{d}{\pi_{A,B,C,D}}
&
A \otimes ( B \otimes (C \otimes D))
\\
(A \otimes (B \otimes C)) \otimes D
\arrow{rr}{a_{A, B\otimes C, D}}
&{}&
A \otimes ((B \otimes C) \otimes D)
\arrow{u}{id_A \otimes a_{B,C,D}}
\end{tikzcd}$$
$$\begin{tikzcd}
(A \otimes I) \otimes C
\arrow{rd}[swap]{r_A \otimes id_C} [name=T]{}
\arrow{rr}{a_{A,I,C}}
&&
A \otimes (I \otimes C)
\arrow{ld}{id_A \otimes l_C} [swap,name=V]{}
 \arrow[Rightarrow, shorten >=0.7cm, shorten <=0.7cm,from=T, to=V,"\mu_{A,C}"]
\\
&
A \otimes C
&
\end{tikzcd}$$
$$\begin{tikzcd}
(I \otimes B) \otimes C
\arrow{rd}[swap]{l_B \otimes id_C} [name=T]{}
\arrow{rr}{a_{I,B,C}}
&&
I \otimes (B \otimes C)
\arrow{ld}{l_{B\otimes C}} [swap,name=V]{}
 \arrow[Rightarrow, shorten >=0.7cm, shorten <=0.7cm,from=T, to=V,"L_{B,C}"]
 &
 (A \otimes B) \otimes I
\arrow{rd}[swap]{r_{A \otimes B}} [name=Y]{}
\arrow{rr}{a_{A,B,I}}
&&
A \otimes (B \otimes I)
\arrow{ld}{id_A \otimes r_B} [swap,name=U]{}
 \arrow[Rightarrow, shorten >=0.7cm, shorten <=0.7cm,from=Y, to=U,"R_{A,B}"]
\\
&
B \otimes C
&
&
&
A \otimes B
&
\end{tikzcd}$$
\end{itemize}
such that the following conditions are verified :
\begin{itemize}
\item Associativity : For all $A,B,C,D,E$ objects of $\mathcal{C}$, the pasting diagram\\
$
\scalebox{0.4}{
\begin{tikzcd}[ampersand replacement=\&]
\&
(((A \otimes B) \otimes C) \otimes D) \otimes E
\arrow{rr}{a_{(A\otimes B) \otimes C,D,E}}
\arrow{rdd}{a_{A\otimes B,C,D} \otimes id_E}
\arrow{ld}{(a_{A,B,C} \otimes id_D) \otimes id_E}
\&\&
((A \otimes B) \otimes C) \otimes (D \otimes E)
\arrow{rr}{a_{A\otimes B,C,D\otimes E}}
\arrow[Rightarrow, shorten >=0.7cm, shorten <=1.2cm]{dd}{\pi_{A\otimes B,C,D,E} }
\&\&
(A \otimes B) \otimes (C \otimes (D \otimes E))
\arrow{rdd}{a_{A,B,C\otimes (D \otimes E)}}
\arrow[Rightarrow, shorten >=1cm, shorten <=3cm]{dddd}[near end]{a_{id_A,id_B,a_{C,D,E}}}
\&
\\
((A \otimes (B \otimes C)) \otimes D) \otimes E
\arrow{dd}{a_{A,B\otimes C,D} \otimes id_E}
\\
{}\&\&
((A \otimes B) \otimes (C \otimes D)) \otimes E
\arrow[Rightarrow, shorten >=3.2cm, shorten <=3.2cm]{ll}{\pi_{A,B,C,D} \otimes Id_{id_E}}
\arrow{ldd}{a_{A,B,C\otimes D} \otimes id_E}
\arrow{rr}{a_{A \otimes B, C \otimes D, E}}
\&{}\&
(A \otimes B) \otimes ((C \otimes D) \otimes E)
\arrow{rdd}[swap]{a_{A,B,(C \otimes D) \otimes E}}
\arrow[bend left=0]{ruu}{id_{A\otimes B} \otimes a_{C,D,E}}[name=T]{}
\arrow[bend right]{ruu}[swap]{(id_{A}\otimes id_{B}) \otimes a_{C,D,E}}[name=Y]{}
\arrow[Rightarrow, shorten >=1.8cm, shorten <=1.8cm]{ddl}{\pi_{A,B,C\otimes D,E}}
\arrow[Rightarrow, shorten >=0.4cm, shorten <=0.4cm,from=T, to=Y,"\otimes_{id_A,id_B} \otimes Id_{a_{C,D,E}}"]
\&\&
A \otimes (B \otimes (C \otimes (D \otimes E)))
\\
(A \otimes ((B \otimes C) \otimes D)) \otimes E
\arrow{rd}{(id_A \otimes a_{B,C,D}) \otimes id_E}
\\
\&
(A \otimes (B \otimes (C \otimes D))) \otimes E
\arrow{rr}{a_{A,B \otimes (C \otimes D),E}}
\&\&
A \otimes ((B \otimes (C \otimes D)) \otimes E)
\arrow{rr}{id_A \otimes a_{B, C\otimes D, E}}
\&\&
A \otimes (B \otimes ((C \otimes D) \otimes E))
\arrow{ruu}[swap]{id_A \otimes ( id_B \otimes a_{C,D,E})}
\&
\end{tikzcd}}
$ must be equal to the pasting diagram \\
$\scalebox{0.4}{
\begin{tikzcd}[ampersand replacement=\&]
\&
(((A \otimes B) \otimes C) \otimes D) \otimes E
\arrow{rr}{a_{(A\otimes B) \otimes C,D,E}}
\arrow{ld}[swap]{(a_{A,B,C} \otimes id_D) \otimes id_E}
\&{}
\arrow[Rightarrow, shorten >=1.1cm, shorten <=3.3cm]{dl}[swap, near end]{a_{a_{A,B,C},id_D,id_E}^{-1}}
\&
((A \otimes B) \otimes C) \otimes (D \otimes E)
\arrow{rr}{a_{A\otimes B,C,D\otimes E}}
\arrow[bend right=10]{dl}[swap, near end]{a_{A,B,C} \otimes (id_{D}  \otimes id_{E})}[name=T]{}
\arrow[bend left=10]{dl}[near end]{a_{A,B,C} \otimes id_{D \otimes E}}[name=Y]{}
\arrow[Rightarrow, shorten >=0.4cm, shorten <=0.4cm,from=Y, to=T,swap,"Id_{a_{A,B,C}} \otimes~\otimes_{id_D,id_E}"]
\&\&
(A \otimes B) \otimes (C \otimes (D \otimes E))
\arrow{rdd}{a_{A,B,C\otimes (D \otimes E)}}
\arrow[Rightarrow, shorten >=1.9cm, shorten <=1.9cm]{ddl}{\pi_{A,B,C,D\otimes E}}
\&
\\
((A \otimes (B \otimes C)) \otimes D) \otimes E
\arrow{dd}{a_{A,B\otimes C,D} \otimes id_E}
\arrow{rr}[swap]{a_{A\otimes(B\otimes C),D,E}}
\&{}\&
(A \otimes (B \otimes C)) \otimes (D \otimes E)
\arrow{rrd}{a_{A,B\otimes C, D\otimes E}}
\arrow[Rightarrow, shorten >=1.9cm, shorten <=1.9cm]{ddl}{\pi_{A,B\otimes C,D, E}}
\\
\&\&
\&\&
A \otimes ((B \otimes C) \otimes (D \otimes E))
\arrow{rr}{id_A \otimes a_{B,C,D\otimes E}}
\arrow[Rightarrow, shorten >=0.9cm, shorten <=0.9cm]{dd}{Id_{id_A} \otimes \pi_{,B,C,D, E}}
\&\&
A \otimes (B \otimes (C \otimes (D \otimes E)))
\\
(A \otimes ((B \otimes C) \otimes D)) \otimes E
\arrow{rd}{(id_A \otimes a_{B,C,D}) \otimes id_E}
\arrow{rr}{a_{A,(B \otimes C) \otimes D, E}}
\&{}\&
A \otimes (((B \otimes C) \otimes D) \otimes E)
\arrow{rru}{id_A \otimes a_{B \otimes C,D,E}}
\arrow{dr}{id_A \otimes ( a_{B,C,D} \otimes id_E)}
\arrow[Rightarrow, shorten >=1.1cm, shorten <=1.1cm]{dl}[swap]{a_{id_A,a_{B,C,D},id_E}^{-1}}
\\
\&
(A \otimes (B \otimes (C \otimes D))) \otimes E
\arrow{rr}{a_{A,B \otimes (C \otimes D),E}}
\&\&
A \otimes ((B \otimes (C \otimes D)) \otimes E)
\arrow{rr}{id_A \otimes a_{B, C\otimes D, E}}
\&{}\&
A \otimes (B \otimes ((C \otimes D) \otimes E))
\arrow{ruu}{id_A \otimes ( id_B \otimes a_{C,D,E})}
\&
\end{tikzcd}}
$
\\
\item For all $A,B,C$ objects of $\mathcal{C}$, the pasting diagram
$$\begin{tikzcd}[column sep = small]
&
( A \otimes B) \otimes C
\arrow{rr}{a_{A,B,C}}
\arrow[Rightarrow, shorten >=1.1cm, shorten <=0.5cm]{rd}[swap]{ a_{r_A,id_B,id_C}}
&
&
A \otimes (B \otimes C)
\arrow[Rightarrow, shorten >=2.8cm, shorten <=0.4cm]{ddr}[near start]{ \mu_{A,B\otimes C}}
&
\\
((A \otimes I) \otimes B) \otimes C
\arrow{ru}{(r_A \otimes id_B) \otimes id_C}
\arrow{rd}[swap]{a_{A,I,B} \otimes id_C}
\arrow{rr}[swap]{a_{A\otimes I,B,C}}
&&
(A \otimes I) \otimes (B \otimes C)
\arrow{rr}[swap]{a_{A,I,B\otimes C}}
\arrow[bend left=18]{ru}[near start]{r_A \otimes id_B \otimes id_C}[name=Y,near start]{}
\arrow[bend right=12]{ru}[swap]{r_A \otimes id_{B \otimes C}}[name=T, near start]{}
\arrow[Rightarrow, shorten >=0.0cm, shorten <=0.5cm,from=Y, to=T,"Id_{r_A} \otimes~\otimes_{id_B,id_C}^{-1}"]
\arrow[Rightarrow, shorten >=0.2cm, shorten <=0.1cm]{d}{ \pi_{A,I,B,C}}
&{}&
A \otimes (I \otimes ( B \otimes C))
\arrow{lu}[swap]{id_A \otimes l_{B \otimes C}}
\\
&
(A \otimes ( I \otimes B)) \otimes C
\arrow{rr}[swap]{a_{A,I\otimes B,C}}
&{}
&
A \otimes (( I \otimes B) \otimes C)
\arrow{ru}[swap]{id_A \otimes a_{I,B,C}}
&{}
\end{tikzcd}
$$ must be equal to the pasting diagram 
$$\begin{tikzcd}[column sep = small]
&
( A \otimes B) \otimes C
\arrow{rr}{a_{A,B,C}}
&
&
A \otimes (B \otimes C)
&
\\
((A \otimes I) \otimes B) \otimes C
\arrow{ru}{(r_A \otimes id_B) \otimes id_C}[name=T]{}
\arrow{rd}[swap]{a_{A,I,B} \otimes id_C}
&{}&
&{}
\arrow[Rightarrow, shorten >=0.2cm, shorten <=1.1cm]{r}{Id_{id_A} \otimes L_{B,C}}
&
A \otimes (I \otimes ( B \otimes C))
\arrow{lu}[swap]{id_A \otimes l_{B \otimes C}}
\\
&
(A \otimes ( I \otimes B)) \otimes C
\arrow{rr}[swap]{a_{A,I\otimes B,C}}
\arrow{uu}[swap,near start]{(id_A \otimes l_B) \otimes id_C}[name=Y, near start]{}
\arrow[Rightarrow, shorten >=0.2cm, shorten <=1.1cm,from=T,to=Y]{}[swap, near end]{\mu_{A,B} \otimes Id_{id_C}}
&
&
A \otimes (( I \otimes B) \otimes C)
\arrow{ru}[swap]{id_A \otimes a_{I,B,C}}
\arrow{uu}[swap,near start]{id_A \otimes( l_B \otimes id_C)}[name=Z, near end]{}
\arrow[Rightarrow, shorten >=1.7cm, shorten <=1.6cm,from=Y,to=Z]{}[near end]{a_{id_A,l_B,id_C}}
&
\end{tikzcd}
$$
\item For all $A,B,C$ objects of $\mathcal{C}$, the pasting diagram
$$\begin{tikzcd}[column sep = small]
{}
&
( A \otimes B) \otimes C
\arrow{rr}{a_{A,B,C}}
\arrow[Rightarrow, shorten >=1.5cm, shorten <=5.9cm]{rrrd}[near end]{ a_{id_A,id_B,l_C}}
&
&
A \otimes (B \otimes C)
&
\\
((A \otimes B) \otimes I) \otimes C
\arrow{ru}{r_{A \otimes B} \otimes id_C}[name=X]{}
\arrow{rd}[swap]{a_{A,B,I} \otimes id_C}
\arrow{rr}[swap]{a_{A\otimes B,I,C}}[name=Z]{}
&&
(A \otimes B) \otimes (I \otimes C)
\arrow{rr}[swap]{a_{A,B,I\otimes C}}
\arrow[bend left=12]{lu}[near start]{id_{A \otimes B} \otimes l_C}[name=Y,near start]{}
\arrow[bend right=18]{lu}[swap,near start]{(id_A  \otimes id_B )\otimes l_C}[name=T, near start]{}
\arrow[Rightarrow, shorten >=0.0cm, shorten <=0.5cm,from=Y, to=T,near end,"\otimes_{id_A,id_B} \otimes Id_{l_c}"]
\arrow[Rightarrow, shorten >=0.2cm, shorten <=0.1cm]{d}{ \pi_{A,B,I,C}}
&{}&
A \otimes (B \otimes ( I \otimes C))
\arrow{lu}[swap]{id_A \otimes (id_B \otimes  l_{C})}
\\
&
{(A \otimes ( B \otimes I)) \otimes C}
\arrow[Rightarrow, shorten >=0.7cm, shorten <=0.5cm,from=X,to=Z]{}[]{ \mu_{A\otimes B, C}}
\arrow{rr}[swap]{a_{A,B\otimes I,C}}
&{}
&
A \otimes (( B \otimes I) \otimes C)
\arrow{ru}[swap]{id_A \otimes a_{B,I,C}}
&{}
\end{tikzcd}
$$ must be equal to the pasting diagram 
$$\begin{tikzcd}[column sep = small]
&
( A \otimes B) \otimes C
\arrow{rr}{a_{A,B,C}}
&
&
A \otimes (B \otimes C)
&
\\
((A \otimes B) \otimes I) \otimes C
\arrow{ru}{r_{A \otimes B} \otimes id_C}[name=T]{}
\arrow{rd}[swap]{a_{A,B,I} \otimes id_C}
&{}&
&{}
\arrow[Rightarrow, shorten >=0.2cm, shorten <=1.1cm]{r}{Id_{id_A} \otimes \mu_{B,C}}
&
A \otimes (B \otimes ( I \otimes C))
\arrow{lu}[swap]{id_A \otimes (id_B \otimes l_C)}
\\
&
(A \otimes ( B \otimes I)) \otimes C
\arrow{rr}[swap]{a_{A,B\otimes I,C}}
\arrow{uu}[swap,near start]{(id_A \otimes r_B) \otimes id_C}[name=Y, near start]{}
\arrow[Rightarrow, shorten >=0.2cm, shorten <=1.1cm,from=T,to=Y]{}[swap, near end]{R_{A,B} \otimes Id_{id_C}}
&
&
A \otimes (( B \otimes I) \otimes C)
\arrow{ru}[swap]{id_A \otimes a_{B,I,C}}
\arrow{uu}[swap,near start]{id_A \otimes( r_B \otimes id_C)}[name=Z, near end]{}
\arrow[Rightarrow, shorten >=1.7cm, shorten <=1.6cm,from=Y,to=Z]{}[near end]{a_{id_A,r_B,id_C}}
&
\end{tikzcd}
$$
\end{itemize}

\end{definition}


\begin{definition}
Let $\mathcal{C}, \mathcal{D}$ be two monoidal bicategories. A monoidal pseudofunctor $F : \mathcal{C} \rightarrow \mathcal{D}$ is a pseudofunctor equipped with:
\begin{itemize}
\item a $1-$cell $F_{I}^{\otimes}: I \rightarrow F(I)$
\item a pseudo-natural transformation $F^\otimes$ whose components are of the form : $$F_{A,B}^\otimes : F(A) \otimes F(B) \rightarrow F(A \otimes B)$$
\item three invertible modifications $F^a, F^l, F^r$ whose components are of the form : 
$$
\begin{tikzcd}
(F(A) \otimes F(B)) \otimes F(C)
\arrow{rr}{a_{F(A),F(B),F(C)}}
\arrow{d}{F_{A,B}^{\otimes} \otimes id_{F(A)}}
&&
F(A) \otimes (F(B) \otimes F(C))
\arrow{d}{id_{F(A)} \otimes F_{B,C}^\otimes}
\\
F(A\otimes B) \otimes F(C)
\arrow{d}{F_{A\otimes B, C}^\otimes}
&&
F(A) \otimes F(B \otimes C)
\arrow{d}{F_{A,B\otimes C}^\otimes}
\\
F((A \otimes B) \otimes C)
\arrow{rr}{F(a_{A,B,C})}
\arrow[Rightarrow, shorten >=2.2cm, shorten <=2.1cm]{rruu}[swap]{F^{a}_{A,B,C}}
&&
F(A \otimes (B \otimes C))
\end{tikzcd}$$

$$\begin{tikzcd}
I \otimes F(A)
\arrow{r}{F_{I}^\otimes \otimes id_{F(A)}}
\arrow{d}{l_{F(A)}}[name=A]{}
&
F(I) \otimes F(A)
\arrow{d}{F_{I,A}^\otimes}[name=B]{}
\arrow[Rightarrow, shorten >=1.2cm, shorten <=1.1cm,from=A,to=B]{}[]{F_{A}^l}
&&
F(A) \otimes I
\arrow{r}{id_{F(A)} \otimes F_{i}^\otimes}
\arrow{d}{r_{F(A)}}[name=C]{}
&
F(A) \otimes F(I)
\arrow{d}{F_{A,I}^\otimes}[name=D]{}
\arrow[Rightarrow, shorten >=1.2cm, shorten <=1.1cm,from=C,to=D]{}[near end]{F_{A}^r}
\\
F(A)
&
F(I \otimes A)
\arrow{l}{F(l_A)}
&&
F(A)
&
F(A \otimes I)
\arrow{l}{F(r_A)}
\end{tikzcd}$$

\end{itemize}
such that the following properties are verified :
\begin{itemize}
\item \todo{THE DEMONIC DIAGRAM} For all $A,B,C,D$ objects of $\mathcal{C}$, the following pasting diagram 
$$\begin{tikzcd}
\end{tikzcd}$$
must be equal to the pasting diagram
$$\begin{tikzcd}
\end{tikzcd}$$
\item  For all $A,B$ objects of $\mathcal{C}$, the following pasting diagram 
$$\begin{tikzcd}
&&
(F(A) \otimes I) \otimes F(B)
\arrow{rr}{a_{F(A),I,F(B)}}[name=F, near end]{}
\arrow{d}[swap]{(id_{F(A)} \otimes F_{I}^{\otimes}) \otimes id_{F(B)}}
&&
F(A) \otimes (I \otimes F(B))
\arrow{d}{id_{F(A)} \otimes (F_{I}^{\otimes} \otimes id_{F(B)})}
\arrow[bend left=80]{ddd}{id_{F(A)} \otimes l_{F(B)}}[name=A]{}
\\
F(A \otimes I) \otimes F(B)
\arrow{d}[swap]{F^{\otimes}_{A\otimes I, B}}
&&
(F(A) \otimes F(I)) \otimes F(B) 
\arrow{ll}[swap]{F^{\otimes}_{A,I} \otimes id_{F(B)}}
\arrow{rr}[swap]{a_{F(A),F(I),F(B)}}[name=E, near start]{}
\arrow[Rightarrow, shorten >=0.4cm, shorten <=0.6cm,from=F,to=E,]{}[]{a_{id_{F(A)},F^{\otimes}_I,id_{F(B)}}}
&&
F(A) \otimes (F(I) \otimes F(B))
\arrow{d}[swap]{id_{F(A)} \otimes F^{\otimes}_{I,B}}[name=B]{}
\arrow[Rightarrow, shorten >=0.6cm, shorten <=1.0cm,from=A,to=B]{}[swap, near end]{Id_{id_A} \otimes F^{l}_{B}}
\\
F((A \otimes I) \otimes B)
\arrow{rr}{F(a_{A,I,B})}[name=G]{}
\arrow{rrd}[swap]{F(r_A \otimes id_B)}[name=C]{}
\arrow[Rightarrow, shorten >=2.2cm, shorten <=2.1cm,from=E,to=G]{}[]{F_{A,I,B}^{a~-1}}
&&
F(A \otimes (I \otimes B))
\arrow{d}{F(id_A \otimes l_B)}[name=D]{}
\arrow[Rightarrow, shorten >=0.8cm, shorten <=1.0cm,from=D,to=C]{}[swap]{F(\mu_{A,I,B})^{-1}}
&&
F(A) \otimes F(I \otimes B)
\arrow{ll}[swap]{F^{\otimes}_{A,I\otimes B}}
\arrow{d}{id_{F(A)} \otimes F(l_B)}
\\
&&
F(A \otimes B)
&&
F(A) \otimes F(B)
\arrow{ll}{F_{A,B}^{\otimes}}
\arrow[Rightarrow, shorten >=1.2cm, shorten <=1.1cm]{llu}[swap]{F^{\otimes}_{id_{A},l_B}}
\end{tikzcd}$$
must be equal to the pasting diagram
$$\begin{tikzcd}
(F(A) \otimes F(I)) \otimes F(B)
\arrow{d}[swap]{F^{\otimes}_{A,I} \otimes id_{F(B)}}[name=A]{}
&&
F(A \otimes I) \otimes F(B)
\arrow{rr}{a_{F(A),I,F(B)}}[]{}
\arrow{ll}[swap]{(id_{F(A)} \otimes F^{\otimes}_{I}) \otimes id_{F(B)}}[]{}
\arrow{rd}[swap]{r_{F(A)} \otimes id_{F(B)}}[name=B]{}
&&
F(A) \otimes (I \otimes F(B))
\arrow{ld}{id_{F(A)} \otimes l_{F(B)}}[name=C]{}
\\
F(A \otimes I) \otimes F(B)
\arrow{d}[swap]{F^{\otimes}_{A\otimes I, B}}[]{}
\arrow{rrr}[swap]{F(r_A) \otimes id_{F(B)}}[name=D, near end]{}
&&
&
F(A) \otimes F(B)
\arrow{d}{F^{\otimes}_{A,B}}[]{}
&
\\
F((A\otimes I) \otimes B)
\arrow{rrr}[swap]{F(r_A \otimes id_B)}[name=E, near start]{}
&&
&
F(A \otimes B)
&
\arrow[Rightarrow, shorten >=3.3cm, shorten <=3.0cm,from=B,to=A]{}{F^{r}_{A}\otimes Id_{id_{F(B)}}}[]{}
\arrow[Rightarrow, shorten >=1.8cm, shorten <=1.0cm,from=C,to=B,swap]{}{\mu_{F(A),F(B)}^{-1}}[]{}
\arrow[Rightarrow, shorten >=1.0cm, shorten <=1.5cm,from=D,to=E]{}{F^{\otimes}_{r_A,id_B}}[]{}
\end{tikzcd}$$
\end{itemize}
\end{definition}

\begin{definition}
Let $\mathcal{C},\mathcal{D}$ be two monoidal bicategories and $F,G: \mathcal{C} \rightarrow \mathcal{D}$ two monoidal pseudofunctors. A monoidal pseudonatural transformation $\gamma: F \Rightarrow G$ is a pseudonatural transformation equipped with:
\begin{itemize}
\item an invertible$2-$cell $$\gamma_{I}^{\otimes}: F_{I}^{\otimes} \circ \gamma_I \Rightarrow G_{I}^{\otimes}$$
\item an invertible modification whose components are of the form : 

$$\begin{tikzcd}
F(A) \otimes F(B)
\arrow{rr}{F^{\otimes}_{A,B}}
\arrow[swap]{d}{\gamma_{A} \otimes \gamma_{B}}
&&
F(A \otimes B)
\arrow{d}{\gamma_{A\otimes B}}
\arrow[Rightarrow, shorten >=0.8cm, shorten <=0.5cm]{dll}{\gamma^{\otimes}_{A,B}}
\\
G(A) \otimes G(B) 
\arrow[swap]{rr}{G^{\otimes}_{A,B}}
&&
G(A \otimes B)
\end{tikzcd}
$$
\end{itemize}
such that the following properties are verified : 
\begin{itemize}
\item For all $A$ object of $\mathcal{C}$, the following pasting diagram $$\begin{tikzcd}
&&
F(A) \otimes I
\arrow{rr}{id_{F(A)} \otimes F_{I}^{\otimes}}
\arrow{lld}{\gamma_A \otimes id_I}
\arrow{d}{r_{F(A)}}[name=B]{}
&&
F(A) \otimes F(I)
\arrow{rrd}{\gamma_A \otimes \gamma_I}
\arrow{d}{F^{\otimes}_{A,I}}[name=A]{}
\arrow[Rightarrow, shorten >=0.8cm, shorten <=0.5cm,from=B, to=A]{F_{A}^r}
\\
G(A) \otimes I
\arrow[Rightarrow, shorten >=0.8cm, shorten <=0.5cm]{rr}{r_{\gamma_{A}}}
\arrow{rrd}{r_{G(A)}}
&&
F(A)
\arrow{d}{\gamma_A}
&&
F(A\otimes I)
\arrow{ll}{F(r_A)}
\arrow[Rightarrow, shorten >=0.8cm, shorten <=0.5cm]{rr}{\gamma^{\otimes}_{A,I}}
\arrow{d}{\gamma_{A\otimes I}}
&&
G(A) \otimes G(I)
\arrow{dll}{G^{\otimes}_{A,I}}
\\
&&
G(A)
\arrow[Rightarrow, shorten >=0.8cm, shorten <=0.5cm]{rru}{\gamma_{r_A}}
&&
G(A\otimes I)
\arrow{ll}{G(r_A)}
\end{tikzcd}
$$
is equal to the pasting diagram : $$\begin{tikzcd}
F(A) \otimes I
\arrow{rr}{id_{F(A)} \otimes F^{\otimes}_I}
\arrow{d}{\gamma_A \otimes id_I}
&&
F(A) \otimes F(I)
\arrow{rrd}{\gamma_A \otimes \gamma_I}
\arrow{d}{\gamma_A \otimes id_{F(I)}}
\\
G(A) \otimes I
\arrow{rr}{id_{G(A)} \otimes F^{\otimes}_I}
\arrow{dd}[near end,name=C]{r_{G(A)}}
\arrow[bend right=20,swap]{rrrr}[name=B]{id_{G(A)} \otimes G^{\otimes}_I}
&&
G(A) \otimes F(I)
\arrow{rr}{id_{G(A)} \otimes \gamma_I}
\arrow[Rightarrow,to=B, shorten >=0.0cm, shorten <=0.0cm]{}{Id_{id_{G(A)}} \otimes \gamma^{\otimes}_I}
&&
G(A) \otimes G(I)
\arrow{dd}[near end,name=A]{G^{\otimes}_{A,I}}
\\
\\
G(A)
&&&&
G(A \otimes I)
\arrow{llll}{G(r_A)}
\arrow[Rightarrow,from=C,to=A, shorten >=2.0cm, shorten <=4.2cm,swap, near end]{}{G^{r}_{A}}
\end{tikzcd}
$$

\item For all $A,B,C$ objects of $\mathcal{C}$, the following pasting diagram $$\begin{tikzcd}
(F(A) \otimes F(B)) \otimes F(C)
\arrow{rr}{ F^{\otimes}_{A,B} \otimes id_{F(C)}}
\arrow{d}{a_{F(A),F(B),F(C)}}
&&
F(A \otimes B) \otimes F(C)
\arrow{rr}{F^{\otimes}_{A\otimes B,C}}[name=A]{}
&&
F((A \otimes B) \otimes C)
\arrow{d}{F(a_{A,B,C})}
\\
F(A) \otimes (F(B) \otimes F(C))
\arrow{rr}{id_{F(A)} \otimes F^{\otimes}_{B,C}}[name=B]{}
\arrow{d}{\gamma_A \otimes id_{F(B) \otimes F(C)}}
\arrow[swap,bend right=80]{dd}{\gamma_A \otimes (\gamma_B \otimes \gamma_C)}
&&
F(A) \otimes F(B \otimes C)
\arrow{rr}{F^{\otimes}_{A,B\otimes C}}
\arrow{d}{\gamma_A \otimes id_{F(B \otimes C)}}
\arrow[bend left=80]{dd}{\gamma_A \otimes (\gamma_{B \otimes C})}
&&
F(A \otimes ( B \otimes C))
\arrow{dd}{\gamma_{A \otimes(B \otimes C)}}
\arrow[swap,Rightarrow, shorten >=2.8cm, shorten <=0.5cm, near start]{ddll}{\gamma^{\otimes}_{A,B \otimes C}}
\\
G(A) \otimes (F(B) \otimes F(C))
\arrow{rr}{id_{G(A)} \otimes F^{\otimes}_{B,C}}
\arrow{d}{id_{G(A)} \otimes (\gamma_B \otimes \gamma_C)}
&&
G(A) \otimes F(B \otimes C)
\arrow{d}{id_{G(A)} \otimes \gamma_{B \otimes C}}
\arrow[Rightarrow, shorten >=1.8cm, shorten <=0.5cm]{dll}{\gamma^{\otimes}_{A\otimes B} \otimes Id_{id_{G(C)}}}
\\
G(A) \otimes (G(B) \otimes G(C))
\arrow[swap]{rr}{G^{\otimes}_{A,B\otimes C}}
&&
G(A) \otimes G(B \otimes C)
\arrow[swap]{rr}{G^{\otimes}_{A\otimes B,C}}
&&
G(A \otimes (B \otimes C))
\arrow[Rightarrow, shorten >=1.8cm, shorten <=1.8cm,from=A,to=B]{}{F^{a}_{A,B,C}}
\end{tikzcd}
$$
is equal to the pasting diagram : $$\scalebox{0.8}{
\begin{tikzcd}[ampersand replacement=\&]
\&
(F(A) \otimes F(B)) \otimes F(C)
\arrow{rr}{F^{\otimes}_{A,B} \otimes id_{F(C)}}
\arrow[swap]{d}{(\gamma_A \otimes \gamma_B) \otimes id_{F(C)}}
\arrow[swap,bend right=80]{dd}{(\gamma_A \otimes \gamma_B) \otimes \gamma_C}
\arrow[swap,bend right=40]{dddl}{a_{F(A),F(B),F(C)}}
\&\&
F(A \otimes B) \otimes F(C)
\arrow{rr}{F^{\otimes}_{A\otimes B,C}}
\arrow{d}{\gamma_{A\otimes B} \otimes id_{F(C)}}
\arrow[bend left=80]{dd}{\gamma_{A \otimes B} \otimes \gamma_C}
\arrow[Rightarrow, shorten >=1.1cm, shorten <=0.8cm,swap]{dll}{Id_{id_{F(A)}} \otimes \gamma_{B,C}^{\otimes}}
\&\&
F((A \otimes B) \otimes C)
\arrow{dr}{F(a_{A,B,C})}
\arrow{dd}{\gamma_{(A\otimes B) \otimes C}}
\arrow[Rightarrow, shorten >=2.8cm, shorten <=0.5cm, swap,near start]{ddll}{\gamma_{A\otimes B, C}^{\otimes}}
\\
\&
(G(A) \otimes G(B)) \otimes F(C)
\arrow[swap]{rr}{G^{\otimes}_{A,B} \otimes id_{F(C)}}
\arrow[swap]{d}{id_{G(A) \otimes G(B)} \otimes \gamma_C}
\arrow[Rightarrow, shorten >=1.3cm, shorten <=2.2cm,swap]{ddl}{a_{\gamma_A,\gamma_B,\gamma_C}}
\&\&
G(A \otimes B) \otimes F(C)
\arrow{d}{id_{G(A \otimes B)} \otimes \gamma_C}
\&\&\&
F(A \otimes (B \otimes C))
\arrow{ddl}{\gamma_{A \otimes (B \otimes C)}}
\arrow[Rightarrow, shorten >=0.8cm, shorten <=0.5cm,swap]{dl}{\gamma_{\alpha_{A,B,C}}}
\\
\&
(G(A) \otimes G(B)) \otimes G(C) 
\arrow{rr}{G^{\otimes}_{A,B} \otimes id_{G(C)}}
\arrow{d}{a_{G(A),G(B),G(C)}}
\&\&
G(A \otimes B) \otimes G(C)
\arrow{rr}{G^{\otimes}_{A\otimes B,C}}
\&\&
G((A \otimes B) \otimes C)
\arrow{d}{G(a_{A,B,C})}
\arrow[Rightarrow, shorten >=2.5cm, shorten <=3.4cm,swap]{dllll}{G^{a}_{A,B,C}}
\\
F(A) \otimes (F(B) \otimes F(C))
\arrow{r}{\gamma_A \otimes (\gamma_B \otimes \gamma_C)}
\&
G(A) \otimes (G(B) \otimes G(C))
\arrow{rr}{id_{G(A)} \otimes G^{\otimes}_{B,C} }
\&\&
G(A) \otimes G(B \otimes C)
\arrow{rr}{G^{\otimes}_{A, B\otimes C}}
\&\&
G(A \otimes(B \otimes C))
\end{tikzcd}}$$

\end{itemize}
\end{definition}

\begin{definition}
A monoidal modification $m: \gamma \Rrightarrow \delta : F \Rightarrow G$ between two monoidal pseudonatural transformations $\gamma$ and $\delta$ is a modification verifying the following property : 
$$\begin{tikzcd}
&&
F(I)
\arrow[swap]{dd}[name=B, near start]{}[name=C]{\delta_I}
\arrow[bend left = 80]{dd}[name=D]{\gamma_I}
\\
I
\arrow{rru}{F^{\otimes}_{I}}
\arrow[swap]{rrd}[name=A]{G^{\otimes}_{I}}
&&
&&
=
&
\gamma^{\otimes}_I
\\
&&
G(I)
\arrow[Rightarrow,from = B,to=A, shorten >=0.3cm, shorten <=0.5cm,swap]{}{\delta^{\otimes}_{I}}
\arrow[Rightarrow,from = D,to=C, shorten >=0.3cm, shorten <=0.2cm,swap]{}{m_I}
\end{tikzcd}$$
and, for every object $A,B$ of $\mathcal{C}$, the following diagram
$$
\begin{tikzcd}
F(A) \otimes F(B)
\arrow{rr}{F^{\otimes}_{A,B}}
\arrow{dd}{\gamma_A \otimes \gamma_B}[name=A]{}
\arrow[bend right = 80,swap]{dd}{\delta_A \otimes \delta_B}[name=B]{}
&&
F(A \otimes B)
\arrow[Rightarrow,shorten >=1.0cm, shorten <=1.0cm]{ddll}{\gamma^{\otimes}_{A,B}}
\arrow{dd}{\gamma_{A \otimes B}}
\\
\\
G(A) \otimes G(B)
\arrow{rr}{G^{\otimes}_A,B}
&&
G(A \otimes B)
\arrow[Rightarrow, from=A,to=B,shorten >=0.3cm, shorten <=0.5cm]{}{m_A \otimes m_B}
\end{tikzcd}
$$
is equal to the diagram
$$
\begin{tikzcd}
F(A) \otimes F(B)
\arrow{rr}{F^{\otimes}_{A,B}}
\arrow{dd}{\delta_A \otimes \delta_B}
&&
F(A \otimes B)
\arrow[Rightarrow,shorten >=1.0cm, shorten <=1.0cm]{ddll}{\delta^{\otimes}_{A,B}}
\arrow[swap]{dd}{\delta_{A \otimes B}}[name=A]{}
\arrow[bend left = 80]{dd}{\gamma_{A \otimes B}}[name=B]{}
\\
\\
G(A) \otimes G(B)
\arrow{rr}{G^{\otimes}_A,B}
&&
G(A \otimes B)
\arrow[Rightarrow, from=B,to=A,shorten >=0.3cm, shorten <=0.5cm]{}{m_{A \otimes B}}
\end{tikzcd}
$$
\end{definition}


\todo{Symmetric stuff, a couple of hundred new diagrams}


\begin{definition}
A biadjunction between two pseudo-functors $F:\mathcal{C} \rightarrow \mathcal{D}$ and $G:\mathcal{D} \rightarrow \mathcal{C}$ is given by a pair of pseudo-natural transformations $\eta: Id_{\mathcal{C}} \rightarrow G F$ and $\epsilon : F G \rightarrow Id_{\mathcal{D}}$ along with two invertible modifications with components :
$$
\begin{tikzcd}
G(D)
\arrow{rr}{\eta_{G(D)}}
\arrow[swap]{rrd}{id_{G(D)}}[name=A]{}
&&
G(F(G(D)))
\arrow{d}{G(\epsilon_D)}[name=B]{}
&&&&
F(C)
\arrow{rr}{F(\eta_C)}
\arrow[swap]{rrd}{id_{F(C)}}[name=C]{}
&&
F(G(F(C)))
\arrow{d}{\epsilon_{F(C)}}[name=D]{}
\\
&&
G(D)
&&&&&&
F(C)
\arrow[Rightarrow,shorten >=0.4cm, shorten <=0.4cm,from=A,to=B]{}{s_D}
\arrow[Rightarrow,shorten >=0.4cm, shorten <=0.4cm,from=D,to=C]{}{t_C}
\end{tikzcd}
$$
such that the following diagram equalities hold for all objects $C$ of $\mathcal{C}$ and $D$ of $\mathcal{D}$ :
$$
\begin{tikzcd}
C
\arrow{rr}{\eta_C}
\arrow{d}{\eta_C}
&&
G(F(C))
\arrow[swap]{d}{G(F(\eta_C))}[name=B]{}
\arrow[bend left=50]{rddd}{id_{G(F(C))}}[name=C]{}
\\
G(F(C))
\arrow{rr}{\eta_{G(F(C))}}
\arrow[swap]{rrrdd}{id_{G(F(C))}}[name=A]{}
&&
G(F(G(F(C))))
\arrow{rdd}{G(\epsilon_{F(C)})}[name=D]{}
&&
=
&&
 Id_{id_C \circ \eta_C}
\\
\\
&&
&
G(F(C))
\arrow[Rightarrow,shorten >=0.4cm, shorten <=0.4cm,from=A,to=D]{}{s_{F(C)}}
\arrow[Rightarrow,shorten >=0.4cm, shorten <=0.8cm,from=B,to=C]{}{G(t_C)}
\end{tikzcd}
$$
$$
\begin{tikzcd}
F(G(D))
\arrow[swap,bend right=50]{dddr}{id_{F(G(D))}}[name=A]{}
\arrow[swap]{ddr}{F(\eta_{G(D)})}[name=B]{}
\arrow{ddrrr}{id_{F(G(D))}}[name=C]{}
\\
\\
&
F(G(F(G(D)))
\arrow{rr}{\epsilon_{F(G(D))}}
\arrow{d}{F(G(\epsilon_D))}[name=D]{}
&&
F(G(D))
\arrow{d}{\epsilon_D}
&&
=
&&
Id_{\epsilon_D \circ id_D}
\\
&
F(G(D))
\arrow{rr}{\epsilon_D}
&&
D
\arrow[Rightarrow,shorten >=0.4cm, shorten <=0.8cm,from=A,to=D]{}{F(s_{D})}
\arrow[swap,Rightarrow,shorten >=0.4cm, shorten <=0.8cm,from=B,to=C]{}{t_{G(D)}}
\end{tikzcd}
$$
\end{definition}

\begin{definition}
A monoidal bicategory $\mathcal{C}$ is monoidal closed if the pseudo-functor $_ \otimes B: \mathcal{C} \rightarrow \mathcal{C}$ has a right biadjoint for all objects $B$ of $\mathcal{C}$.
\end{definition}

\begin{definition}
A bicategory $\mathcal{C}$ is cartesian if the diagonal pseudofunctor $\Delta_n:\mathcal{C} \rightarrow \mathcal{C}^n$ has a right biadjoint. 
\end{definition}
\begin{definition}
A cartesian bicategory $\mathcal{C}$ is cartsian closed if the pseudo-functor $_ \times B: \mathcal{C} \rightarrow \mathcal{C}$ has a right biadjoint for all objects $B$ of $\mathcal{C}$.
\end{definition}
\begin{definition}
A pseudo-comonoid $A$ in a monoidal bicategory $\mathcal{C}$ is given by an object $A$ of the bicategory, equipped with :
\begin{itemize}
\item a $1-$cell $J: A \rightarrow I$
\item a $1-$cell $P:C \rightarrow C \otimes C$
\item three invertible $2-$ cells 
$$\begin{tikzcd}
A \otimes A
\arrow{d}{ P \otimes id_A}
&&
A
\arrow{ll}{P}
\arrow{rr}{P}
&&
A \otimes A
\arrow{d}{id_A \otimes P}
\\
(A \otimes A) \otimes A
\arrow{rrrr}{a_{A,A,A}}
\arrow[Rightarrow, shorten >=1.8cm, shorten <=2.0cm]{rrrru}{\alpha}
&&&&
A \otimes (A \otimes A)
\end{tikzcd}
$$
$$
\begin{tikzcd}
A
\arrow{rr}{P}
\arrow[swap]{rrd}{l_{A}^{-1}}[name=A]{}
&&
A \otimes A
\arrow{d}{J \otimes id_A}[name=B]{}
\\
&&
I \otimes A
\arrow[Rightarrow, from=A,to=B, shorten >=0.3cm, shorten <=0.3cm]{}{\lambda}
\end{tikzcd}
$$
$$
\begin{tikzcd}
A
\arrow{rr}{P}
\arrow[swap]{rrd}{r_{A}^{-1}}[name=A]{}
&&
A \otimes A
\arrow{d}{  id_A\otimes J}[name=B]{}
\\
&&
A \otimes I
\arrow[Rightarrow, from=A,to=B, shorten >=0.3cm, shorten <=0.3cm]{}{\rho}
\end{tikzcd}
$$
\end{itemize}
such that the following properties are verified : 
The diagram $$
\begin{tikzcd}
A \otimes I \otimes A
&&
A \otimes A \otimes A
\arrow[swap]{ll}{id_A \otimes J \otimes id_A}
\\
A \otimes A
\arrow{u}{id_A \otimes l_{A}^{-1}}[name=A]{}
\arrow[Rightarrow, swap,shorten >=0.5cm, shorten <=0.5cm]{rr}{\alpha}
\arrow[swap]{urr}{id_A \otimes P}[name=B]{}
&&
A \otimes A
\arrow[swap]{u}{P \otimes id_A}
\\
A
\arrow{u}{P}
\arrow[swap]{urr}{P}
\arrow[Rightarrow, from=A,to=B, shorten >=0.5cm, shorten <=0.5cm]{}{id_A \otimes \lambda}
\end{tikzcd}
$$
is equal to the diagram
$$
\begin{tikzcd}
A \otimes I \otimes A
&&
A \otimes A \otimes A
\arrow[swap]{ll}{id_A \otimes J \otimes id_A}
\\
A \otimes A
\arrow{u}{r_{A}^{-1}}[name=A]{}
\arrow[swap]{urr}{P \otimes id_A}[name=B]{}
\\
A
\arrow{u}{P}
\arrow[Rightarrow, from=A,to=B, shorten >=0.5cm, shorten <=0.5cm]{}{\rho \otimes id_A }
\end{tikzcd}
$$
and the following diagram :
$$
\begin{tikzcd}
A \otimes A \otimes A \otimes A 
&&
A \otimes A \otimes A
\arrow[swap]{ll}{id_A \otimes id_A \otimes P}
&&
A \otimes A
\arrow[swap]{ll}{id_A \otimes P}
\arrow[Rightarrow, swap,shorten >=0.5cm, shorten <=0.5cm]{lld}{\alpha}
\\
A \otimes A \otimes A
\arrow{u}{P \otimes id_A \otimes id_A}
&&
A \otimes A
\arrow{u}{P \otimes id_A}
\arrow{ll}{id_A \otimes P}
\arrow[Rightarrow, swap,shorten >=0.0cm, shorten <=0.0cm]{d}{\alpha}
&&
A
\arrow{u}{P}
\arrow{ll}{P}
\arrow{lld}{P}
\\
&&
A \otimes A
\arrow{ull}{P \otimes id_A}
\end{tikzcd}
$$

must be equal to the diagram : 
$$
\begin{tikzcd}
A \otimes A \otimes A \otimes A 
&&
A \otimes A \otimes A
\arrow[swap]{ll}{id_A \otimes id_A \otimes P}
\arrow[swap,Rightarrow, swap,shorten >=0.0cm, shorten <=0.0cm]{d}{Id_{id_A} \otimes \alpha }
&&
A \otimes A
\arrow[swap]{ll}{id_A \otimes P}
\arrow{lld}{id_A \otimes P}
\arrow[Rightarrow, shorten >=1.2cm, shorten <=1.2cm]{lddl}{\alpha}
\\
A \otimes A \otimes A
\arrow{u}{P \otimes id_A \otimes id_A}
&&
A \otimes A \otimes A
\arrow[swap]{ull}{id_A \otimes P \otimes id_A}
\arrow[Rightarrow, swap,shorten >=0.5cm, shorten <=0.5cm]{ll}{\alpha \otimes Id_{id_A}  }
&&
A
\arrow{u}{P}
\arrow{lld}{P}
\\
&&
A \otimes A
\arrow{ull}{P \otimes id_A}
\arrow{u}{P \otimes id_A}
\end{tikzcd}
$$
\end{definition}
\begin{definition}
A pseudo-comonad on a bicategory $\mathcal{C}$ is given by a pseudo-functor $F: \mathcal{C} \rightarrow \mathcal{C}$, two pseudo-natural transformations $v:F \Rightarrow Id_{\mathcal{C}}$ and $n: F \Rightarrow F \circ F$ called the counit and comultiplications, and three invertible modifications $ \alpha,\lambda,\rho$ whose components are given by the following diagrams : \\
\begin{tikzcd}
F(A)
\arrow{rr}{n_A}
\arrow{d}{n_A}
&&
F(F(A))
\arrow{d}{n_{F(A)}}
\arrow[Rightarrow, shorten >=0.2cm, shorten <=0.2cm]{dll}{\alpha_A}
\\
F(F(A))
\arrow{rr}{F(n_A)}
&&
F(F(F(A)))
\end{tikzcd}
\begin{tikzcd}
&&
F(A)
\arrow[swap]{lld}{id_{F(A)}}[name=A]{}
\arrow{rrd}{id_{F(A)}}[name=C]{}
\arrow{d}{n}[name=B]{}
\arrow[Rightarrow, from=C,to=B, shorten >=0.5cm, shorten <=0.5cm]{}{\rho_A }
\arrow[Rightarrow, from=B,to=A, shorten >=0.5cm, shorten <=0.5cm]{}{\lambda_A }
\\
F(A)
&&
F(F(A))
\arrow{ll}{F(v_A)}
\arrow{rr}{v_{F(A)}}
&&
F(A)
\end{tikzcd}
such that the following properties are verified : 
\todo{}
\end{definition}
\begin{definition}
The Kleisli bicategory $\mathcal{C}_F$ associated to a pseudo-comonad F on a bicategory $\mathcal{C}$ is defined as having the same $0$-cells as $\mathcal{C}$, and whose hom-category $\mathcal{C}_F (A,B)$ is given by $\mathcal{C}(F(A),B)$. \\
The composition in $\mathcal{C}_F$ of $f:F(A)\rightarrow B$ and $g:F(B) \rightarrow C$ is defined as $$g \circ_F f := g \circ f(F) \circ n_A$$.
This definition can easily be extended to provide the required composition functors. The identities in $\mathcal{C}_F$ are given by the components of the counit of the comonad. The $2$-isomorphisms and additional properties of the bicategory come directly from the pseudo-comonad structure.
\end{definition}
\begin{definition}
A linear exponential pseudo-comonad 
\end{definition}
%\begin{definition}\label{definition/pseudofunctor}
%A pseudofunctor is a mapping between bicategories $\mathcal{C}$ and $\mathcal{D}$ where the usual functorial equations $F(f \circ g) = F(f) \circ F(g) $ and $F(Id_A) = Id_{F(A)}$ are only valid up to natural bijectve 2-morphisms in $\mathcal{D}$. 
%\end{definition}
%
%\begin{definition}\label{definition/laxmonoidal}
%Let $(\mathcal{C},\otimes_{\mathcal{C}},1_{\mathcal{C}})$ and $(\mathcal{D},\otimes_{\mathcal{D}},1_{\mathcal{D}})$ be two monoidal bicategories. A lax monoidal pseudofunctor between them is given by : 
%\begin {itemize}
%\item a pseudofunctor $F:\mathcal{C} \rightarrow \mathcal{D}$
%\item a morphism $\epsilon : 1_{\mathcal{D}} \rightarrow F(1_{\mathcal{C})}$
%\item for every pair of objects $A,B \in \mathcal{C}$, a natural transformation $\mu_{A,B}: F(A) \otimes_{\mathcal{D}} F(B) \rightarrow F(A \otimes_{\mathcal{C}} B)$%pseudonatural ???
%\end{itemize}
%satisfying the following conditions :
%\begin{itemize}
%\item associativity : For every triple of objects $A,B,C \in \mathcal{C}$, the following diagram commutes : 
%$$
%\xymatrix @-1.2pc {
%(F(A) \otimes_{\mathcal{D}} F(B)) \otimes_{\mathcal{D}} F(C)  
%\ar[dd]_-{\mu_{A,B} \otimes id}
%\ar[rrrr]_-{a^{\mathcal{D}}_{F(A),F(B),F(C)}}
%&&&& 
%F(A) \otimes_{\mathcal{D}} (F(B) \otimes_{\mathcal{D}} F(C))  
%\ar[dd]_-{id \otimes \mu_{B,C}}
%\\
%\\
%F(A \otimes_{\mathcal{C}} B) \otimes_{\mathcal{D}} F(C)  
%\ar[dd]_-{\mu_{A\otimes B, C}}
%&&&& 
%F(A) \otimes_{\mathcal{D}} F(B \otimes_{\mathcal{C}} C)  
%\ar[dd]_-{\mu_{A,B\otimes C}}
%\\
%\\
%F((A \otimes_{\mathcal{C}} B) \otimes_{\mathcal{C}} C)  
%\ar[rrrr]_-{F(a^{\mathcal{C}}_{A,B,C})}
%&&&& 
%F(A \otimes_{\mathcal{C}} (B \otimes_{\mathcal{C}} C))  
%}
%$$
%where the two morphisms $a^\mathcal{C}, a^\mathcal{D}$ denote the associators of the two tensor products.
%
%\item unality : For every object $A \in \mathcal{C}$, the following diagram and its right symmetry both commute : 
%$$
%\xymatrix @-1.2pc {
%1_\mathcal{D} \otimes_\mathcal{D} F(A) 
%\ar[dd]_-{l^{\mathcal{D}}_{F(A)}}
%\ar[rrrr]_-{\epsilon \otimes id}
%&&&& 
%F(1_\mathcal{C}) \otimes_\mathcal{D} F(A) 
%\ar[dd]_-{\mu_{1_\mathcal{C}, A}}
%\\
%\\
%F(A)  
%&&&& 
%F(1_\mathcal{C} \otimes_{\mathcal{C}} A)  
%\ar[llll]_-{F(l^{\mathcal{C}_{A}})}
%}
%$$
%where $l^\mathcal{C}, l^{\mathcal{D}}$ denote the left unitors of the two tensor products.
%
%\end{itemize}
%\end{definition}
%
%\begin{definition}\label{definition/pseudonatural}
%Let $F,G$ be two pseudofunctors between two bicategories $\mathcal{C}$ and $\mathcal{D}$. A pseudonatural transformation $\phi : F \rightarrow G$ is given by : 
%\begin{itemize}
%\item for every object $A$ of $\mathcal{C}$, a morphism $\phi(A): F(A) \rightarrow G(A)$ of $\mathcal{D}$.
%\item for every morphism $f:A \rightarrow B$ of $\mathcal{C}$, a bijective $2-$morphism $\phi(f): \phi(B) \circ F(f) \Rightarrow G(f) \circ \phi(A)$
%\end{itemize}
%such that
%\begin{itemize}
%\item $\phi$ respects composition of morphisms, meaning that we have an equivalence between 
%$$(\phi(A) \triangleleft G(f,g) )\cdot(\phi(f) \triangleright G(g)) \cdot(F(f)\triangleleft \phi(g))$$ and 
%$$\phi(g \circ f) \cdot (F(f,g) \triangleright  \phi(C) ),$$ 
%both being $2$-morphisms from 
%$$\phi(C) \circ F(g) \circ F(f) \Rightarrow G(g \circ f) \circ \phi(A),$$ 
%where $\cdot$ is the vertical composition between $2$-morphisms, $\triangleleft, \triangleright$ the two versions of the horizontal composition between a morphism and a $2$-morphism, (also called whiskering), anf $F(f,g):F(g) \circ F(f) \Rightarrow F(g \circ f)$ is the bijective $2$-morphism coming from the pseudofunctor $F$.
%\item $\phi$ respects the identity morphisms, meaning we have an equivalence between
%$$L^{\mathcal{D}}_{\phi(A)} \cdot \epsilon^{F}_{id_A} \triangleright \phi(A) $$ and
%$$R^{\mathcal{D}}_{\phi(A)} \cdot \phi(A) \triangleleft \epsilon^{G}_{id_A} \cdot \phi(id_A) $$
%both being $2$-morphisms from
%$$\phi(A) \circ F(id_A)  \Rightarrow \phi(A) $$
%where $L^{\mathcal{D}}_{\phi(A)}: \phi(A) \circ id_{F(A)} \Rightarrow  \phi(A)$ is the left unitor coming from the bicategory $\mathcal{D}$  and $\epsilon^{F}_{id_A}: F(id_A) \Rightarrow id_{F(A)}$ is the bijective $2$-morphism coming from the pseudofunctor $F$.
%\item $\phi$ is natural in the following sense : for every $2$-morphism $\psi: f \Rightarrow g$ with $f,g:A\rightarrow B$, we have an equivalence between $$\phi(g) \cdot F(\psi) \triangleright \phi(B)$$ and $$\phi(A)\triangleleft G(\psi)\cdot \phi(f).$$
%\end{itemize}
%\end{definition}
%
%\begin{definition}\label{definition/pseudocomonad}
%A fully weak comonad $G$ on a bicategory $\mathcal{C}$ is a pseudofunctor, along with pseudonatural transformations $\delta$ and $\epsilon$ that satisfy the usual laws of a comonad up to natural bijectiive 2-morphisms in $\mathcal{C}$.
%\end{definition}

\end{document}














