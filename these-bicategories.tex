%
% lire mccusker-harmer
%

%\documentclass[10pt]{llncs}
%\documentclass[english]{lipics}
\documentclass[a4paper, 12pt, twoside,openright]{report}
\usepackage{amsthm}
\newtheorem{definition}{Definition}
\newtheorem{proposition}{Proposition}
\newtheorem{theorem}{Theorem}
\usepackage[margin=1in]{geometry}
%\usepackage{prentcsmacro}
\usepackage{lscape}
\usepackage[utf8]{inputenc}
\usepackage[T1]{fontenc}
\usepackage{aeguill}
\usepackage{stmaryrd}
\usepackage{amssymb}
\usepackage{shortcuts}
\usepackage{prftree}
%\usepackage{proof}
%\usepackage{xypic}
\usepackage{graphicx}
\usepackage{graphics}
\usepackage[mathscr]{euscript}
%\usepackage{a4wide}
%\usepackage{times}

\usepackage{tikz-cd}
\usetikzlibrary{arrows}
\usetikzlibrary{matrix}
\newcounter{nodemaker}
\setcounter{nodemaker}{0}
\def\twocell#1#2{%
  \global\edef\mynodeone{twocell\arabic{nodemaker}}%
  \stepcounter{nodemaker}%
  \global\edef\mynodetwo{twocell\arabic{nodemaker}}%
  \stepcounter{nodemaker}%
  \ar[#1,phantom,shift left=3,""{name=\mynodeone}]%
  \ar[#1,phantom,shift right=3,""'{name=\mynodetwo}]%
  \ar[Rightarrow,from=\mynodeone,to=\mynodetwo,swap, "#2"]%
}
\usepackage[all]{xy}
%\CompileMatrices



%\addtolength{\textheight}{7.6em}
%\addtolength{\voffset}{-3.5em}

%\addtolength{\textwidth}{3.19em}
%\addtolength{\hoffset}{-1.6em}

%\addtolength{\textwidth}{4em}
%\addtolength{\hoffset}{-2em}

%\newtheorem{thm}{Theorem}
%\newtheorem{proposition}[theorem]{\textbf{Proposition}}
%\newtheorem{defn}{Definition}

%\renewcommand{\vxym}[1]{\vcenter{\xymatrix@C=3ex@R=3ex{#1}}}

\newcommand{\myparagraph}[1]{\paragraph{\textbf{#1}}}

% Asynchronous graphs
\newcommand{\AGrph}{\mathbf{AGrph}}
%\newcommand{\G}{\mathcal{G}}
\newcommand{\G}{G}
\newcommand{\moves}[1]{M_{#1}} % moves in a path
\newcommand{\setofmoves}[1]{M_{#1}} % moves of a ATS
\newcommand{\movesorder}[1]{\leq_{#1}} % order on moves
\newcommand{\lab}[1]{\ell({#1})}
\newcommand{\incompat}{\#}
\newcommand{\lattice}[1]{\mathcal{L}_{#1}}
%\newcommand{\tile}[1]{\diamond_{#1}}
\newcommand{\initpos}[1]{{*}_{#1}}
\newcommand{\hclass}[1]{[#1]} % class
\newcommand{\hcat}[1]{[#1]} % category of paths modulo homotopy
\newcommand{\htcat}[1]{\overline{#1}} % two category generated by an asynchronous graph
\newcommand{\paths}[1]{\mathrm{paths}(#1)} % paths starting from *
\newcommand{\pcompl}[1]{{#1}^\lightning} % path completion from complete positions
\newcommand{\subgraph}[1]{G_{#1}} % subgraph generated by a strategy
\newcommand{\concat}{\cdot} % concatenation of two paths
\newcommand{\hprefix}{\lesssim} % prefix modulo homotopy
\newcommand{\unfolding}[1]{T{#1}}

% Shortcuts
\newcommand{\homotopic}[1]{\sim_{#1}}
\newcommand{\qsim}{\quad\sim\quad}
\newcommand{\lbl}[2]{\ar@{}[#1]|-{#2}}
\newcommand{\hlbl}[1]{\lbl{#1}{\sim}}


% Strategies
%\newcommand{\interact}{\div}
\newcommand{\buffer}[1]{\mathrm{buf}_{#1}}
\newcommand{\Inno}{\mathbf{Inno}}
\newcommand{\alt}[1]{\mathrm{alt(#1)}}
\newcommand{\halting}[1]{\mathrm{halting}(#1)} % fixpoints

\newcommand{\todo}[1]{\textcolor{red}{TODO}: \underline{#1}}
% Closure operators
\newcommand{\clop}[1]{\mathrm{Cl}(#1)} % closure operator
\newcommand{\fixpoints}[1]{\mathrm{fix}(#1)} % fixpoints
\newcommand{\domain}[1]{\mathrm{dom}(#1)} % domain
\newcommand{\dynadom}[1]{\mathrm{dynadom}(#1)} % dynamic domain

% Positions
\newcommand{\pos}[1]{{#1}^\circ} % positions
\newcommand{\cpos}[1]{{#1}^\circ} % complete positions
\newcommand{\tcompl}[1]{{#1}^\top} % top-completion

% Boolean game
\newcommand{\booleangame}{\mathbb{B}} % notation for the boolean game
\newcommand{\questionmove}{\mathtt{q}} %question move
\newcommand{\answermove}{\mathtt{a}} %question move
\newcommand{\falsemove}{\mathtt{false}} %answ\er move --- false
\newcommand{\truemove}{\mathtt{true}} % answer move --- true
\newcommand{\questionposition}{q} % intermediate position q in the boolean game
\renewcommand{\root}{\ast} % root of the game
\newcommand{\tensorpair}[2]{#1\otimes #2} % pair of positions related by a tensor
\newcommand{\opposant}[1]{#1} % opponent colour of moves
\newcommand{\joueur}[1]{#1} % player colour of moves

\newcommand{\presheaf}[1]{\mathscr{P}#1}
\newcommand{\distributor}[1]{[#1]}

% Connectives
\newcommand{\before}{\varolessthan}
\newcommand{\after}{\varogreaterthan}

\newcommand{\Opt}{{\rm Opt}}
\newcommand{\Ppt}{{\rm Ppt}}
\newcommand{\Sw}{{\rm Switch}}
\newcommand{\Kl}{{\rm Kl}}
\newcommand{\restrict}{\hspace{-.05in}\upharpoonright \hspace{-.05in}}
\newcommand{\views}{\succeq}
\newcommand{\hp}{{\rm hp}}
\newcommand{\tensor}{\otimes}
\newcommand{\limp}{\multimap}
\newcommand{\aimp}{\rule[.042in]{.2in}{.003in}\hspace{-.12in}\boxplus} 


\newcommand{\questiontt}{\mathtt{q}}
\newcommand{\answertt}{\mathtt{a}}
\newcommand{\Unit}{\mathbf{C}}
\newcommand{\choice}{\mathtt{choice}}
\newcommand{\Pview}[1]{\ulcorner#1\urcorner}
\newcommand{\plays}[1]{P_{#1}}
\newcommand{\oracle}{\mathbf{oracle}}

\newcommand{\refcat}[1]{\PStrategy{#1}}

\newcommand{\refcatfun}{\PStrategyalone}
\newcommand{\oddcart}{\OTransduction}
\newcommand{\slender}[1]{#1^p}
\newcommand{\refin}[1]{\mathcal{T}_{#1}}
\newcommand{\image}[1]{\mathsf{image}(#1)}
%\newcommand{\comprehensionalone}{Comprehend}
\newcommand{\imagealone}{\mathsf{image}}
\newcommand{\imagestrat}[1]{\image{#1}}
\newcommand{\runtree}[1]{\mathcal{R}_{#1}}

%\newcommand{\Games}{\mathscr{G}}
\newcommand{\Games}{\mathscr{G}}
\newcommand{\BackForth}{\mathscr{BF}}
\newcommand{\BackForthplus}{\mathscr{BF}^{+}}
\newcommand{\BackForthminus}{\mathscr{BF}^{-}}
\newcommand{\SheafGames}{\mathscr{S}}
\newcommand{\Transduction}{\mathscr{T}}
\newcommand{\OTransduction}{\mathscr{T}_{O}}
\newcommand{\PTransduction}{\mathscr{T}_{P}}
%\newcommand{\PStrategy}[1]{\mathscr{S}_{#1}}
\newcommand{\PStrategy}[1]{\mathscr{P}({#1})}
\newcommand{\PStrategyalone}{\mathscr{P}}
\newcommand{\comp}[1]{\mathbf{comp}_{#1}}
%\newcommand{\projectionpaire}[1]{{\mathbf{\pi}_{#1}}}
%\newcommand{\projectionpaire}[1]{\mathcal{\pi}_{#1}}
\newcommand{\projectionpaire}[1]{\pi^2_{#1}}
\newcommand{\projectionpairealone}{\pi^2}
\newcommand{\projectionP}[1]{(\pi_P)_{#1}}
\newcommand{\projectionPalone}{\pi_P}

%\newcommand{\gameA}[1]{A_{[#1]}}
%\newcommand{\projectionpi}[1]{\pi_{[#1]}}
%\newcommand{\projectionpialone}{\pi}
\newcommand{\gameA}[1]{A_{#1}}
\newcommand{\gameAalone}{A}
\newcommand{\gameB}[1]{B_{#1}}
\newcommand{\gameBalone}{B}
\newcommand{\supportS}[1]{S_{#1}}
\newcommand{\supportSalone}{S}
%\newcommand{\support}[1]{\mathsf{support}(#1)}
\newcommand{\support}[2]{\{#1\,|\,#2\}}
\newcommand{\inclusion}[2]{\mathsf{supp}_{\,#2}}
%\newcommand{\inclusion}[2]{\mathsf{in}_{\support{#1}{#2}}}
%\newcommand{\supportalone}{\mathsf{support}}
\newcommand{\supportalone}[1]{\support{#1}{-}}
\newcommand{\projectionpi}[1]{\pi_{#1}}
\newcommand{\projectionpialone}{\pi}
\newcommand{\strategysigma}[1]{\sigma_{#1}}
\newcommand{\strategysigmaalone}{\sigma}
\newcommand{\strategytau}[1]{\tau_{#1}}
\newcommand{\strategytaualone}{\tau}

%\newcommand{\Grothendieck}[1]{\mathbf{Groth}({#1})}
%\newcommand{\Grothendieck}[1]{\int {#1}}
\newcommand{\Grothendieck}[1]{\textbf{tree}({#1})}
%\newcommand{\restricted}[1]{\widetilde{#1}}
\newcommand{\restricted}[1]{#1_P}
\newcommand{\exponential}[1]{{!{#1}}}
%
%\newcommand{\Rel}{\mathbf{Rel}}
\newcommand{\Dist}{\mathbf{Dist}}
\newcommand{\Schedule}{\Upsilon}

\newcommand{\seqcomp}{\centerdot}
\newcommand{\push}[2]{\mathbf{push}_{#1}#2}

\newenvironment{forcedcentertikzcd}
 {\begin{lrbox}{\forcedcentertikzcdbox}\begin{tikzcd}}
 {\end{tikzcd}\end{lrbox}\makebox[0pt]{\usebox{\forcedcentertikzcdbox}}}
\newsavebox{\forcedcentertikzcdbox}

%\title{Categorical combinatorics\\ for non deterministic strategies on simple games}
 % \title{Categorical combinatorics\\ for non deterministic strategies}
%\author{Cl\'ement Jacq and Paul-Andr\'e Melli\`es}
%\institute{Institut de Recherche en Informatique Fondamentale, Universit\'e Paris Diderot}
 % \author{Cl\'ement Jacq\hspace{.2em}}
%  \author{\hspace{.2em} Paul-Andr\'e Melli\`es}
 % \affil{Institut de Recherche en Informatique Fondamentale (IRIF)\\
 %Universit\'e Paris Diderot}
%\title{Categorical combinatorics
%for non deterministic strategies
%on simple games}
%\author{Cl\'ement Jacq and Paul-Andr\'e Melli\`es}
%\affil{Institut de Recherche en Informatique Fondamentale, Universit\'e Paris Diderot}

\begin{document}
\chapter{Bicategorical Models of Linear Logic}
In this chapter, we will refine our definition of a model of linear logic by making explicit some information of the logic that is hidden in the usual model. For this, we will need to shift settings going from the simple categorical setting to the more complex bicategorical one. For this, we start be going back to the formal definition of a categorical model of a logic : 


\begin{definition}
Let $\mathcal{L}$ be a logic. A categorical model of a logic is a category $\mathcal{C}$, along with a mapping $[ - ]$ such that : 
\begin{itemize}
\item for every formula $A$ of $\mathcal{L}$, there is an object $[A]$ in $\mathcal{C}$
\item for every proof $\pi$ of $A\rightarrow B$ of $\mathcal{L}$, there is a morphism $[\pi]: [A] \rightarrow [B]$ in $\mathcal{C}$
\item for every pair of proofs $\pi_1, \pi_2$ such that $\pi_1 =_{\mathcal{L}} \pi_2$, we have$[\pi_1]=[\pi_2]$ in  $\mathcal{C}$
\end{itemize}

\end{definition}


This last line is where our work will focus. In Linear Logic, the equality relation $=_{\mathcal{L}}$ is given the relation raising from the cut-elimination process, or more exactly, the transitive reflexive closure of a weaker version introduced by Melliès in\todo{cite}. The cut-elimination process introduced by Girard is a formal process taking any proof of linear logic and transforming it into a proof using no cut rules, giving us a normalization of proofs. We will recall the rules of cut-elimination as we encounter them in our construction. What the line then means is simply : whenever two proofs are related by cut elimination in any way, they're modelized by the same morphism. Note that this is slightly stronger than just saying ''every proof is modelized by the interpretation of its cutless proof'' as the weaker version of the cut elimination process is not confluent, allowing for a proof to produce two different cutless proofs, which must then have the same model. We will see examples of this in what follows. \\

What we wish to achieve with our model is to refine the existing one by allowing different proofs related by cut-elimination to be interpreted in different but related ways. This means that the proof transformations themselves must be interpreted in the model explicitely. This matches well with the notion of bicategory. Indeed, as proofs are related by proof transformations in our logic, morphisms are related by 2-morphisms in a bicategory. This pushes us towards the following description of a bicategorical model of logic : \\

Let $\mathcal{L}$ be a logic with proof transformations. A bicategorical model of a logic is a bicategory $\mathcal{C}$, along with a mapping $[ - ]$ such that : 
\begin{itemize}
\item for every formula $A$ of $\mathcal{L}$, there is an object $[A]$ in $\mathcal{C}$
\item for every proof $\pi$ of $A\rightarrow B$ of $\mathcal{L}$, there is a morphism $[\pi]: [A] \rightarrow [B]$ in $\mathcal{C}$
\item for every transformation of proofs $t:\pi_1 \Rightarrow \pi_2$ in the logic , there is a 2-morphism $[t]: [\pi_1]\Rightarrow [\pi_2]$ in  $\mathcal{C}$
\end{itemize}

This is not enough though as this definition is missing the equivalent of the last line of the categorical model, which introduced the equalities between proofs required to obtain the coherence structures of categories. We will need something similar to obtain the coherence structures of bicategories. What we will see very soon  is that every basic cut-elimination rule can be matched to one of the usual structural 2-morphisms of certain bicategorical setting. As the basic definition of coherence in bicategories can be summed up to :\\
$$\text{Every formal $2-$diagram consisting of structural $2-$morphisms commutes,}$$
one requirement on proof transformations that will allow for this property to hold for their interpretation is the requirement that any two proof transformation between the same proofs are interpreted in the same way. \todo{this is overkill, but the alternative is offering a full standardization of cut-elimination process} giving us the following definition : 
 
\begin{definition}
Let $\mathcal{L}$ be a logic with proof transformations. A bicategorical model of a logic is a bicategory $\mathcal{C}$, along with a mapping $[ - ]$ such that : 
\begin{itemize}
\item for every formula $A$ of $\mathcal{L}$, there is an object $[A]$ in $\mathcal{C}$
\item for every proof $\pi$ of $A\rightarrow B$ of $\mathcal{L}$, there is a morphism $[\pi]: [A] \rightarrow [B]$ in $\mathcal{C}$
\item for every transformation of proofs $t:\pi_1 \Rightarrow \pi_2$ in the logic , there is a 2-morphism $[t]: [\pi_1]\Rightarrow [\pi_2]$ in  $\mathcal{C}$
\item for every pair of proof transformations $t_1, t_2: \pi_1 \Rightarrow \pi_2$, we have$[t_1]=[t_2]$ in  $\mathcal{C}$
\end{itemize}

\end{definition}  

In this chapter, we will go over the case of Intuitionistic linear logic in full, starting by the core logic , which will include a proper definition of bicategories. 

\section{Core Logic and Bicategories} 
In this section, we start with the core rules of Linear Logic, with no connector, and take the opportunity to recall some basic bicategorical definitions.\\

The very core rules of Linear Logic, and indeed of most logics can be reduced to two simple rules :  
$$ \begin{array}{ccc}
 \prftree[r]{ax}{A\vdash A}&& 
 \prftree[r]{cut}
	{\prftree{\pi_1}{\Gamma \vdash A}
	}
	{\prftree{\pi_2}{A,\Delta \vdash B}
	}
	{\Gamma,\Delta \vdash B}\\
\end{array}
$$
Note that this doesn't give rise to a very interesting logical system as the only proofs that can be built only with these are proofs of $A \vdash A$. They still offer some good insights into the categorical ( or rather bicategorical) structure of a model of logic. \\
Indeed, the axiom rule is modelized by identities and the cut rule by composition of morphisms as we have seen in Chapter 2. The next part is more interesting. We have mentioned earlier that the rules of the cut elimination process will be modelized by $2-$morphisms in a bicategorical models where they were used to produce the constraints in a categorial model. This lines up with the general presentation of higher-order categories where the constraints of the categorical definition are turned into new data in the higher order definition. The new constraints of the definition are then obtained from a coherence theorem. Let us take a look at this basic case with the rules of cut elimination process involving the axiom and cut rules : 
$$
%\resizebox{5cm}{
\begin{array}{cccc}
 \text{l-ax } cut
&
 \prftree[r]{cut}
	{\prftree[r]{ax}
		{A\vdash A}
	}
	{\prftree{\pi_1}{A,\Delta \vdash B}
	}
	{A,\Delta \vdash B}
&
\Rightarrow
&
\prftree{\pi_1}{A,\Delta \vdash B}
\\
\\
\text{r-ax } cut
&
\prftree[r]{cut}
	{\prftree{\pi_1}{\Gamma \vdash B}
	}
	{\prftree[r]{ax}
		{B\vdash B}
	}
	{\Gamma \vdash B}
&
\Rightarrow
&
\prftree{\pi_1}{\Gamma \vdash B}
\\
\\
\text{commut } cut
&
 \prftree[r]{cut}
	{ \prftree[r]{cut}
		{\prftree{\pi_1}{\Gamma \vdash A}
		}
		{\prftree{\pi_2}{A,\Delta \vdash B}
		}
		{\Gamma,\Delta \vdash B}
	}
	{\prftree{\pi_3}{B,\Theta \vdash C}
	}
	{\Gamma,\Delta, \Theta \vdash C}
&
\Rightarrow
&
 \prftree[r]{cut}
	{\prftree{\pi_1}{\Gamma \vdash A}
	}
	{ \prftree[r]{cut}
		{\prftree{\pi_2}{A,\Delta \vdash B}
		}
		{\prftree{\pi_3}{B,\Theta \vdash C}
		}
		{A,\Delta,\Theta \vdash C}
	}
	{\Gamma,\Delta, \Theta \vdash C}
\end{array}
%}
$$
The first two transformations are called the left and right axiom cut transformations, while the last one is called the cut commutation. \\ Note that in those three cases, the detail of the branches $\pi_1,\pi_2,\pi_3$ doesn't matter to the transformation, only the form of their conclusion. We will call those branches the argument branches of the transformations.\\ 
Those rules then translate into the $2-$ morphisms : 
$$id_A \circ [\pi_1] \Rightarrow [\pi_1]$$
$$[\pi_1] \circ id_B \Rightarrow  [\pi_1]$$
$$([\pi_1] \circ  [\pi_2]) \circ  [\pi_3]\Rightarrow  [\pi_1] \circ  ([\pi_2] \circ  [\pi_3])$$

The coherence conditions of the definition of bicategory then match the requirements of our definition of bicategorical model by definition, but we can still look at it a bit more in depth. We can categorize the required equalities of proof transformations in three classes : 
\begin{itemize}
\item The natural equalities. Those are of the following form  :\\

let $T$ be a proof tree with a branch $\pi_1$, $t_1$ a transformation turning $T$ into $T'$ with $\pi_1$ as an argument branch and $t_2$ a transformation turning $\pi_1$ into $\pi'_1$. We want to be able to say that the two following sequences are equalized in the model : 
$$
\begin{array}{ccccc}
\prftree{
	\prftree{\pi_1}{T}}
{\Gamma \vdash A}
&
\Rightarrow^{t_1}
&
\prftree{
	\prftree{\pi_1}{T'}}
{\Gamma \vdash A}
&
\Rightarrow^{t_2}
&
\prftree{
	\prftree{\pi'_1}{T'}}
{\Gamma \vdash A}
\\
\\
\prftree{
	\prftree{\pi_1}{T}}
{\Gamma \vdash A}
&
\Rightarrow^{t_2}
&
\prftree{
	\prftree{\pi'_1}{T}}
{\Gamma \vdash A}
&
\Rightarrow^{t_1}
&
\prftree{
	\prftree{\pi'_1}{T'}}
{\Gamma \vdash A}
\end{array}
$$
Those particular equalities turn into naturality conditions when translated into the bicategorical model.
\item The parallel equalities. Those are of the following form :\\

let $T$ be a proof tree with two branches $\pi_1$ and $\pi_2$, $t_1$ a transformation turning $\pi_1$ into $\pi'_1$ and $t_2$ a transformation turning $\pi_2$ into $\pi'_2$. We want to be able to say that the two sequences $t_1 \cdot t_2$ and $t_2 \cdot t_1$ are equalized in the model. These equations amount to bifunctoriality conditions. For example, with $T$ being the cut rule, the equations become $([t_1] \circ id_{[\pi_2]}) \cdot (id_{[\pi_1]} \circ [t_2] ) = (id_{[\pi_1]} \circ [t_2] ) \cdot ([t_1] \circ id_{[\pi_2]})$ where $\cdot$ is the vertical composition of $2-$morphisms in the hom-categories.

\item The coherent equalities. Those are the ones that actually produce new coherence conditions. They're obtained by looking at proof  trees where we can apply two different proof transformations which conflict with each other by sharing part of the proof that is transformed.  Let us look at such possible conflicting pairs involving the  three cut elimination rules we have introduced so far : 
\begin{itemize}
\item In the case of the following proof tree: 
$$\prftree[r]{cut}
	{ \prftree[r]{cut}
		{\prftree [r]{ax}{A \vdash A}
		}
		{\prftree{\pi_1}{A \vdash B}
		}
		{A \vdash B}
	}
	{\prftree{\pi_2}{B\vdash C}
	}
	{A \vdash C}
$$
one can apply both the left axiom cut transformation and the cut commutation. If we apply the cut commutation, we can then apply the left axiom cut again to obtain : 
$$
\prftree[r]{cut}
	{\prftree{\pi_1}{A \vdash B}
	}
	{\prftree{\pi_2}{B\vdash C}
	}
	{A \vdash C}	
$$
which is the result we also obtain by applying the left axiom cut in the first place. 
\item  In the case of the following proof tree: 
$$
\prftree[r]{cut}
	{ \prftree[r]{cut}
		{\prftree {\pi_1}{A \vdash B}
		}
		{\prftree[r]{ax}{B \vdash B}
		}
		{A \vdash B}
	}
	{\prftree{\pi_2}{B\vdash C}
	}
	{A \vdash C}
$$
one can apply both the right axiom cut transformation and the cut commutation. If we apply the cut commutation, we can then apply the left axiom cut  to obtain : 
$$
\prftree[r]{cut}
	{\prftree{\pi_1}{A \vdash B}
	}
	{\prftree{\pi_2}{B\vdash C}
	}
	{A \vdash C}	
$$
which is the result we also obtain by applying the right axiom cut in the first place. 

\item This case is the dual of the first one, with the right axiom cut transformation.
$$\prftree[r]{cut}
	{ \prftree[r]{cut}
		{\prftree{\pi_1}{A \vdash B}
		}
		{\prftree{\pi_2}{B \vdash C}
		}
		{A \vdash C}
	}
	{\prftree[r]{ax}{C \vdash C}
	}
	{A \vdash C}
$$
\item In the case of the following proof tree : 
$$
\prftree[r]{cut}
	{\prftree[r]{cut}
		{\prftree[r]{cut}
			{\prftree{\pi_1}{A \vdash B}}
			{\prftree{\pi_2}{B\vdash C}}
			{A \vdash C}
		}
		{\prftree{\pi_3}{C \vdash D}}
		{A \vdash D}
	}
	{\prftree{\pi_4}{D\vdash E}}
	{A \vdash E}
$$
We can apply the cut commutation in two different ways, either on the top two cut rules or on the bottom two cut rules. If we start with the top two, we obtain : 
$$
\prftree[r]{cut}
	{\prftree[r]{cut}
		{\prftree{\pi_1}{A \vdash B}}
		{\prftree[r]{cut}
			{\prftree{\pi_2}{B \vdash C}}
			{\prftree{\pi_3}{C\vdash D}}
			{B \vdash D}
		}
		{A \vdash D}	
	}
	{\prftree{\pi_4}{D\vdash E}}
	{A \vdash E}
$$
we can then apply it again on the bottom two to obtain : 
$$
\prftree[r]{cut}
	{\prftree{\pi_1}{A\vdash B}}
	{\prftree[r]{cut}
		{\prftree[r]{cut}
			{\prftree{\pi_2}{B \vdash C}}
			{\prftree{\pi_3}{C\vdash D}}
			{B \vdash D}
		}
		{\prftree{\pi_4}{D \vdash E}}
		{B \vdash E}	
	}
	{A \vdash E}
$$
And we can apply it once more to the top two to finally obtain : 
$$
\prftree[r]{cut}
	{\prftree{\pi_1}{A\vdash B}}
	{\prftree[r]{cut}
		{\prftree{\pi_2}{B \vdash C}}
		{\prftree[r]{cut}
			{\prftree{\pi_3}{C \vdash D}}
			{\prftree{\pi_4}{D\vdash E}}
			{C \vdash E}
		}
		{B \vdash E}	
	}
	{A \vdash E}
$$
\\
On the other end, if we start by applying the cut commutation to the bottom two, we can then apply it once more to the only pair available to obtain the same result.
\end{itemize}
One can see that there are no other potential conflict pairs \todo{errr left and right axiom ???}, due to the structure of the various transformations. Those will give us the four coherence conditions appearing in the proper definition of a bicategory that we now introduce : 

\end{itemize}
\begin{definition}\label{definition/bicategory}
A bicategory $\mathcal{C}$ consists of :
\begin{itemize}
\item A collections of objects $A,B,C$.
\item For each pair of objects $A,B$, a category $\mathcal{C}(A,B)$ whose objects are called morphisms or $1-$cells and whose morphisms are called $2$-morphisms or $2-$cells.
\item For each object $A$, a distinguished $1-$cell $id_A\in \mathcal{C}(A,A)$ called the identity morphism.
\item For each triple of objects $A,B,C$ a functor $$\circ : \mathcal{C}(A,B) \times \mathcal{C}(B,C) \rightarrow \mathcal{C}(A,C)$$ called horizontal composition.
\item For each pair of objects $A,B$, two natural isomorphisms called the left and right unitors: $$l:  id_A \circ - \Rightarrow - :\mathcal{C}(A,B)\rightarrow \mathcal{C}(A,B) \text{ and } r:    - \circ id_B \Rightarrow - :\mathcal{C}(A,B)\rightarrow \mathcal{C}(A,B) $$
\item For each quadruple of objects $A,B,C,D$ a natural isomorphism called the associator $$a: (- \circ -) \circ - \Rightarrow - \circ ( - \circ -): \mathcal{C}(A,B) \times \mathcal{C}(B,C) \times \mathcal{C}(C,D)  \rightarrow \mathcal{C}(A,D)$$
\end{itemize}
such that the following diagrams commute for any object $A,~B,~C,~D,~E$ of $\mathcal{C}$ and $f,~g,~h,~i$ objects of $\mathcal{C}(A,B),~\mathcal{C}(B,C),~\mathcal{C}(C,D),~\mathcal{C}(D,E)$ respectively :
$$\xymatrix @-1.2pc {
((f \circ g) \circ h) \circ i
\ar[rrrr]_-{a(f,g,h) \circ Id_i}
\ar[dd]_-{a(f\circ g, h,i)}
&&&&
(f \circ (g \circ h)) \circ i
\ar[dd]_-{a(f, g \circ h, i)}
\\
\\
(f \circ g) \circ (h \circ i)
\ar[ddrr]_-{a(f,g,h \circ i)}
&&&&
f \circ ((g \circ h) \circ i)
\ar[ddll]^-{ Id_f \circ a(g,h,i)}
\\
\\
&&
f \circ(g \circ (h \circ i))
}
$$




$$
\xymatrix @-1.2pc {
(f \circ Id_B) \circ g
\ar[rrrr]_-{a(f, id_B, g)}
\ar[ddrr]_-{r(f) \circ Id(g) }
&&&&
f \circ (Id_B \circ g)
\ar[ddll]^-{Id_f \circ l(g) }
\\
\\
&&
f \circ g
}
$$

$$
\xymatrix @-1.2pc {
(id_A \circ f) \circ g
\ar[rrrr]_-{a(id_A, f, g)}
\ar[ddrr]_-{l(f) \circ Id(g) }
&&&&
id_A \circ (f \circ g)
\ar[ddll]^-{l( f\circ g) }
\\
\\
&&
f \circ g
}
$$
$$
\xymatrix @-1.2pc {
(f \circ g) \circ id_C
\ar[rrrr]_-{a(f, g, id_C)}
\ar[ddrr]_-{r(f \circ g) }
&&&&
f \circ (g \circ id_C)
\ar[ddll]^-{Id_f \circ r(g) }
\\
\\
&&
f \circ g
}
$$

\end{definition}
We then introduce further bicategorical notions that will prove useful, such as the notion of pseudofunctor, a morphism between bicategories preserving the coherence conditions : 
\begin{definition}\label{definition/pseudofunctor}
Let $\mathcal{C}, \mathcal{D}$ be two bicategories. A pseudofunctor $F:\mathcal{C}\rightarrow \mathcal{D}$ is given by :
\begin{itemize}
\item For each object $A$ of $\mathcal{C}$, an object $F(A)$ of $\mathcal{D}$.
\item For each hom-category $\mathcal{C}(A,B)$ in $\mathcal{C}$, a functor $$F(A,B): \mathcal{C}(A,B) \rightarrow \mathcal{D}(F(A),F(B))$$
\item For each object $A$ of $\mathcal{C}$, an invertible $2$-cell $$F_{id_A}: id_{F(A)} \Rightarrow F(A,B)(id_A)$$
\item For each triple of objects $A,B,C$ of $\mathcal{C}$, a natural isomorphism $\phi$ whose elements are:$$\phi_{f,g} :F(f) \circ F(g) \Rightarrow F(f \circ g)$$ for $f,g$ objects of $\mathcal{C}(A,B), \mathcal{C}(B,C)$ respectively
\end{itemize}
such that, for any object $A,~B,~C,~D$ of $\mathcal{C}$, any objects $f,~g,~h$ of $\mathcal{C}(A,B),~\mathcal{C}(B,C),~\mathcal{C}(C,D)$ respectively, the following diagrams commute :

$$\xymatrix @-1.2pc {
F(f) \circ (F(g) \circ F(h))
\ar[dd]_-{Id_{F(f)} \circ \phi_{g,h} }
&&&&
(F(f) \circ F(g)) \circ F(h)
\ar[llll]_-{a(F(f),F(g),F(h)}
\ar[dd]_-{\phi_{f,g} \circ Id_{F(h)}}
\\
\\
F(f) \circ F(g \circ h)
\ar[dd]_-{\phi_{f,g \circ h} }
&&&&
F(f \circ g) \circ F(h)
\ar[dd]_-{\phi_{f \circ g, h} }
\\
\\
F(f\circ(g\circ h))
&&&&
F((f \circ g) \circ h)
\ar[llll]_-{F(a(f,g,h))}
}
$$
$$
\xymatrix @-1.2pc {
F(f)
&&&
F(f \circ id_B)
\ar[lll]_-{F(r(f))}
\\
\\
F(f) \circ id_{F(B)}
\ar[uu]_-{r(F(f))}
\ar[rrr]_-{ Id_{F(f)}  \circ F_{id_B} }
&&&
F(f) \circ F(id_B)
\ar[uu]_-{\phi_{f, id_B}}
}
$$

$$
\xymatrix @-1.2pc {
F(f)
&&&
F(id_A \circ f )
\ar[lll]_-{F(l(f))}
\\
\\
 id_{F(A)} \circ F(f) 
\ar[uu]_-{l(F(f))}
\ar[rrr]_-{ F_{id_A}  \circ Id_{F(f)}  }
&&&
 F(id_A) \circ F(f) 
\ar[uu]_-{\phi_{id_A, f}}
}
$$
\end{definition}
Note that every space of  proof transformations between proofs of a fixed statement must be modelized by a category, which is very natural, as this space is natively a category with composition being the application of proof transformations sequentially and the identity being the proof transformation that does nothing. \\

Next, like in categorical theory in chapter $2$, we introduce morphisms between pseudofunctors in the form of pseudo-natural transformations.
Starting from this definition and for the rest of the chapter, we'll represent the coherence conditions between $2-$ morphisms in a way that is more visual. For example, a $2-$morphism $t:f\Rightarrow g$ where $f,g:A\rightarrow B$ are morphisms, will be represented : 
$$\begin{tikzcd}
A
\arrow[bend left]{rr}{f} [name=U]{}
\arrow[bend right]{rr}[swap]{g}[name=D]{}
 &&
B
  \arrow[Rightarrow,from=U, to=D,"t"]
\end{tikzcd}$$ 
Combinations of $2-$ morphisms are then represented geometrically in a type of figure called pasting diagrams, and the coherence conditions are described as an equality of pasting diagrams. One thing of note is that those pasting diagrams hide the use of the associator, as composition is simply represented as a sequence of arrows, without any notion of priority. \\

\begin{definition}
Let $\mathcal{C}, \mathcal{D}$ be two bicategories, and $F,G:\mathcal{C} \rightarrow \mathcal{D}$ two pseudofunctors. A pseudo-natural transformation $\gamma: F \Rightarrow G$ is given by :
\begin{itemize}
\item for every object $A$ of $\mathcal{C}$, a $1-$cell $\gamma_A: F(A) \rightarrow G(A)$
\item for every pair of objects $A,B$ of $\mathcal{C}$ and every $1-$cell $f$ of $\mathcal{C}(A,B)$, an invertible $2-$cell $\gamma_f$:
$$\begin{tikzcd}
  F(A) 
  \arrow{r}{\gamma_A}
  \arrow{d}[swap]{F(f)}
  &
   G(A) 
   \arrow{d}{G(f)}
  \\
  F(B) 
  \arrow{r}[swap]{\gamma_B}
  &
  \twocell{ul}{\gamma_f}
  G(B)
\end{tikzcd}$$

\end{itemize} 
such that the following properties are verified : 
\begin{itemize}
\item Naturality : For every $2-$cell $\tau:f \Rightarrow g : A \rightarrow B$, the $2-$cells associated to the following pasting diagrams are equal : 

$$\begin{forcedcentertikzcd}
  F(A) 
  \arrow{rr}{\gamma_A}
  \arrow{dd}[swap]{F(f)}
  &&
   G(A) 
   \arrow[bend right]{dd}[swap]{G(f)} [name=U]{}
   \arrow[bend left]{dd}{G(g)}[name=D]{}
   \arrow[Rightarrow,from=U, to=D,"G(\tau)"]
   &&
  F(A) 
  \arrow{rr}{\gamma_A}
   \arrow[bend right]{dd}[swap]{F(f)} [name=E]{}
   \arrow[bend left]{dd}{F(g)}[name=R]{}
  \arrow[Rightarrow,from=E, to=R,"F(\tau)"]
   &&
   G(A) 
   \arrow{dd}{G(g)}
  \\
  &&&=&&&
  \\
  F(B) 
  \arrow{rr}[swap]{\gamma_B}
  &&
  \twocell{uull}{\gamma_f}
  G(B)
 &&
  F(B) 
  \arrow{rr}[swap]{\gamma_B}
  &&
  \twocell{uull}{\gamma_g}
  G(B)
  \end{forcedcentertikzcd}$$

\item Unitality :For every object $A$ of $\mathcal{C}$, the $2-$cells associated to the following pasting diagrams are equal : 
$$\begin{forcedcentertikzcd}
  F(A) 
  \arrow{rr}{\gamma_A}
  \arrow{dd}[swap]{Id_{F(A)}}
  &&
   G(A) 
   \arrow[bend right]{dd}[swap]{Id_{G(A)}} [name=U]{}
   \arrow[bend left]{dd}{G(Id_A)}[name=D]{}
   \arrow[Rightarrow,from=U, to=D,"G_A"]
   &&
  F(A) 
  \arrow{rrr}{\gamma_A}
   \arrow[bend right]{dd}[swap]{Id_{F(A)}} [name=E]{}
   \arrow[bend left]{dd}{F(Id_A)}[name=R]{}
  \arrow[Rightarrow,from=E, to=R,"F_A"]
   &&&
   G(A) 
   \arrow{dd}{G(Id_A)}
  \\
  &\equiv&&=&&&
  \\
  F(A) 
  \arrow{rr}[swap]{\gamma_A}
  &&
%  \twocell{uull}{\gamma_f}
  G(A)
 &&
  F(A) 
  \arrow{rrr}[swap]{\gamma_A}
  &&&
  \twocell{uulll}{\gamma_{Id_A}}
  G(A)
  \end{forcedcentertikzcd}$$
\item Compositionality : for every triple of objects $A,B,C$ of $\mathcal{C}$ and every pair of $1-$cells $f,g$ of $\mathcal{C}(A,B),~\mathcal{C}(B,C)$ respectively, the $2-$ cells associated to the following pasting diagrams are equal : 

$$\begin{forcedcentertikzcd}
F(A)
\arrow{rr}{F(f)}
\arrow[bend right]{rrrr}[swap]{F(g \circ f)} [name=E]{}
\arrow{dd}{\gamma_A}
&&
F(B)
\arrow{rr}{F(g)}
\arrow[Rightarrow,to=E,"F_{g,f}"]
&&
F(C)
\arrow{dd}{\gamma_C}
&&
F(A)
\arrow{rr}{F(f)}
\twocell{rrd}{\gamma_{f}}
\arrow{dd}{\gamma_A}
&&
F(B)
\arrow{rr}{F(g)}
\arrow{d}{\gamma_B}
\twocell{rrd}{\gamma_{g}}
&&
F(C)
\arrow{dd}{\gamma_C}
\\
&&&&&=&&&
G(B)
\arrow[Rightarrow,shorten >=0.3cm,shorten <=0.3cm]{d}{G_{g,f}}
\arrow{rrd}{G(g)}
&&{}
\\
G(A)
\arrow{rrrr}{G(g \circ f)}
&&
&&
G(C)
&&
G(A)
\arrow{rru}{G(f)}
\arrow{rrrr}[swap]{G(g \circ f)}
&&
{}
&&
G(C)
\end{forcedcentertikzcd}$$
\end{itemize}
\end{definition}
Finally for this part, we go one step further by introducing a notion of morphism between pseudo-natural transformations in the form of modifications : 
\begin{definition}
Let $\gamma,\delta:F \Rightarrow G: \mathcal{C} \rightarrow \mathcal{D}$ be two pseudo-natural transformations., a modification $m: \gamma \Rrightarrow \delta$ is given by a $2-$cell $m_A: \gamma_A \Rightarrow \delta_A$ for every object $A$ of $\mathcal{C}$ such that for every $f:A\rightarrow B$ in $\mathcal{C}$, we have : 
$$\begin{tikzcd}
F(A)
  \arrow[bend right]{rr}[swap]{\gamma_A} [name=E]{}
   \arrow[bend left]{rr}{\delta_A}[name=R]{}
  \arrow[Rightarrow, shorten >=0.2cm, shorten <=0.2cm,from=E, to=R,"m_A"]
  \arrow{ddd}{F(f)}
 &&
G(A)
\arrow{ddd}{G(f)}
&&
F(A)
\arrow{rr}{\delta_A}
\arrow{ddd}{F(f)}
&{}&
G(A)
\arrow{ddd}{G(f)}
\\
{}&{}&{}
{}&{}&{}
{}&{}&{}
\\
\\
F(B)
\arrow{rr}{\gamma_B}
 &&
G(B)
\twocell{uull}{\gamma_f}
&&
F(B)
  \arrow[bend right]{rr}[swap]{\gamma_B} [name=T]{}
   \arrow[bend left]{rr}{\delta_B}[name=Y]{}
  \arrow[Rightarrow, shorten >=0.2cm, shorten <=0.2cm,from=T, to=Y,"m_B"]
&&
G(B)
\twocell{uuull}{\delta_f}
\end{tikzcd}$$

\end{definition}

\todo{move to relevant section}
\begin{definition}
Let $A,B$ be two objects in a bicategory $\mathcal{C}$. An equivalence from $A$ to $B$ is given by :
\begin{itemize}
\item a pair of $1-$cells $f:A\rightarrow B$ and $g:B \rightarrow A$.
\item a pair of invertible $2-$cells $e:id_A \Rightarrow g \circ f$ and $e': id_B \Rightarrow f \circ g$.
\end{itemize}
We say that $f$ is an equivalence if such $g,e,e'$ exist. 
\end{definition}

\section{MILL and monoidal bicategories}
 We now move up to the first proper fragment of Linear Logic that we will study,  multiplicative intuitionistic linear logic (MILL), whose rules we have seen in Chapter 2 and are recalled here in Figure \ref{figure/MILLrules}. What arises very quickly by adapting our study from chapter 2 is the apparent need of $\otimes$ being a pseudo-bifunctor. %\todo{add proper def}\\

 \begin{figure}
 \begin{center}
$$ \begin{array}{ccccccc}
 \prftree[r]{ax}{A\vdash A}&& 
 \prftree[r]{l$-\otimes$}{\Gamma,A,B \vdash C}{\Gamma, A\otimes B \vdash C}&&
 \prftree[r]{r$-\otimes$}{\Gamma \vdash A}{\Delta \vdash B}{\Gamma,\Delta \vdash A \otimes B}&&
 \prftree[r]{cut}{\Gamma \vdash A}{A,\Delta \vdash B}{\Gamma,\Delta \vdash B}\\
 &&&&&&\\
 \prftree[r]{const}{\vdash I}&&
 \prftree[r]{l$-\multimap$}{\Gamma \vdash A}{B,\Delta \vdash C}{\Gamma,\Delta,A\multimap B \vdash C}&&
 \prftree[r]{r$-\multimap$}{\Gamma ,  A \vdash B}{\Gamma \vdash A \multimap B}&&
 \prftree[r]{exch}{\Gamma,A,B,\Delta \vdash C}{\Gamma,B,A,\Delta \vdash C}\\
 \end{array}$$
 \end{center}
 \caption{Rules for MILL \label{figure/MILLrules}}
 \end{figure}


Indeed, the required functoriality of $\otimes$ in the hom-categories rises naturally from the categorical structure of proof transformations over proofs of a fixed statement. \\

Before looking at the proof transformations involving $\otimes$ and the required $2-$morphisms to complete the definition of bifunctor, let us get back at the first proof transformation we introduced in this chapter : 
$$
\prftree[r]{cut}
	{\prftree[r]{ax}
		{A\vdash A}
	}
	{\prftree{\pi_1}{A,\Delta \vdash B}
	}
	{A,\Delta \vdash B}
\Rightarrow
\prftree{\pi_1}{A,\Delta \vdash B}
$$
We claimed earlier that this proof transformation was modelized by the $2-$ morphism $$id_A \circ [\pi_1] \Rightarrow [\pi_1]$$ which is a bit of a shortcut as this is only strictly true when $\Delta$ is empty. This serves to highlight one of the main difference between the cut rule of linear logic and the categorical notion of composition. For a composition to be allowed in a categorical or bicategorical setting, the domain of one morphism must be equal to the codomain of the other, while in logic, for a cut rule to be applicable, there only needs to be a common formula in the conclusion of one statement and the hypothesis of another. This basically means that cut is akin to some kind of partial composition. Hence, the proper $2-$morphism associated to this proof transformation would intuitively be : $$(id_A \otimes id_{\Delta} \circ [\pi_1] \Rightarrow [\pi_1]$$
%Though the codomain of this $2-$ morphism is not the model of the initial proof $$
%\prftree[r]{cut}
%	{\prftree[r]{ax}
%		{A\vdash A}
%	}
%	{\prftree{\pi_1}{A,\Delta \vdash B}
%	}
%	{A,\Delta \vdash B}
%$$
%it is the model of a proof 
%but the proof $$
%\prftree[r]{cut}
%	{\prftree[r]{$r-\otimes$}
%		{\prftree[r]{ax}
%			{A\vdash A}
%		}
%		{
%		\prftree[r]{ax}
%			{\Delta\vdash \Delta}
%		}
%	{A,\Delta \vdash A, \Delta}
%	}
%	{\prftree{\pi_1}{A,\Delta \vdash B}
%	}
%	{A,\Delta \vdash B}$$

Note that in this case, this shortcut is of no consequence as all the context data can be handled through use of $\otimes$ and the associated rules. However, it was important to highlight as the proof transformation rules for $\otimes$ make heavy use of the partial composition of cut in ways that don't allow as easy a shortcut. Let us look at those rules : 
$$%\scalebox{0.8}{
 \begin{array}{cccc}
\eta \otimes
&
 \prftree[r]{ax}
	{A\otimes B \vdash A \otimes B} 
&
\Rightarrow
&
\prftree[r]{$l-\otimes$}
	{\prftree[r]{$r-\otimes$}
		{\prftree[r]{ax}{A\vdash A}}
		{\prftree[r]{ax}{B \vdash B}}
		{A,B \vdash A \otimes B}
	}
	{A \otimes B \vdash A \otimes B}
\\
\\
\text{ r-$\otimes$  l-$\otimes$  } cut
&
\prftree[r]{cut}
	{\prftree[r]{$r-\otimes$}
		{\prftree{\pi_1}{A\vdash B}}
		{\prftree{\pi_2}{C \vdash D}}
		{A,C \vdash B \otimes D}
	}
	{\prftree[r]{$l-\otimes$}
		{\prftree{\pi_3}{B,D \vdash E}}
		{B \otimes D \vdash E}
	}
	{A,C  \vdash E}
&
\Rightarrow
&
\prftree[r]{cut}
	{\prftree{\pi_1}{A\vdash B}}
	{\prftree[r]{cut}
		{\prftree{\pi_2}{C \vdash D}}
		{\prftree{\pi_3}{B,D \vdash E}}
		{B,C\vdash E}
	}
	{A,C\vdash E}
\\
\\
\text{r-$\otimes$ } rcut_1
&
\prftree[r]{cut}
	{\prftree{\pi_1}{A\vdash B}}
	{
	\prftree[r]{$r-\otimes$}
		{\prftree{\pi_2}{B \vdash D}}
		{\prftree{\pi_3}{C \vdash E}}
		{B,C \vdash D \otimes E}
	}
	{A,C \vdash D \otimes E}
&
\Rightarrow
&
\prftree[r]{$r-\otimes$}
	{\prftree[r]{cut}
		{\prftree{\pi_1}{A\vdash B}}
		{\prftree{\pi_2}{B \vdash D}}
		{A \vdash D}
	}
	{\prftree{\pi_3}{C \vdash E}}
	{A,C \vdash D \otimes E}
\\
\\
\text{r-$\otimes$ } rcut_2
&
\prftree[r]{cut}
	{\prftree{\pi_1}{A\vdash B}}
	{
	\prftree[r]{$r-\otimes$}
		{\prftree{\pi_2}{C \vdash D}}
		{\prftree{\pi_3}{B \vdash E}}
		{C,B \vdash D \otimes E}
	}
	{C,A \vdash D \otimes E}
&
\Rightarrow
&
\prftree[r]{$r-\otimes$}
	{\prftree{\pi_2}{C \vdash D}}
	{\prftree[r]{cut}
		{\prftree{\pi_1}{A\vdash B}}
		{\prftree{\pi_3}{B \vdash E}}
		{A \vdash E}
	}
	{C,A \vdash D \otimes E}
\\
\\
\text{commut }cut_2
&
 \prftree[r]{cut}
	{\prftree{\pi_1}{A\vdash B}}
	{\prftree[r]{cut}
		{\prftree{\pi_2}{C\vdash D}}
		{\prftree{\pi_3}{B,D\vdash E}}
		{A,D\vdash E}
	}
	{A,C 
\vdash E}
&
\Rightarrow
&
 \prftree[r]{cut}
	{\prftree{\pi_2}{C\vdash D}}
	{\prftree[r]{cut}
		{\prftree{\pi_1}{A\vdash B}}
		{\prftree{\pi_3}{B,D\vdash E}}
		{B,C\vdash E}
	}
	{A,C \vdash E}
\\
\\
\text{l-$\otimes$ } lcut
&
 \prftree[r]{cut}
	{
		\prftree[r]{$l-\otimes$}
		{\prftree{\pi_1}{A,B\vdash C}}
		{A \otimes B \vdash C}
	}
	{\prftree{\pi_2}{C\vdash D}}
	{A \otimes B \vdash D}
&
\Rightarrow
&
\prftree[r]{$l-\otimes$}
	{
	\prftree[r]{cut}
		{\prftree{\pi_1}{A,B\vdash C}}
		{\prftree{\pi_2}{C\vdash D}}
		{A,B \vdash D}
	}
	{A \otimes B \vdash D}
\\
\\
\text{l-$\otimes$ } rcut
&
\prftree[r]{cut}
	{\prftree{\pi_1}{A\vdash B}}
	{
		\prftree[r]{l-$\otimes$}
			{\prftree{\pi_2}{B,C,D\vdash E}}
			{B,C\otimes D \vdash E}
	}
	{A,C\otimes D \vdash E}
&
\Rightarrow
&
\prftree[r]{l-$\otimes$}
	{
		\prftree[r]{cut}
			{\prftree{\pi_1}{A\vdash B}}
			{\prftree{\pi_2}{B,C,D\vdash E}}
		{A,C,D \vdash E}
	}	
	{A,C\otimes D \vdash E}	
\end{array}%}
$$
which turn into the following $2-$ morphisms:
$$id_{A\otimes B} \Rightarrow id_A \otimes id_B$$
$$([\pi_1] \otimes [\pi_2]) \circ [\pi_3] \Rightarrow ([\pi_1] \otimes id_C) \circ ((id_B \otimes [\pi_2]) \circ [\pi_3])$$
$$([\pi_1] \otimes id_C) \circ ([\pi_2] \otimes [\pi_3]) \Rightarrow ([\pi_1] \circ [\pi_2]) \otimes [\pi_3] $$
$$(  id_C \otimes [\pi_1]) \circ ([\pi_2] \otimes [\pi_3]) \Rightarrow  [\pi_2] \otimes ([\pi_1] \circ[\pi_3])$$
$$([\pi_1] \otimes id_C) \circ ((id_B \otimes [\pi_2]) \circ \pi_3) \Rightarrow (id_A \otimes [\pi_2]) \circ (([\pi_1] \otimes id_D) \circ [\pi_3]) $$
$$id_{[\pi_1]\circ [\pi_2]}$$
$$id_{([\pi_1] \otimes (id_C \otimes id_D)) \circ [\pi_2]}$$
That first $2-$morphism is exactly one of those required by the definition of pseudofunctor. As for the next three, they can all be seen as specific cases of another, more general $2-$morphism, though with some additional $2-$morphisms composed to handle the added identities. The more general morphism is as such : 

$$(f \otimes g) \circ (h \otimes i) \Rightarrow (f\circ h) \otimes (g \circ i)$$ with $f: A\rightarrow B, g:C \rightarrow D, h:B \rightarrow E, i: D\rightarrow F$.\\

This $2-$morphism is exactly the second morphism required by the definition of a pseudo-functor ($F(f) \circ F(g) \Rightarrow F(f \circ g)$). Requiring the existence of this $2-$morphism would thus make $\otimes$ a pseudo-functor in our model, and be enough to modelize the related proof transformations. \\ Moreover, this requirement is not a tremendous restriction on the logic, as the proof transformation associated to that $2-$morphism, given by : 
$$ \begin{array}{ccc}
 \prftree[r]{cut}
	{
		\prftree[r]{$l-\otimes$}
		{
			\prftree[r]{$r-\otimes$}
			{\prftree{\pi_1}{A\vdash B}}
			{\prftree{\pi_2}{C\vdash D}}
			{A , C \vdash B \otimes D}
		}
		{A \otimes C \vdash B \otimes D}
	}
	{
		\prftree[r]{$l-\otimes$}
		{
			\prftree[r]{$r-\otimes$}
			{\prftree{\pi_3}{B\vdash E}}
			{\prftree{\pi_4}{D\vdash F}}
			{B , D \vdash E \otimes F}
		}
		{B \otimes D \vdash E \otimes F}
	}
	{A \otimes C \vdash E \otimes F}
&
\Rightarrow
&
\prftree[r]{$l-\otimes$}
	{
		\prftree[r]{$r-\otimes$}
			{
				\prftree[r]{cut}
					{\prftree{\pi_1}{A\vdash B}}
					{\prftree{\pi_3}{B\vdash E}}
					{A \vdash E}
			}
			{
				\prftree[r]{cut}
					{\prftree{\pi_2}{C\vdash D}}
					{\prftree{\pi_4}{D\vdash F}}
					{C \vdash F}
			}
			{A , C \vdash E \otimes F}
	}
	{A \otimes C \vdash E \otimes F}
\end{array}
$$
can be generated from the proof transformations we have outlined. Indeed, starting from the initial proof, we can use in sequence : $$\text{l-$\otimes$ } cut
 \cdot \text{r-$\otimes$ l-$\otimes$ } cut  \cdot \text{r-$\otimes$ } cut_2 \cdot \text{r-$\otimes$ } cut_1$$ to obtain the second proof. \\


Next, the $\text{commut } cut_2$ rule can be interpreted as a bifunctorial property, which means we only have two proof transformation rules left to study, 
$\text{l-$\otimes$ } lcut$ and $\text{l-$\otimes$ } rcut$. Those two outline a glitch in the usual definition of categorical model of $LL$. Indeed, the choice of interpreting contexts $A_1,...A_n$ as a tensor product of objects $[A_1] \otimes ... \otimes [A_n]$ creates an impossibility to dinstinguish an interpretation of $A,B \vdash C$ and an interpretation of $A \otimes B \vdash C$. This results in particular in difficulties in interpreting the $l-\otimes$ proof rule and all proof transformations that it induces. The role of the $l-\otimes$ proof rule can be seen as giving an order of priority to the applications of the tensor product, whether effective or implied by context. Indeed, a proof of $A,B,C \vdash D$ can be turned either into  a proof of $A \otimes (B \otimes C) \vdash D$ or $(A\otimes B) \otimes C \vdash D$ through simple applications of the $l-\otimes$ rule. Those two proofs would not a priori be interpreted in the same way as we have no prior associativity hypothesis on $\otimes$. we will thus require such associativity in our model, which, in bicategorical terms, take the form of a pseudo-natural transformation $$a_{A,B,C}: (A \otimes B) \otimes C \rightarrow A \otimes (B\otimes C)$$  equipped with an invertible modification $$\begin{tikzcd}
((A \otimes B) \otimes C) \otimes D
\arrow{r}{a_{A \otimes B,C,D}}
\arrow{d}{a_{A,B,C} \otimes id_D}
&
(A \otimes B) \otimes (C \otimes D)
\arrow{r}{a_{A,B,C\otimes D}}
\arrow[Rightarrow,shorten >=0.3cm,shorten <=0.3cm]{d}{\pi_{A,B,C,D}}
&
A \otimes ( B \otimes (C \otimes D))
\\
(A \otimes (B \otimes C)) \otimes D
\arrow{rr}{a_{A, B\otimes C, D}}
&{}&
A \otimes ((B \otimes C) \otimes D)
\arrow{u}{id_A \otimes a_{B,C,D}}
\end{tikzcd}$$
Similarly to the categorical case, the constant $I$ must be interpreted as a unit of the bifunctor, which, in a bicategorical setting, induces a few $2-$morphisms related to the equalities in the categorical setting.\\

From this, using the same idea of conflict between two potential proof transformation that we used earlier this chapter, we can infer the equalities of $2-$morphisms required by the pseudo-functor definition, and a few additional ones between our new $2-$morphisms, which are the ones included in the definition of a monoidal bicategory, which follows : 
\begin{definition}
A monoidal bicategory $\mathcal{C}$ is a bicategory equipped with : 
\begin{itemize}
\item a unit object $I$.
\item a pseudo-functor $\otimes : \mathcal{C} \times \mathcal{C} \rightarrow \mathcal{C}$
\item three pseudo-natural transformations $a,l,r$ whose components are equivalences and given by : 
$$a_{A,B,C}: (A \otimes B) \otimes C \rightarrow A \otimes (B\otimes C)$$
$$l_A : I \otimes A \rightarrow A$$
$$r_A : A \otimes I \rightarrow A$$
\item four invertible modifications $\pi, \mu, L, R$ whose components are given by :
$$\begin{tikzcd}
((A \otimes B) \otimes C) \otimes D
\arrow{r}{a_{A \otimes B,C,D}}
\arrow{d}{a_{A,B,C} \otimes id_D}
&
(A \otimes B) \otimes (C \otimes D)
\arrow{r}{a_{A,B,C\otimes D}}
\arrow[Rightarrow,shorten >=0.3cm,shorten <=0.3cm]{d}{\pi_{A,B,C,D}}
&
A \otimes ( B \otimes (C \otimes D))
\\
(A \otimes (B \otimes C)) \otimes D
\arrow{rr}{a_{A, B\otimes C, D}}
&{}&
A \otimes ((B \otimes C) \otimes D)
\arrow{u}{id_A \otimes a_{B,C,D}}
\end{tikzcd}$$
$$\begin{tikzcd}
(A \otimes I) \otimes C
\arrow{rd}[swap]{r_A \otimes id_C} [name=T]{}
\arrow{rr}{a_{A,I,C}}
&&
A \otimes (I \otimes C)
\arrow{ld}{id_A \otimes l_C} [swap,name=V]{}
 \arrow[Rightarrow, shorten >=0.7cm, shorten <=0.7cm,from=T, to=V,"\mu_{A,C}"]
\\
&
A \otimes C
&
\end{tikzcd}$$
$$\begin{forcedcentertikzcd}
(I \otimes B) \otimes C
\arrow{rd}[swap]{l_B \otimes id_C} [name=T]{}
\arrow{rr}{a_{I,B,C}}
&&
I \otimes (B \otimes C)
\arrow{ld}{l_{B\otimes C}} [swap,name=V]{}
 \arrow[Rightarrow, shorten >=0.7cm, shorten <=0.7cm,from=T, to=V,"L_{B,C}"]
 &
 (A \otimes B) \otimes I
\arrow{rd}[swap]{r_{A \otimes B}} [name=Y]{}
\arrow{rr}{a_{A,B,I}}
&&
A \otimes (B \otimes I)
\arrow{ld}{id_A \otimes r_B} [swap,name=U]{}
 \arrow[Rightarrow, shorten >=0.7cm, shorten <=0.7cm,from=Y, to=U,"R_{A,B}"]
\\
&
B \otimes C
&
&
&
A \otimes B
&
\end{forcedcentertikzcd}$$
\end{itemize}
such that the following conditions are verified :
\begin{itemize}
\item Associativity : For all $A,B,C,D,E$ objects of $\mathcal{C}$, the two following  pasting diagrams must be equal:\\
\begin{landscape}
$
\scalebox{0.7}{
\begin{forcedcentertikzcd}[ampersand replacement=\&, column sep = small]
\&
(((A \otimes B) \otimes C) \otimes D) \otimes E
\arrow{rr}{a_{(A\otimes B) \otimes C,D,E}}
\arrow{rdd}{a_{A\otimes B,C,D} \otimes id_E}
\arrow{ld}{(a_{A,B,C} \otimes id_D) \otimes id_E}
\&\&
((A \otimes B) \otimes C) \otimes (D \otimes E)
\arrow{rr}{a_{A\otimes B,C,D\otimes E}}
\arrow[Rightarrow, shorten >=0.7cm, shorten <=1.2cm]{dd}{\pi_{A\otimes B,C,D,E} }
\&\&
(A \otimes B) \otimes (C \otimes (D \otimes E))
\arrow{rdd}{a_{A,B,C\otimes (D \otimes E)}}
\arrow[Rightarrow, shorten >=1cm, shorten <=3cm]{dddd}[near end]{a_{id_A,id_B,a_{C,D,E}}}
\&
\\
((A \otimes (B \otimes C)) \otimes D) \otimes E
\arrow{dd}{a_{A,B\otimes C,D} \otimes id_E}
\\
{}\&\&
((A \otimes B) \otimes (C \otimes D)) \otimes E
\arrow[Rightarrow, shorten >=3.2cm, shorten <=3.2cm]{ll}{\pi_{A,B,C,D} \otimes Id_{id_E}}
\arrow{ldd}{a_{A,B,C\otimes D} \otimes id_E}
\arrow{rr}{a_{A \otimes B, C \otimes D, E}}
\&{}\&
(A \otimes B) \otimes ((C \otimes D) \otimes E)
\arrow{rdd}[swap]{a_{A,B,(C \otimes D) \otimes E}}
\arrow[bend left=0]{ruu}{id_{A\otimes B} \otimes a_{C,D,E}}[name=T]{}
\arrow[bend right]{ruu}[swap]{(id_{A}\otimes id_{B}) \otimes a_{C,D,E}}[name=Y]{}
\arrow[Rightarrow, shorten >=1.8cm, shorten <=1.8cm]{ddl}{\pi_{A,B,C\otimes D,E}}
\arrow[Rightarrow, shorten >=0.4cm, shorten <=0.4cm,from=T, to=Y,"\otimes_{id_A,id_B} \otimes Id_{a_{C,D,E}}"]
\&\&
A \otimes (B \otimes (C \otimes (D \otimes E)))
\\
(A \otimes ((B \otimes C) \otimes D)) \otimes E
\arrow{rd}{(id_A \otimes a_{B,C,D}) \otimes id_E}
\\
\&
(A \otimes (B \otimes (C \otimes D))) \otimes E
\arrow{rr}{a_{A,B \otimes (C \otimes D),E}}
\&\&
A \otimes ((B \otimes (C \otimes D)) \otimes E)
\arrow{rr}{id_A \otimes a_{B, C\otimes D, E}}
\&\&
A \otimes (B \otimes ((C \otimes D) \otimes E))
\arrow{ruu}[swap]{id_A \otimes ( id_B \otimes a_{C,D,E})}
\&
\end{forcedcentertikzcd}}$
\\[6.5cm]
$\scalebox{0.7}{
\begin{forcedcentertikzcd}[ampersand replacement=\&, column sep = small]
\&
(((A \otimes B) \otimes C) \otimes D) \otimes E
\arrow{rr}{a_{(A\otimes B) \otimes C,D,E}}
\arrow{ld}[swap]{(a_{A,B,C} \otimes id_D) \otimes id_E}
\&{}
\arrow[Rightarrow, shorten >=1.1cm, shorten <=3.3cm]{dl}[swap, near end]{a_{a_{A,B,C},id_D,id_E}^{-1}}
\&
((A \otimes B) \otimes C) \otimes (D \otimes E)
\arrow{rr}{a_{A\otimes B,C,D\otimes E}}
\arrow[bend right=10]{dl}[swap, near end]{a_{A,B,C} \otimes (id_{D}  \otimes id_{E})}[name=T]{}
\arrow[bend left=10]{dl}[near end]{a_{A,B,C} \otimes id_{D \otimes E}}[name=Y]{}
\arrow[Rightarrow, shorten >=0.4cm, shorten <=0.4cm,from=Y, to=T,swap,"Id_{a_{A,B,C}} \otimes~\otimes_{id_D,id_E}"]
\&\&
(A \otimes B) \otimes (C \otimes (D \otimes E))
\arrow{rdd}{a_{A,B,C\otimes (D \otimes E)}}
\arrow[Rightarrow, shorten >=1.9cm, shorten <=1.9cm]{ddl}{\pi_{A,B,C,D\otimes E}}
\&
\\
((A \otimes (B \otimes C)) \otimes D) \otimes E
\arrow{dd}{a_{A,B\otimes C,D} \otimes id_E}
\arrow{rr}[swap]{a_{A\otimes(B\otimes C),D,E}}
\&{}\&
(A \otimes (B \otimes C)) \otimes (D \otimes E)
\arrow{rrd}{a_{A,B\otimes C, D\otimes E}}
\arrow[Rightarrow, shorten >=1.9cm, shorten <=1.9cm]{ddl}{\pi_{A,B\otimes C,D, E}}
\\
\&\&
\&\&
A \otimes ((B \otimes C) \otimes (D \otimes E))
\arrow{rr}{id_A \otimes a_{B,C,D\otimes E}}
\arrow[Rightarrow, shorten >=0.9cm, shorten <=0.9cm]{dd}{Id_{id_A} \otimes \pi_{,B,C,D, E}}
\&\&
A \otimes (B \otimes (C \otimes (D \otimes E)))
\\
(A \otimes ((B \otimes C) \otimes D)) \otimes E
\arrow{rd}{(id_A \otimes a_{B,C,D}) \otimes id_E}
\arrow{rr}{a_{A,(B \otimes C) \otimes D, E}}
\&{}\&
A \otimes (((B \otimes C) \otimes D) \otimes E)
\arrow{rru}{id_A \otimes a_{B \otimes C,D,E}}
\arrow{dr}{id_A \otimes ( a_{B,C,D} \otimes id_E)}
\arrow[Rightarrow, shorten >=1.1cm, shorten <=1.1cm]{dl}[swap]{a_{id_A,a_{B,C,D},id_E}^{-1}}
\\
\&
(A \otimes (B \otimes (C \otimes D))) \otimes E
\arrow{rr}{a_{A,B \otimes (C \otimes D),E}}
\&\&
A \otimes ((B \otimes (C \otimes D)) \otimes E)
\arrow{rr}{id_A \otimes a_{B, C\otimes D, E}}
\&{}\&
A \otimes (B \otimes ((C \otimes D) \otimes E))
\arrow{ruu}{id_A \otimes ( id_B \otimes a_{C,D,E})}
\&
\end{forcedcentertikzcd}}$
\end{landscape}
\item For all $A,B,C$ objects of $\mathcal{C}$, the pasting diagram
$$\begin{forcedcentertikzcd}[column sep = small]
&
( A \otimes B) \otimes C
\arrow{rr}{a_{A,B,C}}
\arrow[Rightarrow, shorten >=1.1cm, shorten <=0.5cm]{rd}[swap]{ a_{r_A,id_B,id_C}}
&
&
A \otimes (B \otimes C)
\arrow[Rightarrow, shorten >=2.8cm, shorten <=0.4cm]{ddr}[near start]{ \mu_{A,B\otimes C}}
&
\\
((A \otimes I) \otimes B) \otimes C
\arrow{ru}{(r_A \otimes id_B) \otimes id_C}
\arrow{rd}[swap]{a_{A,I,B} \otimes id_C}
\arrow{rr}[swap]{a_{A\otimes I,B,C}}
&&
(A \otimes I) \otimes (B \otimes C)
\arrow{rr}[swap]{a_{A,I,B\otimes C}}
\arrow[bend left=18]{ru}[near start]{r_A \otimes id_B \otimes id_C}[name=Y,near start]{}
\arrow[bend right=12]{ru}[swap]{r_A \otimes id_{B \otimes C}}[name=T, near start]{}
\arrow[Rightarrow, shorten >=0.0cm, shorten <=0.5cm,from=Y, to=T,"Id_{r_A} \otimes~\otimes_{id_B,id_C}^{-1}"]
\arrow[Rightarrow, shorten >=0.2cm, shorten <=0.1cm]{d}{ \pi_{A,I,B,C}}
&{}&
A \otimes (I \otimes ( B \otimes C))
\arrow{lu}[swap]{id_A \otimes l_{B \otimes C}}
\\
&
(A \otimes ( I \otimes B)) \otimes C
\arrow{rr}[swap]{a_{A,I\otimes B,C}}
&{}
&
A \otimes (( I \otimes B) \otimes C)
\arrow{ru}[swap]{id_A \otimes a_{I,B,C}}
&{}
\end{forcedcentertikzcd}
$$ must be equal to the pasting diagram 
$$\begin{forcedcentertikzcd}[column sep = small]
&
( A \otimes B) \otimes C
\arrow{rr}{a_{A,B,C}}
&
&
A \otimes (B \otimes C)
&
\\
((A \otimes I) \otimes B) \otimes C
\arrow{ru}{(r_A \otimes id_B) \otimes id_C}[name=T]{}
\arrow{rd}[swap]{a_{A,I,B} \otimes id_C}
&{}&
&{}
\arrow[Rightarrow, shorten >=0.2cm, shorten <=1.1cm]{r}{Id_{id_A} \otimes L_{B,C}}
&
A \otimes (I \otimes ( B \otimes C))
\arrow{lu}[swap]{id_A \otimes l_{B \otimes C}}
\\
&
(A \otimes ( I \otimes B)) \otimes C
\arrow{rr}[swap]{a_{A,I\otimes B,C}}
\arrow{uu}[swap,near start]{(id_A \otimes l_B) \otimes id_C}[name=Y, near start]{}
\arrow[Rightarrow, shorten >=0.2cm, shorten <=1.1cm,from=T,to=Y]{}[swap, near end]{\mu_{A,B} \otimes Id_{id_C}}
&
&
A \otimes (( I \otimes B) \otimes C)
\arrow{ru}[swap]{id_A \otimes a_{I,B,C}}
\arrow{uu}[swap,near start]{id_A \otimes( l_B \otimes id_C)}[name=Z, near end]{}
\arrow[Rightarrow, shorten >=1.7cm, shorten <=1.6cm,from=Y,to=Z]{}[near end]{a_{id_A,l_B,id_C}}
&
\end{forcedcentertikzcd}
$$
\item For all $A,B,C$ objects of $\mathcal{C}$, the pasting diagram
$$\begin{forcedcentertikzcd}[column sep = small]
{}
&
( A \otimes B) \otimes C
\arrow{rr}{a_{A,B,C}}
\arrow[Rightarrow, shorten >=1.5cm, shorten <=5.9cm]{rrrd}[near end]{ a_{id_A,id_B,l_C}}
&
&
A \otimes (B \otimes C)
&
\\
((A \otimes B) \otimes I) \otimes C
\arrow{ru}{r_{A \otimes B} \otimes id_C}[name=X]{}
\arrow{rd}[swap]{a_{A,B,I} \otimes id_C}
\arrow{rr}[swap]{a_{A\otimes B,I,C}}[name=Z]{}
&&
(A \otimes B) \otimes (I \otimes C)
\arrow{rr}[swap]{a_{A,B,I\otimes C}}
\arrow[bend left=12]{lu}[near start]{id_{A \otimes B} \otimes l_C}[name=Y,near start]{}
\arrow[bend right=18]{lu}[swap,near start]{(id_A  \otimes id_B )\otimes l_C}[name=T, near start]{}
\arrow[Rightarrow, shorten >=0.0cm, shorten <=0.5cm,from=Y, to=T,near end,"\otimes_{id_A,id_B} \otimes Id_{l_c}"]
\arrow[Rightarrow, shorten >=0.2cm, shorten <=0.1cm]{d}{ \pi_{A,B,I,C}}
&{}&
A \otimes (B \otimes ( I \otimes C))
\arrow{lu}[swap]{id_A \otimes (id_B \otimes  l_{C})}
\\
&
{(A \otimes ( B \otimes I)) \otimes C}
\arrow[Rightarrow, shorten >=0.7cm, shorten <=0.5cm,from=X,to=Z]{}[]{ \mu_{A\otimes B, C}}
\arrow{rr}[swap]{a_{A,B\otimes I,C}}
&{}
&
A \otimes (( B \otimes I) \otimes C)
\arrow{ru}[swap]{id_A \otimes a_{B,I,C}}
&{}
\end{forcedcentertikzcd}
$$ must be equal to the pasting diagram 
$$\begin{forcedcentertikzcd}[column sep = small]
&
( A \otimes B) \otimes C
\arrow{rr}{a_{A,B,C}}
&
&
A \otimes (B \otimes C)
&
\\
((A \otimes B) \otimes I) \otimes C
\arrow{ru}{r_{A \otimes B} \otimes id_C}[name=T]{}
\arrow{rd}[swap]{a_{A,B,I} \otimes id_C}
&{}&
&{}
\arrow[Rightarrow, shorten >=0.2cm, shorten <=1.1cm]{r}{Id_{id_A} \otimes \mu_{B,C}}
&
A \otimes (B \otimes ( I \otimes C))
\arrow{lu}[swap]{id_A \otimes (id_B \otimes l_C)}
\\
&
(A \otimes ( B \otimes I)) \otimes C
\arrow{rr}[swap]{a_{A,B\otimes I,C}}
\arrow{uu}[swap,near start]{(id_A \otimes r_B) \otimes id_C}[name=Y, near start]{}
\arrow[Rightarrow, shorten >=0.2cm, shorten <=1.1cm,from=T,to=Y]{}[swap, near end]{R_{A,B} \otimes Id_{id_C}}
&
&
A \otimes (( B \otimes I) \otimes C)
\arrow{ru}[swap]{id_A \otimes a_{B,I,C}}
\arrow{uu}[swap,near start]{id_A \otimes( r_B \otimes id_C)}[name=Z, near end]{}
\arrow[Rightarrow, shorten >=1.7cm, shorten <=1.6cm,from=Y,to=Z]{}[near end]{a_{id_A,r_B,id_C}}
&
\end{forcedcentertikzcd}
$$
\end{itemize}

\end{definition}

Thus, in order to interpret $MILL$, a bicategory must be at least monoidal, to interpret properly the linear conjunction $\otimes$. In what follows, we will be interested in looking at pseudofunctors that preserve the monoidal structure, whose definition follows, alongside the higher order steps like natural transformations and modifications that preserve the natural structure : 

\begin{definition}
Let $\mathcal{C}, \mathcal{D}$ be two monoidal bicategories. A monoidal pseudofunctor $F : \mathcal{C} \rightarrow \mathcal{D}$ is a pseudofunctor equipped with:
\begin{itemize}
\item a $1-$cell $F_{I}^{\otimes}: I \rightarrow F(I)$
\item a pseudo-natural transformation $F^\otimes$ whose components are of the form : $$F_{A,B}^\otimes : F(A) \otimes F(B) \rightarrow F(A \otimes B)$$
\item three invertible modifications $F^a, F^l, F^r$ whose components are of the form : 
$$
\begin{tikzcd}
(F(A) \otimes F(B)) \otimes F(C)
\arrow{rr}{a_{F(A),F(B),F(C)}}
\arrow{d}{F_{A,B}^{\otimes} \otimes id_{F(A)}}
&&
F(A) \otimes (F(B) \otimes F(C))
\arrow{d}{id_{F(A)} \otimes F_{B,C}^\otimes}
\\
F(A\otimes B) \otimes F(C)
\arrow{d}{F_{A\otimes B, C}^\otimes}
&&
F(A) \otimes F(B \otimes C)
\arrow{d}{F_{A,B\otimes C}^\otimes}
\\
F((A \otimes B) \otimes C)
\arrow{rr}{F(a_{A,B,C})}
\arrow[Rightarrow, shorten >=2.2cm, shorten <=2.1cm]{rruu}[swap]{F^{a}_{A,B,C}}
&&
F(A \otimes (B \otimes C))
\end{tikzcd}$$

$$\begin{tikzcd}
I \otimes F(A)
\arrow{r}{F_{I}^\otimes \otimes id_{F(A)}}
\arrow{d}{l_{F(A)}}[name=A]{}
&
F(I) \otimes F(A)
\arrow{d}{F_{I,A}^\otimes}[name=B]{}
\arrow[Rightarrow, shorten >=1.2cm, shorten <=1.1cm,from=A,to=B]{}[]{F_{A}^l}
&&
F(A) \otimes I
\arrow{r}{id_{F(A)} \otimes F_{i}^\otimes}
\arrow{d}{r_{F(A)}}[name=C]{}
&
F(A) \otimes F(I)
\arrow{d}{F_{A,I}^\otimes}[name=D]{}
\arrow[Rightarrow, shorten >=1.2cm, shorten <=1.1cm,from=C,to=D]{}[near end]{F_{A}^r}
\\
F(A)
&
F(I \otimes A)
\arrow{l}{F(l_A)}
&&
F(A)
&
F(A \otimes I)
\arrow{l}{F(r_A)}
\end{tikzcd}$$

\end{itemize}
such that the following properties are verified :
\begin{itemize}
\item \todo{THE DEMONIC DIAGRAM} For all $A,B,C,D$ objects of $\mathcal{C}$, the following pasting diagram 
$$\begin{tikzcd}
\end{tikzcd}$$
must be equal to the pasting diagram
$$\begin{tikzcd}
\end{tikzcd}$$
\item  For all $A,B$ objects of $\mathcal{C}$, the following pasting diagram 
$$\begin{forcedcentertikzcd}
&&
(F(A) \otimes I) \otimes F(B)
\arrow{rr}{a_{F(A),I,F(B)}}[name=F, near end]{}
\arrow{d}[swap]{(id_{F(A)} \otimes F_{I}^{\otimes}) \otimes id_{F(B)}}
&&
F(A) \otimes (I \otimes F(B))
\arrow{d}{id_{F(A)} \otimes (F_{I}^{\otimes} \otimes id_{F(B)})}
\arrow[bend left=80]{ddd}{id_{F(A)} \otimes l_{F(B)}}[name=A]{}
\\
F(A \otimes I) \otimes F(B)
\arrow{d}[swap]{F^{\otimes}_{A\otimes I, B}}
&&
(F(A) \otimes F(I)) \otimes F(B) 
\arrow{ll}[swap]{F^{\otimes}_{A,I} \otimes id_{F(B)}}
\arrow{rr}[swap]{a_{F(A),F(I),F(B)}}[name=E, near start]{}
\arrow[Rightarrow, shorten >=0.4cm, shorten <=0.6cm,from=F,to=E,]{}[]{a_{id_{F(A)},F^{\otimes}_I,id_{F(B)}}}
&&
F(A) \otimes (F(I) \otimes F(B))
\arrow{d}[swap]{id_{F(A)} \otimes F^{\otimes}_{I,B}}[name=B]{}
\arrow[Rightarrow, shorten >=0.6cm, shorten <=1.0cm,from=A,to=B]{}[swap, near end]{Id_{id_A} \otimes F^{l}_{B}}
\\
F((A \otimes I) \otimes B)
\arrow{rr}{F(a_{A,I,B})}[name=G]{}
\arrow{rrd}[swap]{F(r_A \otimes id_B)}[name=C]{}
\arrow[Rightarrow, shorten >=2.2cm, shorten <=2.1cm,from=E,to=G]{}[]{F_{A,I,B}^{a~-1}}
&&
F(A \otimes (I \otimes B))
\arrow{d}{F(id_A \otimes l_B)}[name=D]{}
\arrow[Rightarrow, shorten >=0.8cm, shorten <=1.0cm,from=D,to=C]{}[swap]{F(\mu_{A,I,B})^{-1}}
&&
F(A) \otimes F(I \otimes B)
\arrow{ll}[swap]{F^{\otimes}_{A,I\otimes B}}
\arrow{d}{id_{F(A)} \otimes F(l_B)}
\\
&&
F(A \otimes B)
&&
F(A) \otimes F(B)
\arrow{ll}{F_{A,B}^{\otimes}}
\arrow[Rightarrow, shorten >=1.2cm, shorten <=1.1cm]{llu}[swap]{F^{\otimes}_{id_{A},l_B}}
\end{forcedcentertikzcd}$$
must be equal to the pasting diagram
$$\begin{forcedcentertikzcd}
(F(A) \otimes F(I)) \otimes F(B)
\arrow{d}[swap]{F^{\otimes}_{A,I} \otimes id_{F(B)}}[name=A]{}
&&
F(A \otimes I) \otimes F(B)
\arrow{rr}{a_{F(A),I,F(B)}}[]{}
\arrow{ll}[swap]{(id_{F(A)} \otimes F^{\otimes}_{I}) \otimes id_{F(B)}}[]{}
\arrow{rd}[swap]{r_{F(A)} \otimes id_{F(B)}}[name=B]{}
&&
F(A) \otimes (I \otimes F(B))
\arrow{ld}{id_{F(A)} \otimes l_{F(B)}}[name=C]{}
\\
F(A \otimes I) \otimes F(B)
\arrow{d}[swap]{F^{\otimes}_{A\otimes I, B}}[]{}
\arrow{rrr}[swap]{F(r_A) \otimes id_{F(B)}}[name=D, near end]{}
&&
&
F(A) \otimes F(B)
\arrow{d}{F^{\otimes}_{A,B}}[]{}
&
\\
F((A\otimes I) \otimes B)
\arrow{rrr}[swap]{F(r_A \otimes id_B)}[name=E, near start]{}
&&
&
F(A \otimes B)
&
\arrow[Rightarrow, shorten >=3.3cm, shorten <=3.0cm,from=B,to=A]{}{F^{r}_{A}\otimes Id_{id_{F(B)}}}[]{}
\arrow[Rightarrow, shorten >=1.8cm, shorten <=1.0cm,from=C,to=B,swap]{}{\mu_{F(A),F(B)}^{-1}}[]{}
\arrow[Rightarrow, shorten >=1.0cm, shorten <=1.5cm,from=D,to=E]{}{F^{\otimes}_{r_A,id_B}}[]{}
\end{forcedcentertikzcd}$$
\end{itemize}
\end{definition}

\begin{definition}
Let $\mathcal{C},\mathcal{D}$ be two monoidal bicategories and $F,G: \mathcal{C} \rightarrow \mathcal{D}$ two monoidal pseudofunctors. A monoidal pseudonatural transformation $\gamma: F \Rightarrow G$ is a pseudonatural transformation equipped with:
\begin{itemize}
\item an invertible$2-$cell $$\gamma_{I}^{\otimes}: F_{I}^{\otimes} \circ \gamma_I \Rightarrow G_{I}^{\otimes}$$
\item an invertible modification whose components are of the form : 

$$\begin{tikzcd}
F(A) \otimes F(B)
\arrow{rr}{F^{\otimes}_{A,B}}
\arrow[swap]{d}{\gamma_{A} \otimes \gamma_{B}}
&&
F(A \otimes B)
\arrow{d}{\gamma_{A\otimes B}}
\arrow[Rightarrow, shorten >=0.8cm, shorten <=0.5cm]{dll}{\gamma^{\otimes}_{A,B}}
\\
G(A) \otimes G(B) 
\arrow[swap]{rr}{G^{\otimes}_{A,B}}
&&
G(A \otimes B)
\end{tikzcd}
$$
\end{itemize}
such that the following properties are verified : 
\begin{itemize}
\item For all $A$ object of $\mathcal{C}$, the following pasting diagram $$\begin{tikzcd}
&&
F(A) \otimes I
\arrow{rr}{id_{F(A)} \otimes F_{I}^{\otimes}}
\arrow{lld}{\gamma_A \otimes id_I}
\arrow{d}{r_{F(A)}}[name=B]{}
&&
F(A) \otimes F(I)
\arrow{rrd}{\gamma_A \otimes \gamma_I}
\arrow{d}{F^{\otimes}_{A,I}}[name=A]{}
\arrow[Rightarrow, shorten >=0.8cm, shorten <=0.5cm,from=B, to=A]{F_{A}^r}
\\
G(A) \otimes I
\arrow[Rightarrow, shorten >=0.8cm, shorten <=0.5cm]{rr}{r_{\gamma_{A}}}
\arrow{rrd}{r_{G(A)}}
&&
F(A)
\arrow{d}{\gamma_A}
&&
F(A\otimes I)
\arrow{ll}{F(r_A)}
\arrow[Rightarrow, shorten >=0.8cm, shorten <=0.5cm]{rr}{\gamma^{\otimes}_{A,I}}
\arrow{d}{\gamma_{A\otimes I}}
&&
G(A) \otimes G(I)
\arrow{dll}{G^{\otimes}_{A,I}}
\\
&&
G(A)
\arrow[Rightarrow, shorten >=0.8cm, shorten <=0.5cm]{rru}{\gamma_{r_A}}
&&
G(A\otimes I)
\arrow{ll}{G(r_A)}
\end{tikzcd}
$$
is equal to the pasting diagram : $$\begin{tikzcd}
F(A) \otimes I
\arrow{rr}{id_{F(A)} \otimes F^{\otimes}_I}
\arrow{d}{\gamma_A \otimes id_I}
&&
F(A) \otimes F(I)
\arrow{rrd}{\gamma_A \otimes \gamma_I}
\arrow{d}{\gamma_A \otimes id_{F(I)}}
\\
G(A) \otimes I
\arrow{rr}{id_{G(A)} \otimes F^{\otimes}_I}
\arrow{dd}[near end,name=C]{r_{G(A)}}
\arrow[bend right=20,swap]{rrrr}[name=B]{id_{G(A)} \otimes G^{\otimes}_I}
&&
G(A) \otimes F(I)
\arrow{rr}{id_{G(A)} \otimes \gamma_I}
\arrow[Rightarrow,to=B, shorten >=0.0cm, shorten <=0.0cm]{}{Id_{id_{G(A)}} \otimes \gamma^{\otimes}_I}
&&
G(A) \otimes G(I)
\arrow{dd}[near end,name=A]{G^{\otimes}_{A,I}}
\\
\\
G(A)
&&&&
G(A \otimes I)
\arrow{llll}{G(r_A)}
\arrow[Rightarrow,from=C,to=A, shorten >=2.0cm, shorten <=4.2cm,swap, near end]{}{G^{r}_{A}}
\end{tikzcd}
$$

\item For all $A,B,C$ objects of $\mathcal{C}$, the following pasting diagram $$\begin{forcedcentertikzcd}
(F(A) \otimes F(B)) \otimes F(C)
\arrow{rr}{ F^{\otimes}_{A,B} \otimes id_{F(C)}}
\arrow{d}{a_{F(A),F(B),F(C)}}
&&
F(A \otimes B) \otimes F(C)
\arrow{rr}{F^{\otimes}_{A\otimes B,C}}[name=A]{}
&&
F((A \otimes B) \otimes C)
\arrow{d}{F(a_{A,B,C})}
\\
F(A) \otimes (F(B) \otimes F(C))
\arrow{rr}{id_{F(A)} \otimes F^{\otimes}_{B,C}}[name=B]{}
\arrow{d}{\gamma_A \otimes id_{F(B) \otimes F(C)}}
\arrow[swap,bend right=80]{dd}{\gamma_A \otimes (\gamma_B \otimes \gamma_C)}
&&
F(A) \otimes F(B \otimes C)
\arrow{rr}{F^{\otimes}_{A,B\otimes C}}
\arrow{d}{\gamma_A \otimes id_{F(B \otimes C)}}
\arrow[bend left=80]{dd}{\gamma_A \otimes (\gamma_{B \otimes C})}
&&
F(A \otimes ( B \otimes C))
\arrow{dd}{\gamma_{A \otimes(B \otimes C)}}
\arrow[swap,Rightarrow, shorten >=2.8cm, shorten <=0.5cm, near start]{ddll}{\gamma^{\otimes}_{A,B \otimes C}}
\\
G(A) \otimes (F(B) \otimes F(C))
\arrow{rr}{id_{G(A)} \otimes F^{\otimes}_{B,C}}
\arrow{d}{id_{G(A)} \otimes (\gamma_B \otimes \gamma_C)}
&&
G(A) \otimes F(B \otimes C)
\arrow{d}{id_{G(A)} \otimes \gamma_{B \otimes C}}
\arrow[Rightarrow, shorten >=1.8cm, shorten <=0.5cm]{dll}{\gamma^{\otimes}_{A\otimes B} \otimes Id_{id_{G(C)}}}
\\
G(A) \otimes (G(B) \otimes G(C))
\arrow[swap]{rr}{G^{\otimes}_{A,B\otimes C}}
&&
G(A) \otimes G(B \otimes C)
\arrow[swap]{rr}{G^{\otimes}_{A\otimes B,C}}
&&
G(A \otimes (B \otimes C))
\arrow[Rightarrow, shorten >=1.8cm, shorten <=1.8cm,from=A,to=B]{}{F^{a}_{A,B,C}}
\end{forcedcentertikzcd}
$$
is equal to the pasting diagram : $$\scalebox{0.8}{
\begin{forcedcentertikzcd}[ampersand replacement=\&]
\&
(F(A) \otimes F(B)) \otimes F(C)
\arrow{rr}{F^{\otimes}_{A,B} \otimes id_{F(C)}}
\arrow[swap]{d}{(\gamma_A \otimes \gamma_B) \otimes id_{F(C)}}
\arrow[swap,bend right=80]{dd}{(\gamma_A \otimes \gamma_B) \otimes \gamma_C}
\arrow[swap,bend right=40]{dddl}{a_{F(A),F(B),F(C)}}
\&\&
F(A \otimes B) \otimes F(C)
\arrow{rr}{F^{\otimes}_{A\otimes B,C}}
\arrow{d}{\gamma_{A\otimes B} \otimes id_{F(C)}}
\arrow[bend left=80]{dd}{\gamma_{A \otimes B} \otimes \gamma_C}
\arrow[Rightarrow, shorten >=1.1cm, shorten <=0.8cm,swap]{dll}{Id_{id_{F(A)}} \otimes \gamma_{B,C}^{\otimes}}
\&\&
F((A \otimes B) \otimes C)
\arrow{dr}{F(a_{A,B,C})}
\arrow{dd}{\gamma_{(A\otimes B) \otimes C}}
\arrow[Rightarrow, shorten >=2.8cm, shorten <=0.5cm, swap,near start]{ddll}{\gamma_{A\otimes B, C}^{\otimes}}
\\
\&
(G(A) \otimes G(B)) \otimes F(C)
\arrow[swap]{rr}{G^{\otimes}_{A,B} \otimes id_{F(C)}}
\arrow[swap]{d}{id_{G(A) \otimes G(B)} \otimes \gamma_C}
\arrow[Rightarrow, shorten >=1.3cm, shorten <=2.2cm,swap]{ddl}{a_{\gamma_A,\gamma_B,\gamma_C}}
\&\&
G(A \otimes B) \otimes F(C)
\arrow{d}{id_{G(A \otimes B)} \otimes \gamma_C}
\&\&\&
F(A \otimes (B \otimes C))
\arrow{ddl}{\gamma_{A \otimes (B \otimes C)}}
\arrow[Rightarrow, shorten >=0.8cm, shorten <=0.5cm,swap]{dl}{\gamma_{\alpha_{A,B,C}}}
\\
\&
(G(A) \otimes G(B)) \otimes G(C) 
\arrow{rr}{G^{\otimes}_{A,B} \otimes id_{G(C)}}
\arrow{d}{a_{G(A),G(B),G(C)}}
\&\&
G(A \otimes B) \otimes G(C)
\arrow{rr}{G^{\otimes}_{A\otimes B,C}}
\&\&
G((A \otimes B) \otimes C)
\arrow{d}{G(a_{A,B,C})}
\arrow[Rightarrow, shorten >=2.5cm, shorten <=3.4cm,swap]{dllll}{G^{a}_{A,B,C}}
\\
F(A) \otimes (F(B) \otimes F(C))
\arrow{r}{\gamma_A \otimes (\gamma_B \otimes \gamma_C)}
\&
G(A) \otimes (G(B) \otimes G(C))
\arrow{rr}{id_{G(A)} \otimes G^{\otimes}_{B,C} }
\&\&
G(A) \otimes G(B \otimes C)
\arrow{rr}{G^{\otimes}_{A, B\otimes C}}
\&\&
G(A \otimes(B \otimes C))
\end{forcedcentertikzcd}}$$

\end{itemize}
\end{definition}

\begin{definition}
A monoidal modification $m: \gamma \Rrightarrow \delta : F \Rightarrow G$ between two monoidal pseudonatural transformations $\gamma$ and $\delta$ is a modification verifying the following property : 
$$\begin{tikzcd}
&&
F(I)
\arrow[swap]{dd}[name=B, near start]{}[name=C]{\delta_I}
\arrow[bend left = 80]{dd}[name=D]{\gamma_I}
\\
I
\arrow{rru}{F^{\otimes}_{I}}
\arrow[swap]{rrd}[name=A]{G^{\otimes}_{I}}
&&
&&
=
&
\gamma^{\otimes}_I
\\
&&
G(I)
\arrow[Rightarrow,from = B,to=A, shorten >=0.3cm, shorten <=0.5cm,swap]{}{\delta^{\otimes}_{I}}
\arrow[Rightarrow,from = D,to=C, shorten >=0.3cm, shorten <=0.2cm,swap]{}{m_I}
\end{tikzcd}$$
and, for every object $A,B$ of $\mathcal{C}$, the following diagram
$$
\begin{tikzcd}
F(A) \otimes F(B)
\arrow{rr}{F^{\otimes}_{A,B}}
\arrow{dd}{\gamma_A \otimes \gamma_B}[name=A]{}
\arrow[bend right = 80,swap]{dd}{\delta_A \otimes \delta_B}[name=B]{}
&&
F(A \otimes B)
\arrow[Rightarrow,shorten >=1.0cm, shorten <=1.0cm]{ddll}{\gamma^{\otimes}_{A,B}}
\arrow{dd}{\gamma_{A \otimes B}}
\\
\\
G(A) \otimes G(B)
\arrow{rr}{G^{\otimes}_A,B}
&&
G(A \otimes B)
\arrow[Rightarrow, from=A,to=B,shorten >=0.3cm, shorten <=0.5cm]{}{m_A \otimes m_B}
\end{tikzcd}
$$
is equal to the diagram
$$
\begin{tikzcd}
F(A) \otimes F(B)
\arrow{rr}{F^{\otimes}_{A,B}}
\arrow{dd}{\delta_A \otimes \delta_B}
&&
F(A \otimes B)
\arrow[Rightarrow,shorten >=1.0cm, shorten <=1.0cm]{ddll}{\delta^{\otimes}_{A,B}}
\arrow[swap]{dd}{\delta_{A \otimes B}}[name=A]{}
\arrow[bend left = 80]{dd}{\gamma_{A \otimes B}}[name=B]{}
\\
\\
G(A) \otimes G(B)
\arrow{rr}{G^{\otimes}_A,B}
&&
G(A \otimes B)
\arrow[Rightarrow, from=B,to=A,shorten >=0.3cm, shorten <=0.5cm]{}{m_{A \otimes B}}
\end{tikzcd}
$$
\end{definition}


\todo{Symmetric stuff, a couple of hundred new diagrams}



Let us now get back to the remaining rules of $MILL$ that  we have not interpreted yet, ie the ones involving the linear implication $\multimap$.  
First things first, through a reasoning similar to what we did for $\otimes$ and supported by the following proof transformation rules, we can state that $\multimap$ must be interpreted as a pseudo-bifunctor, though contravariant in its first argument. 
$$%\scalebox{0.8}{
 \begin{array}{cccc}
\eta \multimap
&
 \prftree[r]{ax}
	{A\multimap B \vdash A \multimap B} 
&
\Rightarrow
&
\prftree[r]{$r-\multimap$}
	{\prftree[r]{$l-\multimap$}
		{\prftree[r]{ax}{A\vdash A}}
		{\prftree[r]{ax}{B \vdash B}}
		{A, A \multimap B \vdash B}
	}
	{A \multimap B \vdash A \multimap B}
\\
\\
\text{ r-$\multimap$  l-$\multimap$  } cut
&
\prftree[r]{cut}
	{\prftree[r]{$r-\multimap$}
		{\prftree{\pi_1}{A,B\vdash C}}
		{B \vdash A \multimap C}
	}
	{\prftree[r]{$l-\multimap$}
		{\prftree{\pi_2}{D \vdash A}}
		{\prftree{\pi_3}{C \vdash E}}
		{D,A\multimap C \vdash E}
	}
	{B,D  \vdash E}
&
\Rightarrow
&
\prftree[r]{cut}
	{\prftree[r]{cut}
		{\prftree{\pi_2}{D \vdash A}}
		{\prftree{\pi_1}{A,B \vdash C}}
		{B,D\vdash C}
	}
	{\prftree{\pi_3}{C\vdash E}}
	{B,D\vdash E}
\\
\\
\text{l-$\multimap$ } lcut
&
\prftree[r]{cut}
	{
	\prftree[r]{$l-\multimap$}
		{\prftree{\pi_1}{A \vdash B}}
		{\prftree{\pi_2}{C \vdash D}}
		{A,B\multimap C \vdash D }
	}
	{\prftree{\pi_3}{D\vdash E}}
	{A,B \multimap C \vdash  E}
&
\Rightarrow
&
\prftree[r]{$l-\multimap$}
	{\prftree{\pi_1}{A \vdash B}}
	{\prftree[r]{cut}
		{\prftree{\pi_2}{C \vdash D}}
		{\prftree{\pi_3}{D\vdash E}}
		{C \vdash E}
	}
	{A,B \multimap C \vdash  E}
\\
\\
\text{l-$\multimap$ } rcut_1
&
\prftree[r]{cut}
	{\prftree{\pi_1}{A\vdash B}}
	{
	\prftree[r]{$l-\multimap$}
		{\prftree{\pi_2}{B \vdash C}}
		{\prftree{\pi_3}{D \vdash E}}
		{B,C\multimap D \vdash  E}
	}
	{A, C \multimap D \vdash E}
&
\Rightarrow
&
\prftree[r]{$l-\multimap$}
	{\prftree[r]{cut}
		{\prftree{\pi_1}{A\vdash B}}
		{\prftree{\pi_2}{B \vdash C}}
		{A \vdash C}
	}
	{\prftree{\pi_3}{D \vdash E}}
	{A, C \multimap D \vdash E}
\\
\\
\text{l-$\multimap$ } rcut_2
&
\prftree[r]{cut}
	{\prftree{\pi_1}{A\vdash B}}
	{
	\prftree[r]{$l-\multimap$}
		{\prftree{\pi_2}{C \vdash D}}
		{\prftree{\pi_3}{B,E \vdash F}}
		{C,B,D\multimap E \vdash  F}
	}
	{C,A, D \multimap E \vdash F}
&
\Rightarrow
&
\prftree[r]{$l-\multimap$}
	{\prftree{\pi_2}{C \vdash D}}
	{\prftree[r]{cut}
		{\prftree{\pi_1}{A\vdash B}}
		{\prftree{\pi_3}{B,E \vdash F}}
		{A,E \vdash F}
	}
{C,A, D \multimap E \vdash F}
\\
\\
\text{r-$\multimap$ } rcut
&
\prftree[r]{cut}
	{\prftree{\pi_1}{A\vdash B}}
	{
	\prftree[r]{$r-\multimap$}
		{\prftree{\pi_2}{B,C \vdash D}}
		{B \vdash  C\multimap D}
	}
	{A \vdash C\multimap D}
&
\Rightarrow
&
\prftree[r]{$r-\multimap$}
	{\prftree[r]{cut}
		{\prftree{\pi_1}{A\vdash B}}
		{\prftree{\pi_2}{B,C \vdash D}}
		{A,C \vdash D}
	}
{A \vdash C\multimap D}
\end{array}%}
$$

\begin{definition}
A biadjunction between two pseudo-functors $F:\mathcal{C} \rightarrow \mathcal{D}$ and $G:\mathcal{D} \rightarrow \mathcal{C}$ is given by a pair of pseudo-natural transformations $\eta: Id_{\mathcal{C}} \rightarrow G F$ and $\epsilon : F G \rightarrow Id_{\mathcal{D}}$ along with two invertible modifications with components :
$$
\begin{tikzcd}
G(D)
\arrow{rr}{\eta_{G(D)}}
\arrow[swap]{rrd}{id_{G(D)}}[name=A]{}
&&
G(F(G(D)))
\arrow{d}{G(\epsilon_D)}[name=B]{}
&&&&
F(C)
\arrow{rr}{F(\eta_C)}
\arrow[swap]{rrd}{id_{F(C)}}[name=C]{}
&&
F(G(F(C)))
\arrow{d}{\epsilon_{F(C)}}[name=D]{}
\\
&&
G(D)
&&&&&&
F(C)
\arrow[Rightarrow,shorten >=0.4cm, shorten <=0.4cm,from=A,to=B]{}{s_D}
\arrow[Rightarrow,shorten >=0.4cm, shorten <=0.4cm,from=D,to=C]{}{t_C}
\end{tikzcd}
$$
such that the following diagram equalities hold for all objects $C$ of $\mathcal{C}$ and $D$ of $\mathcal{D}$ :
$$
\begin{tikzcd}
C
\arrow{rr}{\eta_C}
\arrow{d}{\eta_C}
&&
G(F(C))
\arrow[swap]{d}{G(F(\eta_C))}[name=B]{}
\arrow[bend left=50]{rddd}{id_{G(F(C))}}[name=C]{}
\\
G(F(C))
\arrow{rr}{\eta_{G(F(C))}}
\arrow[swap]{rrrdd}{id_{G(F(C))}}[name=A]{}
&&
G(F(G(F(C))))
\arrow{rdd}{G(\epsilon_{F(C)})}[name=D]{}
&&
=
&&
 Id_{id_C \circ \eta_C}
\\
\\
&&
&
G(F(C))
\arrow[Rightarrow,shorten >=0.4cm, shorten <=0.4cm,from=A,to=D]{}{s_{F(C)}}
\arrow[Rightarrow,shorten >=0.4cm, shorten <=0.8cm,from=B,to=C]{}{G(t_C)}
\end{tikzcd}
$$
$$
\begin{tikzcd}
F(G(D))
\arrow[swap,bend right=50]{dddr}{id_{F(G(D))}}[name=A]{}
\arrow[swap]{ddr}{F(\eta_{G(D)})}[name=B]{}
\arrow{ddrrr}{id_{F(G(D))}}[name=C]{}
\\
\\
&
F(G(F(G(D)))
\arrow{rr}{\epsilon_{F(G(D))}}
\arrow{d}{F(G(\epsilon_D))}[name=D]{}
&&
F(G(D))
\arrow{d}{\epsilon_D}
&&
=
&&
Id_{\epsilon_D \circ id_D}
\\
&
F(G(D))
\arrow{rr}{\epsilon_D}
&&
D
\arrow[Rightarrow,shorten >=0.4cm, shorten <=0.8cm,from=A,to=D]{}{F(s_{D})}
\arrow[swap,Rightarrow,shorten >=0.4cm, shorten <=0.8cm,from=B,to=C]{}{t_{G(D)}}
\end{tikzcd}
$$
\end{definition}

\begin{definition}
A monoidal bicategory $\mathcal{C}$ is monoidal closed if the pseudo-functor $_ \otimes B: \mathcal{C} \rightarrow \mathcal{C}$ has a right biadjoint for all objects $B$ of $\mathcal{C}$.
\end{definition}
\subsection{the case of MALL}
\begin{definition}
A bicategory $\mathcal{C}$ is cartesian if the diagonal pseudofunctor $\Delta_n:\mathcal{C} \rightarrow \mathcal{C}^n$ has a right biadjoint. 
\end{definition}
\begin{definition}
A cartesian bicategory $\mathcal{C}$ is cartsian closed if the pseudo-functor $_ \times B: \mathcal{C} \rightarrow \mathcal{C}$ has a right biadjoint for all objects $B$ of $\mathcal{C}$.
\end{definition}

\section{ILL and linear exponential comonads}
\begin{definition}
A pseudo-comonoid $A$ in a monoidal bicategory $\mathcal{C}$ is given by an object $A$ of the bicategory, equipped with :
\begin{itemize}
\item a $1-$cell $J: A \rightarrow I$
\item a $1-$cell $P:C \rightarrow C \otimes C$
\item three invertible $2-$ cells 
$$\begin{tikzcd}
A \otimes A
\arrow{d}{ P \otimes id_A}
&&
A
\arrow{ll}{P}
\arrow{rr}{P}
&&
A \otimes A
\arrow{d}{id_A \otimes P}
\\
(A \otimes A) \otimes A
\arrow{rrrr}{a_{A,A,A}}
\arrow[Rightarrow, shorten >=1.8cm, shorten <=2.0cm]{rrrru}{\alpha}
&&&&
A \otimes (A \otimes A)
\end{tikzcd}
$$
$$
\begin{tikzcd}
A
\arrow{rr}{P}
\arrow[swap]{rrd}{l_{A}^{-1}}[name=A]{}
&&
A \otimes A
\arrow{d}{J \otimes id_A}[name=B]{}
\\
&&
I \otimes A
\arrow[Rightarrow, from=A,to=B, shorten >=0.3cm, shorten <=0.3cm]{}{\lambda}
\end{tikzcd}
$$
$$
\begin{tikzcd}
A
\arrow{rr}{P}
\arrow[swap]{rrd}{r_{A}^{-1}}[name=A]{}
&&
A \otimes A
\arrow{d}{  id_A\otimes J}[name=B]{}
\\
&&
A \otimes I
\arrow[Rightarrow, from=A,to=B, shorten >=0.3cm, shorten <=0.3cm]{}{\rho}
\end{tikzcd}
$$
\end{itemize}
such that the following properties are verified : 
The diagram $$
\begin{tikzcd}
A \otimes I \otimes A
&&
A \otimes A \otimes A
\arrow[swap]{ll}{id_A \otimes J \otimes id_A}
\\
A \otimes A
\arrow{u}{id_A \otimes l_{A}^{-1}}[name=A]{}
\arrow[Rightarrow, swap,shorten >=0.5cm, shorten <=0.5cm]{rr}{\alpha}
\arrow[swap]{urr}{id_A \otimes P}[name=B]{}
&&
A \otimes A
\arrow[swap]{u}{P \otimes id_A}
\\
A
\arrow{u}{P}
\arrow[swap]{urr}{P}
\arrow[Rightarrow, from=A,to=B, shorten >=0.5cm, shorten <=0.5cm]{}{id_A \otimes \lambda}
\end{tikzcd}
$$
is equal to the diagram
$$
\begin{tikzcd}
A \otimes I \otimes A
&&
A \otimes A \otimes A
\arrow[swap]{ll}{id_A \otimes J \otimes id_A}
\\
A \otimes A
\arrow{u}{r_{A}^{-1}}[name=A]{}
\arrow[swap]{urr}{P \otimes id_A}[name=B]{}
\\
A
\arrow{u}{P}
\arrow[Rightarrow, from=A,to=B, shorten >=0.5cm, shorten <=0.5cm]{}{\rho \otimes id_A }
\end{tikzcd}
$$
and the following diagram :
$$
\begin{tikzcd}
A \otimes A \otimes A \otimes A 
&&
A \otimes A \otimes A
\arrow[swap]{ll}{id_A \otimes id_A \otimes P}
&&
A \otimes A
\arrow[swap]{ll}{id_A \otimes P}
\arrow[Rightarrow, swap,shorten >=0.5cm, shorten <=0.5cm]{lld}{\alpha}
\\
A \otimes A \otimes A
\arrow{u}{P \otimes id_A \otimes id_A}
&&
A \otimes A
\arrow{u}{P \otimes id_A}
\arrow{ll}{id_A \otimes P}
\arrow[Rightarrow, swap,shorten >=0.0cm, shorten <=0.0cm]{d}{\alpha}
&&
A
\arrow{u}{P}
\arrow{ll}{P}
\arrow{lld}{P}
\\
&&
A \otimes A
\arrow{ull}{P \otimes id_A}
\end{tikzcd}
$$

must be equal to the diagram : 
$$
\begin{tikzcd}
A \otimes A \otimes A \otimes A 
&&
A \otimes A \otimes A
\arrow[swap]{ll}{id_A \otimes id_A \otimes P}
\arrow[swap,Rightarrow, swap,shorten >=0.0cm, shorten <=0.0cm]{d}{Id_{id_A} \otimes \alpha }
&&
A \otimes A
\arrow[swap]{ll}{id_A \otimes P}
\arrow{lld}{id_A \otimes P}
\arrow[Rightarrow, shorten >=1.2cm, shorten <=1.2cm]{lddl}{\alpha}
\\
A \otimes A \otimes A
\arrow{u}{P \otimes id_A \otimes id_A}
&&
A \otimes A \otimes A
\arrow[swap]{ull}{id_A \otimes P \otimes id_A}
\arrow[Rightarrow, swap,shorten >=0.5cm, shorten <=0.5cm]{ll}{\alpha \otimes Id_{id_A}  }
&&
A
\arrow{u}{P}
\arrow{lld}{P}
\\
&&
A \otimes A
\arrow{ull}{P \otimes id_A}
\arrow{u}{P \otimes id_A}
\end{tikzcd}
$$
\end{definition}
\begin{definition}
A pseudo-comonad on a bicategory $\mathcal{C}$ is given by a pseudo-functor $F: \mathcal{C} \rightarrow \mathcal{C}$, two pseudo-natural transformations $v:F \Rightarrow Id_{\mathcal{C}}$ and $n: F \Rightarrow F \circ F$ called the counit and comultiplications, and three invertible modifications $ \alpha,\lambda,\rho$ whose components are given by the following diagrams : \\
\begin{tikzcd}
F(A)
\arrow{rr}{n_A}
\arrow{d}{n_A}
&&
F(F(A))
\arrow{d}{n_{F(A)}}
\arrow[Rightarrow, shorten >=0.2cm, shorten <=0.2cm]{dll}{\alpha_A}
\\
F(F(A))
\arrow{rr}{F(n_A)}
&&
F(F(F(A)))
\end{tikzcd}
\begin{tikzcd}
&&
F(A)
\arrow[swap]{lld}{id_{F(A)}}[name=A]{}
\arrow{rrd}{id_{F(A)}}[name=C]{}
\arrow{d}{n}[name=B]{}
\arrow[Rightarrow, from=C,to=B, shorten >=0.5cm, shorten <=0.5cm]{}{\rho_A }
\arrow[Rightarrow, from=B,to=A, shorten >=0.5cm, shorten <=0.5cm]{}{\lambda_A }
\\
F(A)
&&
F(F(A))
\arrow{ll}{F(v_A)}
\arrow{rr}{v_{F(A)}}
&&
F(A)
\end{tikzcd}
such that the following properties are verified : 
\todo{}
\end{definition}
\begin{definition}
The Kleisli bicategory $\mathcal{C}_F$ associated to a pseudo-comonad F on a bicategory $\mathcal{C}$ is defined as having the same $0$-cells as $\mathcal{C}$, and whose hom-category $\mathcal{C}_F (A,B)$ is given by $\mathcal{C}(F(A),B)$. \\
The composition in $\mathcal{C}_F$ of $f:F(A)\rightarrow B$ and $g:F(B) \rightarrow C$ is defined as $$g \circ_F f := g \circ f(F) \circ n_A$$.
This definition can easily be extended to provide the required composition functors. The identities in $\mathcal{C}_F$ are given by the components of the counit of the comonad. The $2$-isomorphisms and additional properties of the bicategory come directly from the pseudo-comonad structure.
\end{definition}
\begin{definition}
A linear exponential pseudo-comonad 
\end{definition}
And thus we have the following definition for a model of $ILL$ : 
\begin{definition}
A bicategorical model of ILL is a symmetric monoidal bicategory with a linear exponential pseudo-comonad
\end{definition}
We can then recover one of the most interesting properties from categorical models of linear logic in the bicategorical settings with the following theorem :
\begin{theorem}
Let $\mathcal{C},!$ a bicategorical model of ILL. Then the Kleisli bicategory $\mathcal{C}_!$ is cartesian closed.
\end{theorem}
\begin{proof}
The proof will proceed in two steps, first we will show that $\mathcal{C}_!$ is cartesian, and then that it is cartesian closed.
\begin{itemize}
\item To prove that $\mathcal{C}_!$ is cartesian, we have to prove that the diagonal pseudofunctors $\Delta^n : \mathcal{C}_!  \rightarrow \mathcal{C}^{n}_!$  have a right pseudo adjoint, given by $\Pi^n : A_1,...A_n \rightarrow A_1 \& ... \& A_n$. 
\item Next, to prove that $\mathcal{C}_!$ is cartesian closed, we also have to prove the existence of a family of  pseudo-adjunctions in $\mathcal{C}_!$, this time between the pseudo-functors   $\Pi_B : A \rightarrow B \& A$ and $ \Rightarrow_B: A \rightarrow !B \multimap A$  indexed by $B$ object of $\mathcal{C}$.\\

Let us start with a full description of both functors on morphisms and $2-$morphisms: 
First, for $\Pi_B : \mathcal{C}_! \rightarrow \mathcal{C}_!$, we need the effect on morphisms to produce, from a morphism $f: A_1 \rightarrow_{\mathcal{C}_!} A_2$, a morphism $\Pi_B(f) :B \& A_1 \rightarrow_{\mathcal{C}_!} B \& A_2$ , meaning, when looking at the original category $\mathcal{C}$, we need to turn a morphism $f: !A_1 \rightarrow A_2$ into a morphism $\Pi_B(f) :!(B \& A_1) \rightarrow B \& A_2$. This is obtained through the following construction : 
$$\begin{tikzcd}
!(B \& A_1)
\arrow{rr}{s_{B,A_1}}
&&
!B \otimes !A_1
\arrow{rr}{id_{!B} \otimes n_{A_1}}
&&
!B \otimes !!A_1
\arrow{d}{id_{!B} \otimes !f}
\\
B \& A_2
&&
!(B \& A_2)
\arrow{ll}{v_{B \& A_2}}
&&
!B \otimes !A_2
\arrow{ll}{s^{-1}_{B,A_2}}
\end{tikzcd}$$

A consequence of this construction is that the effect on a $2-$morphism $\tau: f \Rightarrow g$ is very obvious, applying the initial $2-$morphism at the only point where the morphism $f$ appears, with the remaining space between morphisms being filled by identities.\\

Next, for $\Rightarrow_B: \mathcal{C}_! \rightarrow \mathcal{C}_!$, we need the effect on morphisms to produce, from a morphism $f: A_1 \rightarrow_{\mathcal{C}_!} A_2$, a morphism $\Rightarrow_B(f) :!B \multimap A_1 \rightarrow_{\mathcal{C}_!} !B \multimap A_2$ , meaning, when looking at the original category $\mathcal{C}$, we need to turn a morphism $f: !A_1 \rightarrow A_2$ into a morphism $\Rightarrow_B(f) :!(!B \multimap A_1) \rightarrow !B \multimap A_2$. This is obtained through the following construction :
$$\begin{tikzcd}
!(!B\multimap A_1)
\arrow{rr}{\eta^{\otimes,\mathcal{C}}_{!B,!(!B\multimap A_1)}}
&&
!B \multimap (!B \otimes !(!B\multimap A_1))
\arrow{rr}{!B\multimap s_{B,!B\multimap A_1}}
&&
!B \multimap !(B \& (!B\multimap A_1))
\arrow{d}{!B \multimap n_{B \& (!B\multimap A_1)}}
\\
&&&&
!B \multimap !!(B \& (!B\multimap A_1))
\arrow{d}{!B \multimap ! s^{-1}_{B, (!B\multimap A_1)}}
\\
!B \multimap !A_1
\arrow{d}{!B\multimap f}
&&
!B \multimap !(!B \otimes !B\multimap A_1)
\arrow{ll}{!B \multimap !(\epsilon^{\otimes, \mathcal{C}}_{!B,A_1}}
&&
!B \multimap !(!B \otimes !(!B\multimap A_1))
\arrow{ll}{!B \multimap ! (id_{!B} \otimes v_{!B\multimap A_1})}
\\
!B \multimap A_2
\end{tikzcd}$$
 
In a similar way, the effect on $2-$morphisms is easy to build.\\

Let us now proceed step by step to build this pseudo-adjunction. First, we need candidates for the pseudo-natural transformations $\eta: id_{\mathcal{C}_!} \rightarrow ~ \Rightarrow_B (\Pi_B)$ and $\epsilon : \Pi_B (\Rightarrow_B )~\rightarrow ~id_{\mathcal{C}_!}$.\\
So we need $\eta_A : A \rightarrow_{\mathcal{C}_!} !B \multimap (B \& A)$ and $\epsilon_A : B \& (!B \multimap A) \rightarrow_{\mathcal{C}_!} A $. They are obtained through the following constructions using the monoidal closed structure of the underlying bicategory $\mathcal{C}$ :

for $\eta_A :$
$$
\begin{tikzcd}
!A
\arrow{rr}{\eta^{\mathcal{C}}_{!B,!A}}
&&
!B \multimap (!B \otimes !A)
\arrow{rr}{!B \multimap s^{-1}_{B,A}}
&&
!B \multimap !(B \& A)
\arrow{rr}{!B \multimap v_{B \& A}}
&&
!B \multimap (B \& A)
\end{tikzcd}$$

and for $\epsilon_A :$
$$
\begin{tikzcd}
!(B \& (!B \multimap A))
\arrow{rr}{s_{B,!B \multimap A}}
&&
!B \otimes !(!B \multimap A)
\arrow{rr}{id_{!B} \otimes v_{!B\multimap A}}
&&
!B \otimes !B \multimap A
\arrow{rr}{\epsilon^{\mathcal{C}}_{!B,A}}
&&
A
\end{tikzcd}$$

This handles the morphism components of the two transformations, we now need to describe their $2-$morphism components which need to be of the form:

$$\begin{tikzcd}
A_1
\arrow{rr}{\eta_{A_1}}
\arrow{d}{f}
&&
!B \multimap (B \& A_1)
\arrow{d}{\Rightarrow_B ( \Pi_B(f))}
\\
A_2
\arrow{rr}{\eta_{A_2}}
\arrow[Rightarrow, shorten >=1.0cm, shorten <=1.1cm]{rru}{\eta_f}
&&
!B \multimap (B \& A_2)
\end{tikzcd}~~
\begin{tikzcd}
B \& !B \multimap A_1
\arrow{rr}{\epsilon_{A_1}}
\arrow{d}{\Pi_B(\Rightarrow_B(f))}
&&
A_1
\arrow{d}{ f}
\\
B \& !B \multimap A_2
\arrow{rr}{\epsilon_{A_2}}
\arrow[Rightarrow, shorten >=1.0cm, shorten <=1.1cm]{rru}{\epsilon_f}
&&
A_2
\end{tikzcd}$$
Those are diagrams in $\mathcal{C}_{!}$, and thus, when expanding them to describe $\eta_f$ and $\epsilon_f$ properly, we will need to remember to use the specific composition of $\mathcal{C}_{!}$.
Let us now look at the full diagram for $\eta$ 
\end{itemize}
\end{proof}
%\begin{definition}\label{definition/pseudofunctor}
%A pseudofunctor is a mapping between bicategories $\mathcal{C}$ and $\mathcal{D}$ where the usual functorial equations $F(f \circ g) = F(f) \circ F(g) $ and $F(Id_A) = Id_{F(A)}$ are only valid up to natural bijectve 2-morphisms in $\mathcal{D}$. 
%\end{definition}
%
%\begin{definition}\label{definition/laxmonoidal}
%Let $(\mathcal{C},\otimes_{\mathcal{C}},1_{\mathcal{C}})$ and $(\mathcal{D},\otimes_{\mathcal{D}},1_{\mathcal{D}})$ be two monoidal bicategories. A lax monoidal pseudofunctor between them is given by : 
%\begin {itemize}
%\item a pseudofunctor $F:\mathcal{C} \rightarrow \mathcal{D}$
%\item a morphism $\epsilon : 1_{\mathcal{D}} \rightarrow F(1_{\mathcal{C})}$
%\item for every pair of objects $A,B \in \mathcal{C}$, a natural transformation $\mu_{A,B}: F(A) \otimes_{\mathcal{D}} F(B) \rightarrow F(A \otimes_{\mathcal{C}} B)$%pseudonatural ???
%\end{itemize}
%satisfying the following conditions :
%\begin{itemize}
%\item associativity : For every triple of objects $A,B,C \in \mathcal{C}$, the following diagram commutes : 
%$$
%\xymatrix @-1.2pc {
%(F(A) \otimes_{\mathcal{D}} F(B)) \otimes_{\mathcal{D}} F(C)  
%\ar[dd]_-{\mu_{A,B} \otimes id}
%\ar[rrrr]_-{a^{\mathcal{D}}_{F(A),F(B),F(C)}}
%&&&& 
%F(A) \otimes_{\mathcal{D}} (F(B) \otimes_{\mathcal{D}} F(C))  
%\ar[dd]_-{id \otimes \mu_{B,C}}
%\\
%\\
%F(A \otimes_{\mathcal{C}} B) \otimes_{\mathcal{D}} F(C)  
%\ar[dd]_-{\mu_{A\otimes B, C}}
%&&&& 
%F(A) \otimes_{\mathcal{D}} F(B \otimes_{\mathcal{C}} C)  
%\ar[dd]_-{\mu_{A,B\otimes C}}
%\\
%\\
%F((A \otimes_{\mathcal{C}} B) \otimes_{\mathcal{C}} C)  
%\ar[rrrr]_-{F(a^{\mathcal{C}}_{A,B,C})}
%&&&& 
%F(A \otimes_{\mathcal{C}} (B \otimes_{\mathcal{C}} C))  
%}
%$$
%where the two morphisms $a^\mathcal{C}, a^\mathcal{D}$ denote the associators of the two tensor products.
%
%\item unality : For every object $A \in \mathcal{C}$, the following diagram and its right symmetry both commute : 
%$$
%\xymatrix @-1.2pc {
%1_\mathcal{D} \otimes_\mathcal{D} F(A) 
%\ar[dd]_-{l^{\mathcal{D}}_{F(A)}}
%\ar[rrrr]_-{\epsilon \otimes id}
%&&&& 
%F(1_\mathcal{C}) \otimes_\mathcal{D} F(A) 
%\ar[dd]_-{\mu_{1_\mathcal{C}, A}}
%\\
%\\
%F(A)  
%&&&& 
%F(1_\mathcal{C} \otimes_{\mathcal{C}} A)  
%\ar[llll]_-{F(l^{\mathcal{C}_{A}})}
%}
%$$
%where $l^\mathcal{C}, l^{\mathcal{D}}$ denote the left unitors of the two tensor products.
%
%\end{itemize}
%\end{definition}
%
%\begin{definition}\label{definition/pseudonatural}
%Let $F,G$ be two pseudofunctors between two bicategories $\mathcal{C}$ and $\mathcal{D}$. A pseudonatural transformation $\phi : F \rightarrow G$ is given by : 
%\begin{itemize}
%\item for every object $A$ of $\mathcal{C}$, a morphism $\phi(A): F(A) \rightarrow G(A)$ of $\mathcal{D}$.
%\item for every morphism $f:A \rightarrow B$ of $\mathcal{C}$, a bijective $2-$morphism $\phi(f): \phi(B) \circ F(f) \Rightarrow G(f) \circ \phi(A)$
%\end{itemize}
%such that
%\begin{itemize}
%\item $\phi$ respects composition of morphisms, meaning that we have an equivalence between 
%$$(\phi(A) \triangleleft G(f,g) )\cdot(\phi(f) \triangleright G(g)) \cdot(F(f)\triangleleft \phi(g))$$ and 
%$$\phi(g \circ f) \cdot (F(f,g) \triangleright  \phi(C) ),$$ 
%both being $2$-morphisms from 
%$$\phi(C) \circ F(g) \circ F(f) \Rightarrow G(g \circ f) \circ \phi(A),$$ 
%where $\cdot$ is the vertical composition between $2$-morphisms, $\triangleleft, \triangleright$ the two versions of the horizontal composition between a morphism and a $2$-morphism, (also called whiskering), anf $F(f,g):F(g) \circ F(f) \Rightarrow F(g \circ f)$ is the bijective $2$-morphism coming from the pseudofunctor $F$.
%\item $\phi$ respects the identity morphisms, meaning we have an equivalence between
%$$L^{\mathcal{D}}_{\phi(A)} \cdot \epsilon^{F}_{id_A} \triangleright \phi(A) $$ and
%$$R^{\mathcal{D}}_{\phi(A)} \cdot \phi(A) \triangleleft \epsilon^{G}_{id_A} \cdot \phi(id_A) $$
%both being $2$-morphisms from
%$$\phi(A) \circ F(id_A)  \Rightarrow \phi(A) $$
%where $L^{\mathcal{D}}_{\phi(A)}: \phi(A) \circ id_{F(A)} \Rightarrow  \phi(A)$ is the left unitor coming from the bicategory $\mathcal{D}$  and $\epsilon^{F}_{id_A}: F(id_A) \Rightarrow id_{F(A)}$ is the bijective $2$-morphism coming from the pseudofunctor $F$.
%\item $\phi$ is natural in the following sense : for every $2$-morphism $\psi: f \Rightarrow g$ with $f,g:A\rightarrow B$, we have an equivalence between $$\phi(g) \cdot F(\psi) \triangleright \phi(B)$$ and $$\phi(A)\triangleleft G(\psi)\cdot \phi(f).$$
%\end{itemize}
%\end{definition}
%
%\begin{definition}\label{definition/pseudocomonad}
%A fully weak comonad $G$ on a bicategory $\mathcal{C}$ is a pseudofunctor, along with pseudonatural transformations $\delta$ and $\epsilon$ that satisfy the usual laws of a comonad up to natural bijectiive 2-morphisms in $\mathcal{C}$.
%\end{definition}

\end{document}














